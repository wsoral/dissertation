\documentclass[man]{apa6}

\usepackage[utf8]{inputenc}
\usepackage[T1]{fontenc}
\usepackage[MeX]{polski}
\usepackage{todonotes}

\makeatletter
\renewcommand\efloat@iwrite[1]{%
   \immediate\expandafter\protected@write\csname efloat@post#1\endcsname{}}
\makeatother

\newcommand{\rowgroup}[1]{\hspace{-1em}#1}
\newcommand{\tdinline}[1]{\todo[inline]{#1}}

\usepackage{csquotes}
\usepackage[style=apa,sortcites=true,sorting=nyt,backend=biber]{biblatex}
\DeclareLanguageMapping{english}{american-apa}
\addbibresource{biblio.bib}

\usepackage{geometry}
\geometry{letterpaper}
\usepackage{graphicx}

\usepackage{threeparttable}
\usepackage{amsmath}

\title{Jakie aspekty indywidualnego i kolektywnego `Ja' wyznaczają
       pozytywną samoocenę. Znaczenie perspektywy sprawcy i biorcy.}
\shorttitle{Wyznaczniki pozytywnej samooceny}
\author{Wiktor Soral}
\affiliation{Uniwersytet Warszawski}

\abstract{Jakie aspekty indywidualnego i kolektywnego `Ja' wyznaczają pozytywną samoocenę. Znaczenie perspektywy sprawcy i biorcy.}

\keywords{indywidualna samoocena, kolektywna samoocena, perspektywa sprawcy, perspektywa biorcy, sprawczość, wspólnotowość}


\begin{document}
\maketitle


Przekonania na temat siebie stanowią nieodłączny element zachowania człowieka \parencite[np.,][]{bandura1991social, deci2000and, rosenberg1965society, tesser1988toward}. Z jednej strony, przekonania te konstytuują osiągnięcia szkolne i zawodowe, poziom zdrowia, dobre relacjami innymi ludźmi, czy choćby ogólne poczucie szczęścia i zadowolenia z życia. Z drugiej strony, rozmaite doświadczenia życiowe takie jak zdany bądź niezdany egzamin, akceptacja bądź odrzucenie przez grupę, zachowanie się w zgodzie lub niezgodzie z wewnętrznymi standardami kształtują przekonania wobec siebie (schematy Ja) składające się na ogólny obraz Ja \parencite{markus1977self}. \\

To na ile ukształtowany obraz Ja zostanie zaakceptowany lub odrzucony, na ile będzie związany z pozytywną lub negatywną reakcją emocjonalną, przekłada się na poziom globalnej samooceny człowieka \parencite{brown1993self, rosenberg1965society}. Osoby lubiące poszczególne elementy obrazu siebie i uważające poszczególne aspekty Ja za wartościowe charakteryzowały się będą wysokim poziomem samooceny. Natomiast osoby, które nie wartościują pozytywnie różnych aspektów Ja, bądź których postawy wobec poszczególnych aspektów Ja są ambiwalentne lub niepewne charakteryzować się będą niskim poziomem samooceny \parencite{baumeister1989self}.\\

Tak opisana samoocena jest rodzajem sądu społecznego tyle, że dotyczącego własnej osoby. Podobnie jak wszystkie sądy opiera się ona na tych elementach obrazu Ja, które są dostępne w pamięci osoby (\emph{available}) oraz tych które uległy aktywizacji (\emph{accessible}) w momencie dokonywania samooceny \parencite[zob. np., ][]{higgins1996knowledge}. O tak opisanej samoocenie będzie decydowała ocena tych aspektów Ja, które najczęściej ulegają aktywizacji, czyli tych stanowiących centralne aspekty obrazu Ja.\\

Celem niniejszej rozprawy jest określenie, które treści stanowią centralne aspekty obrazu Ja i są głównymi wyznacznikami pozytywnej samooceny. Choć jest to pytanie zasadnicze, na które niejednokrotnie i na różne sposoby próbowano odpowiedzieć \parencite[np.,][]{brambilla2014importance, gebauer2013agency, wojciszke2011self}, dyskusja nad podstawami samooceny wydaje się być nadal nierozstrzygnięta.\\

Brak rozstrzygnięcia wynika pośrednio z faktu niejasności co do funkcji samooceny i ogólniej co do funkcji Ja. Istnieje co najmniej kilka różnych teorii mówiących o tym, do czego służy człowiekowi samoocena \parencite[np., ][]{leary1995self, pyszczynski2004people}, ale jak na razie brak jest jednej teorii -- unifikującej. Przeglądowi teorii opisujących funkcjonalne aspekty samooceny poświęcony zostanie pierwszy rozdział tej rozprawy.\\

Na pytanie o wyznaczniki samooceny nie można też odpowiedzieć bez głębszego zrozumienia, czym jest tożsamość człowieka, i że na Ja składa się nie tylko poczucie odrębności od innych, ale również poczucie wspólnoty z innymi \parencite[np.,][]{brewer1996we}. W związku z tym na obraz Ja składają się nie tylko aspekty właściwe jednostce, ale również takie które są właściwe jednostce jako członkowi grupy społecznej lub grupie jako całości. Dotychczasowe prace badawcze nad wyznacznikami samooceny w niewielkim stopniu próbowały radzić sobie z tą wielopoziomowością tożsamości i obrazu Ja. To zagadnienie zostanie przedstawione w drugim rozdziale rozprawy, ale będzie się pojawiało również w innych jej częściach. W szczególności rozróżnienie na indywidualne i kolektywne aspekty Ja posłuży do częściowego wyjaśnienia, dlaczego w dotychczasowych pracach trudno było o konsensus, co do podstawowych wyznaczników pozytywnej samooceny.\\

Mówiąc o podstawowych wyznacznikach pozytywnej samooceny łatwo popaść w pułapkę wielowymiarowości. Istnieje ogrom rozmaitych cech, które mogą być używane zarówno do oceny siebie jak i innych. Jednak wśród badaczy postrzegania społecznego istnieje silny konsensus mówiący, że ten ogrom cech jest w znacznej mierze tłumaczony przez dwa podstawowe wymiary \parencite{judd2005fundamental, fiske2007universal}. Pierwszy z tych wymiarów odnosi się do funkcjonowania w sytuacjach zadaniowych (\emph{sprawczość}), natomiast drugi do funkcjonowania w sytuacjach społecznych \emph{wspólnotowość}. Te dwa wymiary postrzegania zostaną omówione w trzecim rozdziale rozprawy.\\

Czwarty rozdział rozprawy stanowi próba połączenia koncepcji dotyczących samooceny, tożsamości oraz treści postrzegania społecznego, której efektem jest sformułowanie przewidywań odnośnie tego, które aspekty indywidualnego i kolektywnego Ja wyznaczają pozytywną samoocenę. Jako podstawę teoretyczną dla tych przewidywań przyjęto model dwóch perspektyw - sprawcy vs. biorcy \parencite{abele2014communal}, który pozwala określić kiedy w sądach społecznych większe znaczenie mają treści sprawcze, a kiedy treści wspólnotowe. Konkretnie w tym rozdziale podjęta zostanie próba nakreślenia relacji pomiędzy tożsamością indywidualną vs. kolektywną a perspektywą sprawcy vs. biorcy.\\

Przewidywania teoretyczne są weryfikowane i omawiane w ostatnich rozdziałach, gdzie zaprezentowano wyniki 4 badań - dwóch korelacyjnych i dwóch eksperymentalnych. Rozprawę zamyka omówienie i dyskusja wyników, przedstawiono tam znaczenie teoretyczne, metodologiczne i praktyczne otrzymanych rezultatów. \\

\newpage
\section{Część teoretyczna}
\subsection{Funkcje samooceny}

`Dlaczego ludzie potrzebują samooceny?' -- takie pytanie postawione przez \textcite{pyszczynski2004people} dobrze odzwierciedla pogląd większości współczesnych badaczy \parencite[zob. również, ][]{bandura1994self, leary1995self}, że samoocena człowieka pełni określone funkcje. Nie jest ona tylko dodatkiem do ludzkiego zachowania, lecz elementem niezbędnym dla prawidłowego działania. Jako tak istotny element, jest ona szczególnie chroniona przez procesy związane z Ja, które pomagają utrzymywać jej wysoki poziom \parencite{greenwald1980totalitarian}. Jak zauważają np. \textcite{leary2000nature} nie chodzi jednak tylko o to, żeby mieć wysoki poziom samooceny, która sama w sobie jest tylko rodzajem funkcjonalnego `monitora'.\\

\textcite{leary2000nature} na podstawie przeglądu literatury wymieniają sześć funkcji samooceny. Po pierwsze, samoocena reguluje dobrostan psychiczny i pozytywny afekt -- gdy samoocena wzrasta ludzie odczuwają pozytywne i przyjemne emocje, natomiast gdy samoocena spada ludzie odczuwają nieprzyjemne, negatywne emocje. Po drugie, samoocena informuje jak dobrze ludzie radzą sobie z zagrożeniem psychologicznym. Innymi słowy, samoocena jest wysoka gdy ludzie próbują radzić sobie z zagrożeniem i jest niska gdy ludzie unikają psychologicznego zagrożenia. Po trzecie, zgodnie z teorią autodeterminacji \parencite{deci2000and} wysoka samoocena jest miarą autentyczności Ja, czyli tego na ile ludzie w autonomiczny sposób zachowują się w zgodzie ze swoim wewnętrznym potencjałem. Po czwarte, istotą samooceny może być podtrzymywanie wysokiego statusu w relacjach z innymi osobami lub ramach grup społecznych. W takim rozumieniu dążenie do podtrzymywania wysokiej samooceny można zredukować do dążenia do podtrzymywania swojej dominacji nad innymi. Po piąte, samoocena pełni rolę bufora chroniącego przed świadomością własnej śmiertelności. W końcu szósta funkcja, postulowana przez \textcite{leary2000nature}, opisuje samoocenę jako wskaźnik dopasowania do grupy -- socjometr. Zgodnie z tą koncepcją niska samoocena mogłaby sygnalizować zagrożenie wykluczeniem społecznym, np. gdy osoba nie przestrzega standardów danej społeczności.\\

Biorąc te wszystkie koncepcje pod uwagę, wiedzę dotyczącą Ja można najogólniej rozumieć, jako rodzaj systemu monitorującego zachowanie człowieka  i informującego, kiedy kontynuować powziętą czynność, a kiedy danej czynności zaprzestać \parencite{higgins1996self}. W tym rozdziale zostaną przedstawione trzy obszary, w których samoocena działa jako monitor ludzkiego zachowania. Pierwszy, z tych obszarów obejmuje sytuacje zadaniowe, związane z implementacją intencji i sprawowaniem kontroli. Drugi obszar odnosi się do funkcjonowania w sytuacjach społecznych i grupowych, w których celem nadrzędnym jest uzyskanie aprobaty grupy i uniknięcie wykluczenia społecznego. Trzeci obszar opisujący zachowanie w szerszym kontekście społecznym i kulturowym, odwołuje się do potrzeby afirmacji siebie i wykraczania poza postrzeganie siebie tylko jako jednostkę.\\

\subsubsection{Samoocena w sytuacjach zadaniowych}

Poziom samooceny jest jednym z najczęściej cytowanych korelatów osiągnięć szkolnych \parencite{hansford1982relationship, davies1999reading} i zawodowych \parencite{judge2001relationship}. To, że sukces podnosi samoocenę, a porażka prowadzi do spadku samooceny wydaje się oczywiste. Mniej oczywiste jest jednak to, iż wysoka samoocena może bardziej przyczyniać się do sukcesów niż samoocena niska \parencite{bandura1982self}. Te obie zależności można zrozumieć jeżeli przyjmie się, że samoocena jest zasobem wspomagającym w ustanawianiu celów i realizacji zadań.\\

Jeżeli przyjąć, że samoocena jest zasobem wspomagającym w realizacji celów, to jest ona w takim razie bliska pojęciu własnej skuteczności zaproponowanemu przez \textcite{bandura1977self, bandura1994self}. Postrzegana własna skuteczność odnosi się do przekonań osoby, dotyczących jej możliwości kierowania biegiem zdarzeń, tak aby radzić sobie z antycypowanymi wyzwaniami. \textcite{bandura1982self} wskazuje, że wysoki poziom przekonań o własnej skuteczności sprzyja lepszemu wykonywaniu rozmaitych zadań, radzeniu sobie z fobiami, stresem i uzależnieniami, czy bardziej wytrwałemu podążaniu ścieżką kariery. Choć pojęcie samooceny nie jest równoznaczne pojęciu własnej skuteczności (ta pierwsza odnosi się do ogólnych przekonań, natomiast ta druga do przekonań w ramach danej sytuacji lub zadania) to należy zaznaczyć, że według nowszych prac przekonania o sobie, jako o kimś zdolnym do wywierania pożądanego wpływu na rzeczywistość, stanowią istotny element ogólnego poczucia własnej wartości \parencite{tafarodi2001two}.\\

Biorąc pod uwagę tą koncepcję, samoocena stanowi podstawę planowanego działania. Ludzie unikają działań, które przekraczają ich możliwości i podejmują się działań o adekwatnym poziomie trudności \parencite{bandura1977self}. Samoocena wysoka skłania do podjęcia działania i próby radzenia sobie z zagrożeniami, natomiast samoocena niska związana jest ze skłonnością do unikania działania i ucieczki przed zagrożeniem \parencite{bednar1989self}. Wyniki uzyskane przez \textcite{mcfarlin1984knowing} wykazały też, że osoby o wysokiej samoocenie są bardziej wytrwałe w dążeniu do celów i dłużej starają się rozwiązywać problemy przed którymi stają \parencite[zob. również ][]{baumeister1993ego}. Wytrwałość w dążeniu do celów może być oczywiście bezproduktywna i przez to dezadaptatywna, niekiedy lepiej poddać się i skupić swoje wysiłki na innych celach. Jak wskazały jednak nowsze badania \parencite{di2002self}, przy ponoszeniu wielokrotnych porażek to osoby o wysokiej samoocenie szybciej wykrywają bezproduktywność wytrwałości i szybciej skupiają swoją uwagę na innych celach. W tych samych badaniach wykazano, że osoby o wysokiej samoocenie są lepiej `skalibrowane', tj. dostrzegają lepiej niż osoby o niskiej samoocenie, czy ich zachowanie przynosi założone rezultaty czy nie.\\

Również charakter motywacji jest tym co odróżnia osoby o wysokiej samoocenie od osób o samoocenie niskiej \parencite{baumeister1985self}. Celem osób o wysokiej samoocenie jest doskonalenie własnych zalet, dążenie do tego aby stawać się w coraz lepszym i zyskiwać nowe kompetencje. Z kolei, celem osób o niskiej samoocenie jest zaradzenie swoim słabościom, po to aby móc radzić sobie z codziennymi problemami. O ile osoby o wysokiej samoocenie czerpią wewnętrzną motywację z osiągniętych sukcesów, to w przypadku osób o niskiej samoocenie ta motywacja jest efektem poniesionych porażek. Nie znaczy to jednak, że osoby o wysokiej samoocenie gorzej odczuwają porażki. Przeciwnie, poniesione porażki prowadzą u nich wręcz do większego skupienia poznawczego na pozytywnych aspektach Ja \parencite{dodgson1998self}. Lepsze radzenie sobie z porażkami u osób o wysokiej samoocenie może wynikać również z tego, że lepiej oceniają one efektywność własnego działania lub nawet czasami ulegają one złudzeniu własnej skuteczności \parencite{alloy1979judgment}.\\

Podsumowując, samoocena odgrywa znaczącą rolę w sytuacjach zadaniowych. Informuje ona o adekwatności posiadanych wewnętrznych zasobów do poradzenia sobie z zadaniami i realizacji własnych aspiracji. Co więcej, przekonanie o wysokiej samoocenie może sprzyjać radzeniu sobie z zadaniami przekraczającymi własne możliwości i przez to być głównym źródłem rozwoju własnych kompetencji.

\subsubsection{Samoocena w interakcjach społecznych}

Ostracyzm, poczucie bycia wykluczonym lub ignorowanym przez innych, jest jednym z najbardziej uniwersalnych fenomenów, których może doświadczyć człowiek. Obserwowane na poziomie mózgu efekty wykluczenia społecznego przypominają te pojawiające się w momencie wystąpienia bólu fizycznego \parencite{eisenberger2003does}. Wykluczeniu towarzyszy wzrost aktywności przedniego zakrętu kory obręczy (\emph{anterior cingulate cortex}) -- istotnego elementu systemu monitorującego wystąpienie konfliktu pomiędzy automatyczną reakcją i aktualnymi celami podmiotu \parencite{bush2000cognitive}. Również \textcite{williams2007ostracism} wskazuje, że pierwszą reakcją na wykluczenie społeczne jest automatyczna reakcja bólu. Po niej pojawiają się zagrożenia dla poczucia przynależności, samooceny, poczucia kontroli i sensu, a także nasilenie emocji smutku i gniewu. Następnie, w stadium refleksyjnym dokonywana jest ocena sytuacji: ocena źródeł i powodów wykluczenia, oraz indywidualnych predyspozycji, które mogły przyczynić się do ostracyzmu. Biorąc pod uwagę szereg nieprzyjemnych konsekwencji bycia wykluczonym \parencite[patrz również][]{leary1990responses}, łatwo zrozumieć że dążenie do oddalenia tego zagrożenie stanowi jedną z podstawowych motywacji człowieka.\\

W swoich pomysłowych badaniach \textcite{leary1995self} pokazali, że oczekiwania co do wpływu określonych zachowań na reakcję innych osób są zbieżne z oczekiwaniami do wpływu tych zachowań na samoocenę, tj. zachowania mogące prowadzić do odrzucenia społecznego są również postrzegane jako takie, które mogą przyczyniać się do spadku samooceny. Co więcej, wykazali oni istotny związek samooceny z poczuciem wykluczenia, tzn. osoby o wyższej samoocenie czuły się w większym stopniu akceptowane przez innych, natomiast osoby o niskim poziomie samooceny czuły się bardziej ostracyzowane. Uzyskiwane wyniki wskazywały, że być może dążenie do wysokiej samooceny nie jest dążeniem do niej samej, ale dążeniem do czegoś jeszcze, czego samoocena jest tylko wskaźnikiem.\\

Omawiane badania stały się podstawą niezwykle popularnej teorii samooceny jako socjometru. \textcite{leary2000nature} definiują socjometr jako wewnętrzny monitor informujący na ile osoba jest ceniona (lub nie) jako partner relacji społecznych lub jako przedstawiciel grup społecznych. Tak rozumiana samoocena monitoruje jakość relacji w jakie uwikłana jest osoba, i motywuje do podjęcia działań, które mogłyby zapobiec odrzuceniu przez innych. Ponadto, samoocena jako socjometr działa nie tylko w odniesieniu do aktualnego zachowania, ale odnosi się również do ogólnego oczekiwania, że jest i będzie się akceptowanym przez innych. Najogólniej ujmując samoocena jest niezbędna, aby móc zaspokajać podstawową potrzebę przynależności \parencite{baumeister1995need}.\\

Zgodnie z teorią socjometru zagrożenie samooceny będzie skłaniało do podjęcia działań odbudowujących podstawę relacji społecznych, i np. ze skłonnością do prezentowania się pozytywnie na wartościowanych przez innych wymiarach. Inna możliwość głosi jednak, że podstawą samooceny jako socjometru jest nie tyle monitorowanie jakości relacji społecznych, co raczej monitorowanie własnego statusu i prestiżu społecznego w ramach tych relacji \parencite{barkow1980prestige}. Innymi słowy, w tym podejściu potrzeba przynależności jest zaspokajana, nie tyle przez ukazanie siebie jako wartościowego partnera relacji społecznych, co raczej przez dominowanie innych osób w relacjach interpersonalnych. Co więcej, jak wykazały niedawne badania \parencite{zeigler2013status}, samoocena nie tylko monitoruje, ale również sygnalizuje własny status w relacjach społecznych, tzn. inni ludzie na podstawie jedynie obserwacji zazwyczaj dość trafnie są w stanie ocenić na ile podmiot cechuje się wysoką lub niską samooceną.\\

W końcu, warto, również zauważyć, że według niedawnych analiz samoocena przyczynia się również do wzrostu postrzeganego wsparcia ze strony partnerów relacji społecznych \parencite{marshall2014self}, ale niekoniecznie jest tak że wsparcie społeczne podnosi poziom samooceny. Tą zależność można zrozumieć, jeżeli przyjmie się że istotą bycia w grupie i otrzymywania wsparcia od jej członków jest nie podnoszenie samooceny, ale raczej ogólne podnoszenie dobrostanu i jakości życia. Samoocena w tym przypadku jedynie wspomaga w otrzymaniu wsparcia od innych.\\

Podsumowując, w myśl przedstawionych koncepcji bez samooceny niemożliwe byłoby utrzymywanie harmonijnych relacji z innymi ludźmi. Osoby nie byłyby w stanie w odpowiedni sposób modyfikować swoich zachowań, aby móc siebie wzajemnie akceptować i być częścią grupy. Stąd niemożliwe byłoby wytworzenie niewielkich grup, które stałyby się podstawą większych zbiorowości oraz całych kultur.

\subsubsection{Samoocena w kontekście grupowym i kulturowym}

Wedle jednej z najbardziej popularnych teorii opisujących funkcje samooceny -- teorii opanowani trwogi \parencite{pyszczynski2004people} -- samoocena jest wytworem kultury człowieka. Samoocena została wytworzona przez człowieka jako bufor przeciwko wszechogarniającej go wizji własnej śmiertelności. Teoria opanowania trwogi postuluje, że człowiek jako zwierzę pragnie uchronić się od własnej śmierci. Jednak tym, co odróżnia człowieka od innych zwierząt, jest świadomość, że pomimo wszelkich pragnień i starań nie będzie on w stanie uchronić się od swojego ostatecznego losu, w końcu umrze. Zdaniem autorów teorii opanowania trwogi, samoocena została ukształtowana przez człowieka w tym samym czasie, w którym zaczął on uświadamiać sobie nieuniknioność śmierci.\\

Aby lepiej zrozumieć, czym według autorów teorii opanowania trwogi jest samoocena, należy wyjaśnić w jaki sposób jej wysoki poziom miałby redukować lęk przed śmiercią. W ujęciu omawianej teorii, koncepcją bardzo bliską samooceny jest kulturowy światopogląd, czyli ``wytworzone przez człowieka i podzielane symboliczne koncepcje rzeczywistości zapewniające poczucie sensu, porządku i trwałości egzystencji'' \parencite[][, s. 436]{pyszczynski2004people}. Ten światopogląd jest źródłem rozmaitych wartości i standardów. Jeżeli osoba, żyje zgodnie ze światopoglądem, który uważa za powszechnie podzielany upewnia ją to, co do poczucia własnej wartości i przyczynia się do wzrostu jej samooceny. W takim ujęciu samoocena informuje o przynależności do pewnego kontekstu kulturowego, który pozwala na uzyskanie symbolicznej nieśmiertelności.\\

Od momentu powstania, szereg badań dostarczył poparcia dla omawianej teorii. Wykazano, że wysoka samoocena przyczynia się do spadku ogólnego poziomu lęku oraz lęku związanego z oczekiwaniem bolesnego szoku \parencite{greenberg1992people}, wysoka samoocena przyczynia się również do wzrostu przewidywanej długości swojego życia \parencite{greenberg1993effects} oraz do spadku dostępności poznawczej myśli związanych ze śmiercią \parencite{harmon1997terror}.\\

Wykazano również, że aktywizacja myśli o śmierci prowadzi do zwiększonej motywacji do podwyższenia samooceny \parencite{greenberg1992people} oraz do zwiększonej motywacji do obrony wyznawanego światopoglądu, np. poprzez bardziej negatywne reakcje na treści krytykujące wyznawane wartości kulturowe \parencite{harmon1997terror} lub poprzez wykluczanie osób mogących zagrażać światopoglądowi grupowemu \parencite{castano2004case}. Co jednak istotne, skłonność do obrony światopoglądu znika u osób, u których eksperymentalnie podniesiono poziom samooceny \parencite{harmon1997terror}. Tą zależność można zrozumieć, jeżeli przyjmie się że obrona światopoglądu kulturowego oraz dążenie do wysokiej samooceny są konsekwencją tej samej motywacji. Według omawianej teorii, tą motywacją jest właśnie buforowanie myśli o własnej śmiertelności.\\

Podsumowując, rolą samooceny jest buforowanie myśli o własnej śmierci. Wyznając podzielane wartości kulturowe ludzie upewniają się co do własnej wartościowości, i to pozwala im powstrzymać wszechogarniającą trwogę. W tym kontekście samoocena ma szereg różnych implikacji dla przewidywania zachowania człowieka. Może ona prowadzić do większej skłonności podporządkowywania się normom społecznym, może przyczyniać się do wielu chlubnych zachowań (np. większej skłonności do pomagania innym), ale może być również przyczyną moralnej transgresji obserwowanej nie tylko u jednostki, ale w ramach całej kultury lub grupy społecznej.

\newpage

\subsection{Tożsamość człowieka a samoocena indywidualna i kolektywna}

W poprzedniej części opisano rozmaite funkcje samooceny. Pokazano, że samoocena może spełniać inne funkcje w zależności od sytuacji lub kontekstu: Inna jest funkcja samooceny w sytuacjach zadaniowych, inna w sytuacjach interakcji z innymi osobami lub grupami społecznymi, a jeszcze inna gdy samoocenę umieścić w kontekście kulturowym. Te rozmaite sytuacje i konteksty wiążą się z odmiennymi poziomami analizy, a co jeszcze bardziej istotne mogą wiązać się z różnymi reprezentacjami Ja \parencite{brewer1996we} oraz innymi motywacjami.\\

Jak wskazują klasyczne teorie psychologiczne, zachowanie człowieka kierowane jest nie tylko potrzebami autonomii i sprawowania kontroli nad biegiem zdarzeń \parencite[np.,][]{wortman1975responses, deci2000and}, ale również niezwykle podstawową potrzebą przynależności do grup społecznych \parencite{baumeister1995need}. Biorąc to pod uwagę, zrozumiałe jest że człowiek mógł wykształcić różne mechanizmy pozwalające mu monitorować poziom zaspokojenia tych potrzeb, oraz sygnalizujące wystąpienie ich deprywacji. Niewątpliwie Ja (\emph{self}) jest jednym z mechanizmów pełniących taką rolę \parencite[patrz,][]{higgins1989self, higgins1996self}. W niniejszej części wykazane zostanie, że na tożsamość i Ja składają się nie tylko komponenty związane z motywacjami o charakterze indywidualnym, ale również z motywacjami o charakterze zbiorowym.\\

\subsubsection{Różne ujęcia tożsamości i Ja}

Dla porządku należy wskazać, że idea wielopoziomowości w myśleniu o Ja pojawiła się już u Williama Jamesa \parencite{james1890principles}, którego zresztą uznaje się za autora pojęcia Ja. James uważał, że Ja jest złożone z czterech elementów. Na pierwszy z nich, Ja materialne, składają się obiekty najbliższe człowiekowi i zarazem najsilniej wartościowane, jego ciało, ubrania, najbliższa rodzina i dom. Na drugi element, Ja społeczne, składają się relacje i interakcje człowieka z innymi ludźmi oraz z całym społeczeństwem. Trzeci element, Ja duchowe (\emph{spiritual Self}) jest kategorią obejmującą sumienie człowieka, jego moralną wrażliwość, poczucie autonomii i wolnej woli. W końcu, ostatni element, czyste Ego (\emph{pure Ego}), jest tym komponentem, który rozpoznaje treść własnych myśli i łączy je ze sobą w świadomości, oddzielając przy tym treści uznawane za niewłasne.\\

Współcześnie najbardziej wpływowe koncepcje omawiające ideę wielopoziomowości Ja oparte są na teorii tożsamości społecznej \parencite{tajfel1986social} oraz jej kontynuacji, teorii autokategoryzacji \parencite{turner1987rediscovering, turner1994self}. Jedno z podstawowych założeń tych teorii głosi, że Ja jest złożone z różnych kategorii. W ramach tych kategorii można wyróżnić dwa komponenty: tożsamość osobistą (\emph{personal identity}) oraz tożsamość społeczną (\emph{social identity}). Tożsamość osobista to ta kategoria Ja definiująca osobę jako unikalną i odróżnialną od innych przedstawicieli grupy własnej. Z kolei, zgodnie z definicją \textcite{tajfel1981human}, tożsamość społeczna to ``ta część obrazu Ja wynikająca z jego przynależności do grupy społecznej (lub grup), posiadająca dla danej osoby szczególne znaczenie wartościujące i emocjonalne''(s. 255). Tożsamość społeczna odnosi się do tych aspektów grupy własnej, które odróżniają ją od grupy obcej (lub grup).\\

Przedstawione teorie postulują, że obie tożsamości stanowią naturalne aspekty egzystencji człowieka. Co jednak istotne, postulują one również, że obie tożsamości są ze sobą niejako w konflikcie. Jest on spowodowany tym, że w momencie pojawienia się wyrazistej kategorii społecznej i dokonywania porównań międzygrupowych, dochodzi do procesu depersonalizacji \parencite[zob. np.,][]{turner1994self}, polegającego na tym, że osoba zaczyna w mniejszym stopniu postrzegać siebie jako unikalną, a w większym stopniu jako zamiennego (\emph{interchangeable}) przedstawiciela grupy własnej. Innymi słowy, wzbudzenie tożsamości społecznej sprawia, że mniej wyrazista staje się kategoria tożsamości osobistej. Wzbudzenie tożsamości społecznej prowadzi do szeregu różnych konsekwencji wśród których można wymienić nie tylko zwiększoną skłonność do faworyzowania swoich i dyskryminowania obcych \parencite{tajfel1986social}, ale również większą skłonność do podporządkowania się normom grupowym \parencite{reicher1995social}. \\

Wśród polskich autorów, podobna do teorii autokategoryzacji koncepcja pojawia się w pracach \textcite{jarymowicz1994poznawcza}. Według autorki tożsamość osobista to subsystem wiedzy o Ja, na który składają się atrybuty spostrzegane jako najbardziej charakterystyczne i dystynktywne dla Ja. Z kolei tożsamość społeczna jest subsystemem wiedzy o sobie i innych, na który składają się atrybuty spostrzegane jako najbardziej charakterystyczne i zarazem wyróżniające kategorię My spośród innych kategorii społecznych. Idąc dalej autorka wyróżnia 3 rodzaje My: \emph{my grupowe}, skupione na najbliższych, będących w bezpośredniej relacji (twarzą w twarz), \emph{my kategorialne},
skupione na bardziej abstrakcyjnych kategoriach, pochodzących od etykiet językowych, np. my Polacy, oraz \emph{my atrybucyjne}, gdzie układem odniesienia dla tworzenia kategorii my są właściwości Ja, np. my miłośnicy jazzu.\\

Zarówno tożsamość osobista jak i tożsamość społeczna są związane z motywacją do pozytywnego wartościowania siebie i uzyskania tzw. pozytywnej dystynktywności \parencite{tajfel1986social}. Należy jednak zaznaczyć, że postrzeganie siebie w kategoriach zarówno indywidualnych jak i kolektywnych jest również efektem procesów motywacyjnych. Jak zauważa \textcite{brewer1991social} ludzie są motywowani zarówno do postrzegania siebie jako podobnych, jak i różniących się od innych \parencite[patrz również,][]{brewer2007importance}. Tworząc więzi z grupami społecznymi zaspokajają potrzebę bycia podobnym do innych, ale zarazem zaburzają potrzebę odróżniania się od innych. Postrzegając siebie w wąskich, indywidualnych kategoriach, zaspokajają potrzebę odróżniania się od innych, i jednocześnie zaburzają potrzebę bycia podobnymi. Popychani tymi dwiema motywacjami, ostatecznie definiują siebie w kategoriach, które pozwalają na osiągnięcie optymalnej dystynktywności. Warto zaznaczyć, że również w tym ujęciu aspekty indywidualne i kolektywne Ja są równie ważnymi, choć konkurującymi ze sobą składowymi ludzkiej tożsamości.\\

Różni inni autorzy dokonują nieco bardziej szczegółowych podziałów, traktując jednak przy tym teorię tożsamości społecznej jako podstawę dla własnych teorii. Przykładowo \textcite{cheek1989identity} wyróżnia: tożsamość osobistą, tożsamość społeczną, i tożsamość zbiorową. Według cytowanego autora tożsamość osobista odnosi się do osobistych wartości, celów i emocji. Tożsamość społeczna, inaczej niż podejściu tożsamości społecznej, odnosi się do Ja w stosunku do innych osób -- popularności, atrakcyjności, lub reputacji Ja. W końcu, tożsamość zbiorowa, podobnie jak w podejściu tożsamości społecznej odnosi się do tych aspektów Ja związanych rasą, przynależnością etniczną, wyznawaną religią, i ogólnie poczucia przynależności do pewnej wspólnoty.\\

Podobną kategoryzację proponują \textcite{brewer1996we}, gdzie wyróżniają trzy podobne sposoby opisywania Ja. Ja osobiste odnosi się do tego co odróżnia Ja od innych, co czyni osobę unikalną. Ja relacyjne odwołuje się do powiązań ze znaczącymi innymi, i do przyjmowanych ról w diadycznych relacjach. Natomiast, ja zbiorowe jest najbliżej pojęciu tożsamości społecznej proponowanemu w podejściu tożsamości społecznej.\\

W końcowej części niniejszego przeglądu należy wskazać, że ogromne znaczenie dla formowania się tożsamości ma fakt bycia osadzonym w pewnym kontekście kulturowym. W swojej przełomowej pracy \textcite{markus1991culture} wskazują, że osoby osadzone w kulturze indywidualistycznej (np. w USA) formują Ja o zdecydowanie innym charakterze niż osoby osadzone w kulturze kolektywistycznej (np. w Japonii). U tych pierwszych dochodzi do formowania się Ja, które autorzy określają jako niezależne, natomiast w tej drugiej dochodzi do formowania się Ja określanego jako współzależne. Ja niezależne jest odseparowane od kontekstu, osadzone w pewnych ramach i stabilne. Jego główną rolą jest realizacja wewnętrznego potencjału, możliwość ekspresji i wyrażania siebie. W ramach Ja niezależnego inni to osobne podmioty, w oparciu o które dokonuje się porównań społecznych i walidacji przekonań o sobie. Z kolei, ja współzależne jest silnie powiązane z kontekstem, a przez to elastyczne i labilne. Jego główną rolą jest dopasowanie się do innych, określenie właściwej roli społecznej, i realizacja interesu innych osób. W ramach Ja współzależnego inni stanowią część Ja, w oparciu o których dokonuje się samodefinicji i określenia kim się jest. Łatwo zauważyć podobieństwa tej koncepcji do pozostałych przedstawionych poprzenio. Pomimo, że koncepcja \textcite{markus1991culture} dotyczy różnic międzykulturowych, wydaje się że jest ona niezwykle pomocna w zrozumieniu tego, co może być uniwersalne dla różnych poziomów tożsamości człowieka.

\subsubsection{Kolektywna samoocena i jej funkcje}

W niniejszej części wykazano, że na tożsamość osoby składają się nie tylko aspekty właściwe dla niego jako jednostki, ale również te właściwe dla osoby jako przedstawiciela grupy społecznej. Co więcej, szereg dowodów wskazuje na to, że ludzie są motywowani nie tylko do utrzymywania pozytywnego obrazu siebie jako jednostki \parencite[patrz,][]{greenwald1980totalitarian}, ale również do utrzymywania pozytywnego obrazu zbiorowości do których należą \parencite{tajfel1986social}. Idąc dalej tym tokiem rozumowania, ludzie mogą różnić się nie tylko w kwestii przekonań co do wartościowości indywidualnych aspektów Ja, ale również w kwestii przekonań co do wartościowości grup i kategorii społecznych z którymi się identyfikują. Ten pierwszy aspekt jest związany z poziomem indywidualnej samooceny, natomiast ten z drugi z poziomem kolektywnej samooceny \parencite{crocker1990collective}.\\

Idea rozróżniania pomiaru indywidualnej i kolektywnej samooceny została zapoczątkowana prawdopodobnie pod wpływem wyników badań \textcite{crocker1985prejudice} sugerujących, że osoby o wysokiej jak i niskiej indywidualnej samoocenie w podobnym stopniu przejawiają faworyzację grupy własnej. Taki wynik stał oczywiście w sprzeczności z podstawowym postulatem teorii tożsamości społecznej \parencite{tajfel1986social}, głoszącym że celem faworyzowania swoich (i dyskryminowania obcych) jest uzyskanie pozytywnego obrazu własnej osoby\footnote{Przesłanka wyprowadzona z teorii tożsamości społecznej głosiła, że o niskiej samoocenie będą w większym stopniu faworyzowały grupę własną}. Zaproponowany podział stanowił więc jedno z rozwiązań istotnego problemu, z którym borykali się twórcy teorii tożsamości społecznej, dotyczącego motywacyjnej roli samooceny jako wyznacznika uprzedzeń \parencite[zob. np.,][]{abrams1988comments}. Badania raportowane przez \textcite{crocker1990collective} wykazały, że to nie indywidualna, ale raczej kolektywna samoocena jest podstawą efektów faworyzacji postulowanych przez teorię tożsamości społecznej, i wskazały na użyteczność nowego konstruktu.\\

Zgodnie z propozycją \textcite{luhtanen1992collective} na kolektywną samoocenę składa się kilka różnych aspektów. Po pierwsze, jest to osobiste przekonanie o wartościowości kategorii lub grupy społecznej, której jest się przedstawicielem. Po drugiej, jest to również przekonanie, że dana grupa lub kategoria społeczna jest uważana za wartościową również przez innych. Po trzecie, na kolektywną samoocenę składa się przekonanie o wartościowości siebie jako członka danej grupy, tj. przekonanie, że osoba może \emph{dać coś z siebie} dla dobra grupy. W końcu, elementem kolektywnej samooceny jest poczucie, że grupa własna jest istotnym elementem obrazu Ja. Niewątpliwie, takie niejednorodne ujęcie kolektywnej samooceny utrudnia zrozumienie jej statusu ontologicznego. Mimo to, opracowana na podstawie tej koncepcji Skala Kolektywnej Samooceny \parencite{luhtanen1992collective} okazała się użytecznym i popularnym narzędziem stosowanym w społecznej psychologii relacji międzygrupowych (na dzień 19.01.2017 artykuł prezentujący skalę ma 2564 cytowania, na podstawie Google Scholar).\\

Podsumowując, jak wykazano w niniejszej części tożsamość człowieka i Ja można ujmować na wiele różnych sposobów. Różne czynniki mogą prowadzić do aktywizacji, w ramach tożsamości, kategorii związanych z jej indywidualnymi bądź kolektywnymi aspektami. Indywidualne i kolektywne aspekty Ja zdają się stanowić część codziennej egzystencji wszystkich ludzi. Oceny tych różnych aspektów nie zawsze muszą ze sobą korelować, a co najważniejsze nie zawsze prowadzą do tych samych efektów. W związku z tym, aby móc lepiej zrozumieć zachowanie człowieka, należy uwzględniać zarówno samoocenę indywidualną, jak i samoocenę kolektywną.\\

\newpage
\subsection{Podstawowe wymiary postrzegania społecznego}

Rozmaite przeszłe doświadczenia oraz nabywane informacje tworzą treść wiedzy o świecie społecznym, w tym także o Ja -- indywidualnym i kolektywnym. Wedle jednego z najbardziej wpływowych modeli pamięci \parencite{anderson1983spreading} wiedza ta kodowana jest w postaci sieci powiązanych ze sobą jednostek pamięciowych. Z kolei, siła powiązań pomiędzy różnymi jednostkami jest również efektem uczenia się pod wpływem przeszłych doświadczeń. Te powiązania przyczyniają się rozpływania się aktywizacji jednej jednostki pamięciowej na inne jednostki powiązane z tą pierwszą. Przykładowo, zachowanie się w zgodzie z wewnętrznymi standardami moralnymi przyczynia się do kodowania w pamięci informacji o Ja jako osobie `uczciwej', ale również `sprawiedliwej'. Kategoria `sprawiedliwa' może być z kolei w efekcie ciągłej ekspozycji skojarzona z inną kategorią, np. `prawa'. W konsekwencji, przyszłe przywołanie kategorii `prawa' będzie prowadziło do aktywizacji nie tylko tej kategorii, ale również do aktywizacji kategorii `uczciwa' i `sprawiedliwa'.\\

Biorąc pod uwagę to, w jaki sposób informacje kodowane są w pamięci, łatwiej jest zrozumieć dlaczego wiedza płynąca z przeszłych doświadczeń może tworzyć nadrzędne struktury lub wymiary. Wśród badaczy poznania społecznego panuje znaczny konsensus co do tego, że różne cechy używane do opisu świata społecznego, tj. postrzegania siebie, innych osób, grup społecznych lub całych kultur, tworzą dwa podstawowe wymiary percepcji społecznej \parencite{fiske2007universal, judd2005fundamental}. Pierwszy z tych wymiarów odnosi się do funkcjonowania w sytuacjach społecznych, natomiast drugi odnosi się do funkcjonowania w sytuacjach zadaniowych. Te dwa wymiary określane były przez różnych autorów jako wspólnotowość i sprawczość \parencite{abele2007agency}, dobroć/słabość społeczna i intelektualna \parencite {rosenberg1968multidimensional}, korzystność dla Innych i korzystność dla Ja \parencite{peeters1992evaluative}, moralność i kompetencje \parencite{wojciszke2005morality}, lub ciepło i kompetencje \parencite{fiske2002model}.\\

W literaturze można znaleźć wiele dowodów wskazujących na odmienną rolę tych dwóch wymiarów w postrzeganiu innych osób lub grup. \textcite{wojciszke2009two} wykazali, że sądy o wspólnotowości decydują o lubieniu innych, natomiast sądy dotyczące sprawczości decydują o szacunku wobec innych. Do podobnych konkluzji prowadzą badania \textcite{fiske1999dis} nad treścią stereotypów, w których pokazano że wymiar ciepła jest związany z postrzeganą współzależnością grup społecznych (grupy współzależne do grupy własnej są bardziej lubiane), natomiast wymiar kompetencji jest związany z postrzeganym statusem grup (grupy o wysokim statusie są bardziej szanowane). Ciepło i kompetencje są podstawą modelu tzw. BIAS map \parencite{cuddy2007bias}, obrazującego w jaki sposób te dwa podstawowy wymiary tłumaczą szereg specyficznych emocji i zachowań w stosunku do różnych grup społecznych.\\

Choć sprawczość i wspólnotowość tworzą dwa osobne wymiary \parencite{rosenberg1968multidimensional}, zazwyczaj w percepcji innych osób (umiarkowanie) korelują ze sobą dodatnio \parencite{judd2005fundamental}, tą korelację tłumaczy się dobrze znanym efektem aureoli. Z kolei w percepcji grup społecznych, zazwyczaj obserwuje się ich znaczne ujemne skorelowanie \parencite{fiske1999dis}, czyli grupy postrzegane jako kompetentne są często postrzegane jako zimne (np. Żydzi, azjaci, bankierzy), natomiast grupy postrzegane jako ciepłe są częściej postrzegane jako niekompetentne (np. osoby starsze lub niepełnosprawne). Tą ambiwalencję stereotypów wyjaśnia się czasem podstawową potrzebą uzasadniania systemu społecznego \parencite[patrz,][]{kay2003complementary}.\\

Istnieje szereg dowodów wskazujących, że spośród tych dwóch wymiarów to wspólnotowość stanowi podstawę sądów dotyczących innych osób \parencite[np.,][]{wojciszke1998dominance}, oraz podstawę tworzenia impresji na podstawie wyglądu twarzy \parencite{willis2006first}. Co więcej słowa związane z tym wymiarem są szybciej rozpoznawane w ramach testu decyzji leksykalnych \parencite{ybarra2001young}. Te i inne dowody wskazują na ewolucyjną adaptatywność tego wymiaru. Wspólnotowość jest ważna w ocenie innych, ponieważ stanowi podstawę dla decyzji o zbliżaniu się lub ucieczce, współpracy lub walce z innymi. Z kolei sądy dotyczące sprawczości opisywane są jako mające mniejszy potencjał inferencyjny i pojawiające się po sądach dotyczących wspólnotowości \parencite[np.,][]{fiske2007universal}, przynoszą one przede wszystkim informację o potencjale innej osoby lub grupy do wyrządzenia krzywdy lub udzielenia pomocy.\\

Niedawne prace wykazały, że część efektów wspólnotowości można wytłumaczyć tylko jednym z jej aspektów: postrzeganiem innych jako moralnych, dobroczynnych i godnych zaufania \parencite[patrz,][]{brambilla2011looking, goodwin2014moral, brambilla2012you, brambilla2014importance}, a niekoniecznie postrzeganiem ich innych jako ciepłych, towarzyskich i radosnych. Niewątpliwie moralność i ciepło stanowią różne aspekty percepcji, większość współczesnych badaczy zgadza się jednak, że są one elementami wspólnotowości \parencite[patrz np.,][]{leach2007group}\footnote{W angielskiej literaturze rozróżnia się terminy \emph{morality} i \emph{sociability} składające się na wymiar nazywany \emph{warmth}}.\\

Podsumowując, istnieje szereg dowodów na to, że dwa wymiary: sprawczość i wspólnotowość stanowią podstawę spostrzegania świata społecznego\footnote{Warto jednak zauważyć niedawną publikację \textcite{koch2016abc}, wskazującą, że przynajmniej w odniesieniu do stereotypów, model dwóch wymiarów może być nieadekwatny.}. W kolejnych sekcjach przedstawiona zostanie rola każdego z nich w wyznaczaniu pozytywnej samooceny.\\

\subsubsection{Postrzeganie sprawczości Ja a samoocena}

O tym że wymiar sprawczości pełni główną rolę w percepcji Ja i jest podstawowym wyznacznikiem (indywidualnej) samooceny przekonuje zebrany materiał empiryczny. \textcite{wojciszke2010sprawczosc} raportuje wyniki swoich 12 badań prowadzonych nie tylko w Polsce, ale również w USA, Niemczech, Holandii, Anglii, Kolumbii i Japonii. Raportowane badania prowadzono nie tylko na studentach, ale również na pracownikach, urzędnikach, i internautach. Jako miar samooceny używano Skali Samooceny Rosenberga \parencite{rosenberg1965society}, Skali Lubienia Siebie \parencite{tafarodi2001two}, Skali Samooceny jako Stanu \parencite{heatherton1991development}, Skali Samooceny Narcystycznej \parencite{raskin1979narcissistic}, oraz metody pomiaru samooceny utajonej opartej na preferencji swoich inicjałów w stosunku do innych liter \parencite{koole20030nature}. Niewielka meta-analiza przeprowadzona przez autora tej rozprawy na danych raportowanych przez \textcite{wojciszke2010sprawczosc} wskazała, że średnia wielkość związku sprawczości z indywidualną samooceną wynosiła 0,51 [95\% CI: 0,44; 0,56]. Analiza wykresu lejkowego nie wykazała znacznej asymetryczności i nie wskazała, że estymowana wielkość efektu wynika z selektywności raportowania wyników (choć oczywiście analizowane dane zebrano w ramach tylko jednego zespołu badawczego). Dodatkowa analiza zróżnicowania wyników raportowanych przez \textcite{wojciszke2010sprawczosc} wskazała na ich umiarkowaną heterogeniczność \parencite[patrz,][]{higgins2003measuring}, $Q(11) = $ 21,08, $p = $ 0,03, $I^2 = $ 47,8\%, co wydaje się zrozumiałe przy tak różnych próbach i tak różnych podejściach.\\

Dowody na to, że postrzeganie sprawczości grupy własnej decyduje o poziomie kolektywnej samooceny są mniej spójne i mniej usystematyzowane. Za rolą sprawczości przemawia mogą wyniki \textcite{oldmeadow2010social}, sugerujące, że członkowie grupy mogą dążyć do uzyskania pozytywnej dystynktywności poprzez podkreślanie cech grupy własnej związanych z wymiarem kompetencji. Efekty te autorzy obserwują jednak tylko dla grup o wysokim statusie. W przypadku grup o niskim statusie, ich członkowie przyjmują strategie uzyskania pozytywnej dystynktywności oparte na podkreślaniu cech grupy własnej związanych z wymiarem ciepła. Pośrednich dowodów na to, że sprawczość może być istotnym aspektem oceny grupy własnej dostarczają badania \textcite{wojciszke2008primacy} wskazujące, że oceny kompetencji mogą mieć istotne znaczenie przy ocenianiu bliskiego przyjaciela lub przełożonego (o ile interes przełożonego jest ściśle związany z interesem podwładnego, np. jest tak w organizacjach o profilu biznesowym, natomiast w organizacjach o profilu biurokratycznych interes przełożonego i podwładnego nie zawsze idą ze sobą w parze). Jeżeli przyjąć, że przyjaciel lub przełożony, od którego zależy realizacja naszych celów, stanowią aspekty kolektywnego Ja, to te wyniki wskazują, że również kolektywna samoocena może być wyznaczana przez cechy związane z wymiarem sprawczości.

\subsubsection{Postrzeganie wspólnotowości Ja a samoocena}

O tym że ludzie mają niemal uniwersalną skłonność do podkreślania swojej wspólnotowości i widzenia siebie jako osoby bardziej moralnej niż inni przekonuje cały szereg badań \parencite[np.,][]{allison1989being, van1998being, epley2000feeling, balcetis2008collectivists, paulhus1998egoistic}. Jak jednak zauważa Wojciszke \parencite[np.,]{wojciszke2010sprawczosc}, jeżeli ta skłonność jest taka powszechna, to ma ona charakter stałej populacyjnej. Z tego powodu trudno jest określić jaką rolę odgrywałyby sądy o własnej wspólnotowości dla samooceny.\\

Wojciszke \parencite[np.,][]{wojciszke2006perspektywa} przekonuje, że o ile cechy związane z wymiarem wspólnotowości pełnią główną rolę przy ocenie innych osób, nie mają one związku z (indywidualną) samooceną. Istotnie, w omawianych już 12 badaniach dotyczących wyznaczników samooceny, wspólnotowość zdawała się nie odgrywać znaczącej roli jako wyznacznik samooceny. Meta-analiza przeprowadzona przez autora rozprawy wykazała, że średnia wielkość związku wspólnotowości z indywidualną samooceną wynosiła w tych badaniach 0,01 [95\% CI: -0,05; 0,06]. Analiza wykresu lejkowego nie wykazała jego asymetryczności i nie wskazała, że estymowana wielkość efektu wynika z selektywności raportowania wyników. Należy zaznaczyć, że ten efekt był niezwykle spójny w ramach wszystkich 12 raportowanych przez \textcite{wojciszke2010sprawczosc} badań, $Q(11) = $ 6,63, $p = $ 0,82, $I^2 = $ 0,0\%. \\

\textcite{wojciszke2010sprawczosc} interpretuje brak związku sądów o własnej wspólnotowości (moralności) z samooceną jako konsekwencję dwóch możliwych procesów. Po pierwsze, uważa on że oceny własnej moralności mają charakter aprioryczny, tzn. ludzie przyjmują, że są moralni za pewnik, którego nie trzeba udowadniać. Pewnych dowodów na rzecz tej tezy dostarczają nie tylko cytowane badania nad efektem bycia "bardziej świętym niż inni", ale również badania \textcite{wojciszke2010sprawczosc} wskazujące, że oceny własnej moralności są o wiele wyższe niż oceny własnych kompetencji. Po drugie, oceny własnej moralności mają charakter awersyjny. Przyjmując własną moralność za pewnik, ludzie niechętnie angażują się w sądy i oceny mogące potwierdzić lub zaprzeczyć ich moralności. Dowodów na rzecz tego procesu dostarczają wyniki \textcite{baryla2005wplyw}. W tych pomysłowych badaniach pokazano, że myślenie o sobie w pozytywnych kategoriach moralnych prowadzi do paradoksalnego wzrostu ruminacji, np. wypominania sobie niewłaściwych zachowań z przeszłości. Z kolei myślenie o sobie w kategoriach sprawnościowych prowadziło do przeciwnego efektu, do spadku częstości ruminacji. Te dwa procesy mogą tłumaczyć, dlaczego pomimo tego, że ludzie wysoko wartościują moralny charakter \parencite{gausel2011concern, goodwin2014moral} i przypisują sobie zazwyczaj wysoki poziom moralności, sądy te mogą nie odgrywać znaczącej roli w regulacji zachowania, i paradoksalnie prowadzić mogą do większej transgresji moralnej \parencite{merritt2010moral, monin2001moral, iyer2012sugaring}.\\

Należy w tym miejscu zaznaczyć, że istnieje szereg badań wskazujących, że sądy dotyczące wspólnotowości mogą decydować o poziomie kolektywnej samooceny. \textcite{leach2007group} wykazali, że sądy dotyczące moralności grupy własnej wyjaśniają większą proporcję zróżnicowania ocen tej grupy niż sądy dotyczące ciepła (\epmh{sociability}) lub kompetencji. Co więcej, w ramach tej samej linii badań pokazali oni, że jedynie wzbudzenie przekonania o wysokiej moralności grupowej prowadziło do wzrostu poczucia dumy z przynależności do danej grupy. W przypadku wzbudzenia przekonania o wysokich kompetencjach lub cieple grupy nie zaobserwowano podobnego efektu. O szczególnej roli moralności dla kolektywnej samooceny mogą też świadczyć wyniki \textcite{ellemers2008better} pokazujące, że wzbudzenie przekonania o wysokiej moralności grupy własnej prowadzi do większej chęci angażowania się w działania na rzecz poprawy jej statusu \emph{swoich}. Podobny efekt uzyskali również \textcite{pagliaro2011sharing}. Wydaje się w związku z tym, że o ile wspólnotowość nie odgrywa znaczącej roli jako wyznacznik indywidualnej samooceny, szczególnie wymiar moralności może mieć istotne znaczenie dla funkcjonowania grupowego i oceny kolektywnych aspektów Ja \parencite[patrz również,][]{ellemers2012morality, leach2013groups}.

\newpage
\subsection{Model dwóch perspektyw a wyznaczniki samooceny indywidualnej i kolektywnej}

W przedstawionym w niniejszej pracy przeglądzie, dostarczono dowodów na to, że zarówno sprawczość jak i wspólnotowość są istotnymi wyznacznikami samooceny. Mimo to, w cytowanych badaniach jako istotny wskazywano zazwyczaj tylko jeden z tych wymiarów. Łatwo jednak zauważyć, że autorzy, którzy odmiennie wskazywali istotne wyznaczniki samooceny, stosowali nieco inne metodologie i pytali o nieco inne rzeczy. Przykładowo, prace które wskazywały na istotność sprawczości jako wyznacznika samooceny dotyczyły zazwyczaj indywidualnych aspektów Ja \parencite[np., ][]{wojciszke2011self}. Natomiast autorzy, którzy wskazywali na istotność wspólnotowości jako wyznacznika samooceny pytali zazwyczaj o kolektywne aspekty Ja \parencite[np., ][]{leach2007group, ellemers2008better}. \\

Co więcej, gdy przyjrzeć się teoriom wskazującym na rozmaite funkcje samooceny, również zauważy się, że dotyczą one różnych poziomów Ja. O ile teorie opisujące samoocenę w sytuacjach zadaniowych \parencite[np., ][]{bandura1994self} dotyczą Ja indywidualnego, tak teoria socjometru \parencite{leary1995self} lub teoria opanowania trwogi \parencite{pyszczynski2004people} dotyczą Ja relacyjnego lub Ja kolektywnego. Przykładowo, jeżeli tak jak sugeruje teoria socjometru, samoocena ma informować o byciu akceptowanym lub odrzucanym przez najbliższe osoby, to oceniane treści będą elementami Ja relacyjnego, a niekoniecznie Ja indywidualnego. Z kolei jeżeli samoocena ma informować o postępach w realizacji indywidualnych celów, to treści inne niż Ja indywidualne będą miały niewielkie znaczenie.\\

Na różne podstawy samooceny -- uzależnione od skupienia na Ja niezależnym lub Ja współzależnym -- wskazywali już \textcite{markus1991culture}. W przypadku Ja niezależnego podstawę samooceny stanowi możliwość wyrażenia siebie i upewnianie się co do posiadania wewnętrznych atrybutów. Natomiast w przypadku Ja współzależnego podstawę samooceny stanowi umiejętność przystosowania się i narzucenia sobie ograniczeń w celu utrzymania harmonii z kontekstem społecznym. Innymi słowy w przypadku Ja niezależnego samoocenie podlegają przede wszystkim aspekty pomagające realizować założone cele, natomiast w przypadku Ja współzależnego samoocenie podlegają przede wszystkim aspekty pomagające utrzymywać dobre relacji ze znaczącymi Innymi.\\

Podobnie \textcite{gebauer2013agency} wskazali, że rola sprawczości i wspólnotowości jako wyznaczników samooceny zależy od tego, na ile centralny jest każdy z tych wymiarów wśród danej grupy osób. Przykładowo wykazali oni, że wymiar sprawczości jest bardziej centralny u młodszych, niereligijnych mężczyzn (stanowiących proxy dla kultury indywidualistycznej i Ja niezależnego), natomiast wymiar wspólnotowości jest bardziej centralny u starszych, religijnych kobiet (stanowiących proxy dla kultury kolektywistycznej i Ja współzależnego).\\

\subsubsection{Model sprawcy i biorcy \parencite{wojciszke2006perspektywa}}

Jedną z teoretycznych ram w jakie można ująć różnice pomiędzy wymiarami sprawczości i wspólnotowości jest zaproponowany przez \textcite{wojciszke2006perspektywa} model perspektywy sprawcy i biorcy \parencite[w ostatnich opracowaniach określany również jako model perspektywy aktora i obserwatora,][]{abele2014communal}. Zgodnie z modelem, dokonując sądów społecznych osoba może przyjmować jedną z dwóch perspektyw. To którą perspektywę przyjmie jest uzależnione w od wykonywanej w danym momencie czynności. Jeżeli osoba skupiona jest na implementacji działania i dąży do osiągnięcia założonego celu, wzbudzona zostanie u niej perspektywa sprawcy. Z kolei, jeżeli osoba obserwuje mające dla niej konsekwencje działanie sprawcy, wzbudzona zostanie u niej perspektywa biorcy. Zgodnie z opisywanym modelem, obie perspektywy są komplementarne, tzn. osoba może przyjąć jedną z nich, ale nie obie naraz.\\

Przyjęcie jednej z tych dwóch perspektyw niesie za sobą szereg konsekwencji o charakterze motywacyjnym i poznawczym \parencite[patrz,][]{wojciszke2010sprawczosc}. Po pierwsze, prowadzi do aktywizacji odmiennych treści: W przypadku przyjęcia perspektywy sprawcy aktywizowane są treści o charakterze sprawczym (związane z implementacją działania), natomiast w przypadku przyjęcia perspektywy biorcy aktywizowane są treści o charakterze wspólnotowym (związane z oceną konsekwencji działań sprawcy dla biorcy). Jak wskazuje \textcite{wojciszke2010sprawczosc}, o ile w przypadku perspektywy sprawcy cel jest pryzmatem percepcji, w przypadku perspektywy biorcy cel jest obiektem percepcji. Po drugie, przyjęcie jednej z perspektyw prowadzi do wzbudzenia odmienny motywacji: W przypadku perspektywy sprawcy są to takie motywacje jak osiągnięcia, potrzeba kontroli, i potrzena dominacji, natomiast w przypadku perspektywy biorcy są to motywacje takie jak potrzeba przynależności, akceptacji przez innych i bliskości. Po trzecie, w zależności od przyjętej perspektywy zmieniają się kryteria oceny zachowania: sprawca ocenia przede wszystkim efektywność swojego działania, natomiast biorca ocenia na ile dane zachowanie było przejawem ciepła interpersonalnego lub moralności.\\

Dowodów empirycznych wskazujących na słuszność przedstawionego modelu dostarczyły m. in. badania \textcite{wojciszke1994multiple}, pokazujące, że identyczne zachowanie może być oceniane na zupełnie różne sposoby, w zależności od tego kto dokonuje oceny. Kryterium oceny własnego działanie (w momencie jego implementacji) stanowią cechy związane z wymiarem kompetencji, natomiast kryterium oceny działania innej osoby (o ile działanie jest skierowane w stronę oceniającego) stanowią cechy związane z wymiarem moralności. W innych badaniach \parencite{wojciszke1997parallels} wykazano, że aktywizacja wartości indywidualistycznych (związanych z Ja i perspektywą sprawcy) prowadzi do wzrostu znaczenia wymiaru kompetencji w formowaniu wrażenia o innych, natomiast aktywizacja wartości kolektywistycznych (związanych z innymi i perspektywą biorcy) prowadzi do wzrostu znaczenia wymiaru moralności w formowaniu wrażenie o innych.\\

\subsubsection{Model własny}
Model własny zaproponowany w niniejszej rozprawie jest, opartą na koncepcji sprawcy i biorcy, próbą określenia podstawowych wyznaczników indywidualnej i kolektywnej samooceny. Postuluje on że, choć samoocena indywidualna i samoocena kolektywna są bliskimi sobie konceptami, pełnią one dla Ja odmienne funkcje i są związane z zaspokajaniem odmiennych potrzeb. Ze względu na funkcjonalne różnice samooceny indywidualnej i kolektywnej trudno oczekiwać, aby aspekty wyznaczające wysoki poziom obu były identyczne. Podstawowa hipoteza (H1) wyprowadzona z modelu własnego, głosi że w zależności od poziomu tożsamości (indywidualnej vs. kolektywnej) zmieniały się będą źródła pozytywnej samooceny.\\

Proponowany model głosi, że dokonując oceny siebie, tzn. indywidualnych aspektów Ja, ludzie przyjmują domyślnie perspektywę sprawcy. W związku z tym, (H2) o samoocenie na poziomie indywidualnym będą decydowały aspekty Ja związane z postrzeganą sprawczością. Co więcej, proponowany model postuluje, że dokonując oceny kolektywnych aspektów Ja (siebie w relacji do grupy społecznej) ludzie przyjmują domyślnie perspektywę biorcy. Związana z tym postulatem hipoteza głosi, że (H3) o samoocenie na poziomie kolektywnym będą decydowały aspekty związane z postrzeganą wspólnotowością. Innymi słowy, samoocena indywidualna dotyczy tych aspektów Ja, które służą ekspresji siebie, implementacji działań, oraz realizacji działań. Zaś kolektywna samoocena dotyczy tych aspektów Ja, które służą wpasowaniu siebie do grupy społecznej, z oczekiwaniami co do zachowania się pozostałych przedstawicieli grupy, a w najbardziej abstrakcyjnym ujęciu z oczekiwaniami co do norm i wartości wyznawanych przez reprezentantów grupy własnej.\\

Choć postulowany model zakłada, że indywidualne i kolektywne poziomy Ja związane są z, odpowiednio, perspektywą sprawcy i biorcy, należy podkreślić, że związki te mają charakter domyślny. Oznacza to, że postulowany model uwzględnia rolę czynników sytuacyjnych, które mogą skłonić osobę do zmiany perspektywy. W szczególności model postuluje, że ocena kolektywnego Ja, domyślnie związana z przyjęciem perspektywy biorcy, w pewnych okolicznościach może być dokonywana z perspektywy (kolektywnego) sprawcy. Jeżeli tak jest, to (H4) zmianie z perspektywy biorcy na perspektywę (kolektywnego) sprawcy towarzyszyć będzie wzrost znaczenia sprawczości jako wyznacznika kolektywnej samooceny.\\

Na ile wiadomo autorowi rozprawy, postulowany model nie był nigdy bezpośrednio weryfikowany. Pewnych dowodów na jego poparcie mogą dostarczać wyniki Badania 1 przeprowadzonego przez \textcite{wojciszke1997parallels}. W ramach tego badania wykazano, że wartości indywidualistyczne odnoszą się w większym stopniu do wymiaru kompetencji niż moralności. Z kolei wartości kolektywistyczne odnoszą się w większym stopniu do wymiaru moralności niż do kompetencji. Możliwa jest więc analogiczna zależność, polegająca na związku wymiaru sprawczości z Ja indywidualnym i związku wymiaru wspólnotowości z Ja kolektywny.\\

Z kolei, pośrednich dowodów na poparcie postulatu głoszącego, że zmiana perspektywy może być związana ze zmianą kryteriów oceny grupy własnej dostarczają wyniki \textcite{wojciszke2008primacy}. W tych badaniach wykazano, że postrzeganie współzależności z innymi prowadzi do relatywnego wzrostu znaczenia wymiaru sprawczości w ich ocenianiu. Te badania mogą sugerować, że aktywizacja obrazu grupy własnej jako realizującej cele podmiotu i chroniącej przed zagrożeniem może prowadzić do wzrostu znaczenia wymiaru sprawczości jako wyznacznika kolektywnej samooceny.\\

Podsumowując, przedstawiony w niniejszej rozprawie model jest propozycją rozstrzygnięcia istotnej kontrowersji obecnej we współczesnej psychologii społecznej i psychologii osobowości. Postuluje on, że określenie głównych wyznaczników tożsamości wymaga wpierw wyznaczenia granicy pomiędzy indywidualnymi i kolektywnymi poziomami Ja. Badania przedstawione w kolejnej części stanowią pierwszą próbę weryfikacji przewidywań z niego płynących.\\

\newpage
\section{Część empiryczna}

\subsection{Strategia badawcza}

W niniejszej pracy zaprezentowano cztery badania (łącznie przeprowadzone na, N = 898 osobach), których celem było opisanie relacji pomiędzy wymiarami sprawczości i wspólnotowości a samooceną na poziomie indywidualnym i kolektywnym. Dwa z przedstawionych badań miały charakter korelacyjny, a w przypadku dwóch pozostałych zastosowano schemat eksperymentalny. W trzech pierwszych badaniach starano się sprawdzić, które wymiary są związane lub wpływają na poziom indywidualnej i kolektywnej samooceny, a w ostatnim sprawdzono czy poprzez wzbudzenie perspektywy kolektywnego sprawcy zmianie ulegnie kształt zależności pomiędzy wymiarami sprawczości i wspólnotowości, a kolektywną samooceną. \\

Aspekty związane ze sprawczością zoperacjonalizowano jako postrzegany poziom własnej kompetencji (np. \emph{skuteczności, inteligencji}). Aspekty związane ze wspólnotowością rozdzielono i zoperacjonalizowano jako postrzegany poziom moralności (np. \emph{uczciwości, wiarygodności}) oraz jako postrzegany poziom ciepła w relacjach interpersonalnych (np. postrzegania własnej \emph{towarzyskości}). Podobną operacjonalizację wymiarów można znaleźć np. w pracy \textcite{leach2007group}.\\

W ramach porównywania tożsamości indywidualnej i kolektywnej, poziomy Ja rozdzielono na te związane z obrazem siebie jako jednostki, siebie jako członka grupy narodowej lub grupy narodowej jako całości. Należy zaznaczyć, że ostatnie dwie kategorie stanowią tożsamość kolektywną, ale rozumianą na dwa różne, przeplatające się w literaturze sposoby. W niniejszej pracy skupiono się na kolektywnej tożsamości narodowej, a pominięto inne ważne rodzaje tożsamości zbiorowych.\\

Samoocenę w niniejszym projekcie zoperacjonalizowano jako poziom poziom ogólnych, jawnych przekonań o własnej wartościowości. W ramach niniejszej rozprawy nie poruszano wątków związanych z samooceną utajoną, ani np. z samooceną o charakterze narcystycznym.

\subsection{Strategia analityczna}
Analizy przedstawione w niniejszej rozprawie -- w znaczącej mierze bazujące na ogólnym modelu liniowym -- przeprowadzono w podejściu bayesowskim \parencite[zob. np.,][]{gelman2014bayesian,gill2014bayesian, kruschke2014doing}. Główną zaletą podejścia bayesowskiego jest możliwość uwzględnienia w analizach przeszłej wiedzy (tzw. prior) i przez to, analitycznego powiązania ze sobą wyników linii badań (w odróżnieniu od klasycznego podejścia, w którym analizy są prowadzone tak jak gdyby badania były prowadzone w próżni, w oderwaniu pozostałych). Dzięki uwzględnieniu przeszłej wiedzy, wzrasta pewność co do analizowanych efektów i maleje szansa uzyskania czysto przypadkowych zależności. Ponadto, modele w podejściu bayesowskim cechuje znaczna elastyczność (tj. możliwość kwantyfikowania niepewności pochodzącej z różnych źródeł), przy podobnych lub często mniejszych wymogach co do liczebności badanej próby. Choć metody bayesowskie są niezwykle powszechne w statystyce, w pracach psychologicznych, zdominowanych przez podejście frekwentystyczne, są nadal rzadkością. Z tego względu w tej sekcji przedstawiono krótkie wprowadzenie mające ułatwić zrozumienie terminologii zawartej w części analitycznej rozprawy.\\

W ramach podejścia bayesowskiego zarówno dane, jak i parametry traktowane są jako zmienne losowe i jako takie posiadają ustalone rozkłady\footnote{W przypadku podejścia frekwentystycznego losowe są jedynie dane, natomiast parametry są traktowane jako posiadające jedną stałą wartość.}. Podstawowe równanie określa zależność pomiędzy rozkładami danych i parametrów:

\begin{equation}\label{eq:bayes}
    P(\theta | \mathcal{D}) =
    \frac{P(\mathcal{D} | \theta)\times P(\theta)}
    {P(\mathcal{D})}
\end{equation}

gdzie $P(\theta | \mathcal{D})$ oznacza rozkład prawdopodobieństwa parametrów przy określonych wartościach danych (tzw. rozkład posterior -- po dokonaniu obserwacji), $P(\mathcal{D} | \theta)$ oznacza rozkład prawdopodobieństwa danych przy określonych wartościach parametrów (inaczej funkcja wiarygodności związana z obserwacjami), $P(\theta)$ oznacza bezwarunkowy rozkład parametrów (tzw. rozkład prior -- przed dokonaniem obserwacji), a $P(\mathcal{D})$ oznacza bezwarunkowy rozkład danych (traktowany jako stała i czasami pomijany przy zachowaniu proporcjonalności obu stron równania). Mniej technicznym językiem: przekonania na temat występujących w populacji zależności (posterior) można ująć jako iloczyn wcześniejszej wiedzy (prior) i dokonanych obserwacji (funkcja wiarygodności). Lub jeszcze inaczej: równanie określa w jaki sposób obserwacje zmieniają przekonania badacza dotyczące interesującego go
wycinka rzeczywistości. \\

Celem analizy w podejściu bayesowskim jest uzyskanie rozkładu posterior (oraz takich jego charakterystyk jak wartość oczekiwana, wariancja, itd.) interesujących badacza parametrów. Zgodnie z równaniem \ref{eq:bayes} konieczne są tego informacje na temat trzech elementów. Funkcja wiarygodności jest obliczana podobnie jak w podejściu frekwentystycznym (przykładowo przy estymowaniu parametrów w równaniu regresji), tj. poprzez dobór wartości parametrów maksymalizujących wartość funkcji wiarygodności przy określonych danych. Rozkład prior jest ustalany na podstawie substantywnej wiedzy (pochodzącej z wcześniejszych badań, metaanaliz, na podstawie wiedzy eksperckiej, itd.). Rozkład prior dla niektórych parametrów może być również ustalony na podstawie tego co powszechnie o nich wiadomo (np. wartość odchylenia standardowego nigdy nie jest mniejsza od 0, a wartość absolutna współczynnika korelacji nigdy nie jest większa od 1). Gdy ustalenie rozkładu prior jest trudne lub niemożliwe, powszechne jest stosowanie tzw. nieinformatywnych lub referencyjnych rozkładów, tzn. takich które nie faworyzują żadnej określonej wartości (takim rozkładem może być np. rozkład jednostajny, a w przypadku parametrów oznaczających dyspersję powszechne jest stosowanie rozkładu Gamma, uciętego rozkładu Normalnego, lub uciętego rozkładu Cauchy'ego).\\

Analityczne ustalenie ostatniego elementu równania -- rozkładu brzegowego danych -- najbardziej kłopotliwe. W przypadku rozkładów ciągłych (np. rozkładu Normalnego) taka operacja wymaga policzenia całki:
\begin{equation}
    \int P(\mathcal{D} | \theta)\times P(\theta)\ \mathrm{d}\theta
\end{equation}
co zazwyczaj jest operacją trudną nawet dla komputera, a dość często jest to niemożliwe. \\

Obecnie ten problem jest rozwiązywany przez zastosowanie symulacji Monte Carlo, a w przypadku modeli wielozmiennowych przez jej wariantu -- tzw. Markov chain Monte Carlo \parencite[MCMC,][]{hastings1970monte}. Ten wariant symulacji polega na uruchomieniu ciągu (\emph{chain}) losowych, w niewielkim stopniu skorelowanych wartości, który ``przeszukuje'' przestrzeń parametrów modelu. Po pewnym czasie ciąg osiąga stan tzw. konwergencji, a jego wartości można traktować jako wylosowane niezależnie z rozkładu posterior i po odrzuceniu wartości początkowych (przed konwergencją) posłużyć się pozostałymi wartościami do podsumowania rozkładu. Aby ustalić, czy ciąg osiągnął konwergencję, wykonuje się zazwyczaj jeden (lub kilka) z opracowanych testów diagnostycznych; w niniejszej rozprawie do tego celu wykorzystano m.in. wskaźnik potencjalnej redukcji skali \parencite[\emph{potential scale reduction factor},][]{gelman1992inference}. Wartości tego współczynnika poniżej 1,1 interpretowane są jako brak powodów do odrzucenia ciągu ze względu na brak konwergencji.\\

Rozkład posterior podsumowuje się zazwyczaj podając jego wartość centralną (np. średnią) i miarę dyspersji (np. odchylenie standardowe). Z perspektywy badacza najbardziej istotne jest poznanie przedziału wiarygodności (\emph{credible interval}). Przedział wiarygodności określa fragment przestrzeni parametrów, w którym skupiony jest określony procent rozkładu (np. 95\%); czyli innymi słowy gdzie określonym z prawdopodobieństwem można spodziewać się wartości parametrów\footnote{Choć ta definicja może nasuwać skojarzenia z przedziałem ufności, nie należy mylić ze sobą tych dwóch koncepcji -- przedział ufności ma zdecydowanie inna mniej intuicyjną interpretację \parencite[zob. np.,][]{gill2014bayesian}.}. Przy wykorzystaniu rozkładu posterior można również testować hipotezy i określić prawdopodobieństwo z jakim parametr (np. współczynnik regresji) lub różnica parametrów (np. różnica średnich) jest większa lub mniejsza od określonej wartości (np. 0). Należy podkreślić, że takie prawdopodobieństwo należy traktować jako prawdopodobieństwo warunkowe (tj. prawdopodobieństwo przyjęcia przez parametry określonych wartości przy uwzględnieniu prawdziwości modelu i obserwacji badacza). Ponadto, pomimo, że w niniejszej rozprawie prawdopodobieństwo posterior jest raportowane w podobnej formie jak wartość $p$, należy zaznaczyć, że są to zupełnie różne koncepcje i nie należy mylić ich ze sobą.

\newpage
\subsection{Badanie 1}

Badanie 1 miało charakter eksploracyjny. Jego celem było sprawdzenie w jakim stopniu poziom uogólnionej (indywidualnej i kolektywnej) samooceny jest wyznaczany przez przypisywane sobie cechy związane z wymiarem sprawczości (tj. kompetencji) i wspólnotowości (tj. moralności i ciepła).\\

Dotychczasowe prace postulowały jedną z dwóch możliwości: Zgodnie z pierwszą, wariancja samooceny indywidualnej i kolektywnej będzie w znaczącej większości wyjaśniana przez cechy sprawcze \parencite{wojciszke2011self}, natomiast zgodnie z drugą wariancja samooceny indywidualnej i kolektywnej będzie w znaczącej większości tłumaczona przez cechy wspólnotowe \parencite[przede wszystkim moralność, np.][]{leach2007group}. Trzecia - postulowana przez model własny (H1) - z możliwości głosi, że układ predyktorów będzie różny dla samooceny indywidualnej i kolektywnej.\\

Dodatkowym celem było sprawdzenie, czy w ramach każdego z warunków układ predyktorów samooceny będzie analogiczny dla skal złożonych z cech pozytywnych jak i negatywnych, tj. czy predykcyjna rola niekompetencji, niemoralności oraz zimna będzie symetryczna do roli kompetencji, moralności i ciepła.

\subsubsection{Metoda}

\paragraph{Uczestnicy i schemat badania}
W badaniu wzięło udział $N=98$ studentów (74 kobiety, 21 mężczyzn, 3 osoby nie zadeklarowały swojej płci) w wieku od 19 do 27 lat ($M = $ 21,26; $SD =$ 1,81). Uczestnicy wypełniali jedną z trzech losowo przydzielonych wersji kwestionariusza: a) dotyczącą indywidualnych cech i indywidualnej samooceny ($n = 34$), b) dotyczącą cech i samooceny siebie jako Polaka ($n = 31$), lub c) dotyczącą kolektywnych cech i kolektywnej samooceny Polaków ($n = 33$). Procedurę losowania do warunków przeprowadzono z zachowaniem proporcji płci. Badanie przeprowadzano w salach wykładowych w grupach od 15 do 30 osób. Po zebraniu kwestionariuszy eksperymentator wyjaśniał grupie cel badania i odpowiadał na pytania.

\paragraph{Miary}
W każdym z trzech warunków uczestnicy wypełniali identyczne miary, a zmianom ulegała jedynie forma stwierdzeń. W pierwszym warunku, stwierdzenia rozpoczynały się od zaimka \emph{ja}, w warunku drugim stwierdzenia rozpoczynały się od frazy \emph{ja jako Polak}, natomiast w trzecim warunku od zaimka \emph{my} lub frazy \emph{my Polacy}. Podsumowanie statystyk dla każdej ze skali przedstawiono w Tabeli \ref{tab:0}.

\subparagraph{Skala kompetencji, moralności i ciepła}
W pierwszej kolejności uczestników poproszono o przypisanie sobie 18 cech -- pozytywnych i negatywnych -- związanych z wymiarami kompetencji (np. \emph{kompetentny, słaby}), moralności (np. \emph{moralny, dwulicowy}) i ciepła (np. \emph{ciepły, ponury}); tj. po 3 dla każdej z 6 subskal. Zadaniem każdej osoby była ocena na skali od 1 -- \emph{Zdecydowanie nie} do 7 -- \emph{Zdecydowanie tak} na ile dana cecha do pasuje pasuje do niej, do niej jako Polaka, lub do Polaków jako grupy. Miarę zaczerpnięto z poprzednich badań (Soral, Kofta, niepublikowana praca magisterska) -- jej trafność czynnikową i rzetelność zweryfikowano w dodatkowym badaniu pilotażowym. Mimo to w przypadku skali ciepła zaobserwowano zbyt nieskiej współczynniki rzetelności, $\alpha < $ 0,60  (patrz Tabela \ref{tab:0}). Aby poradzić sobie z tą słabością pomiaru, w części analitycznej postanowiono zastosować korektę o błędy pomiarowe dla każdej ze skal.

\subparagraph{Skala samooceny}
W dalszej części kwestionariusza każdy uczestnik wypełniał skalę samooceny opartą na mierze \textcite{rosenberg1965society}. Skala została zmodyfikowana, aby usunąć z niej stwierdzenia pytające wprost o komponenty związane z wymiarem sprawczości (czyli aby uniknąć redundancji) -- wyboru stwierdzeń dokonano na podstawie wcześniejszego badania pilotażowego. Dodatkowo, aby uzyskać lepsze właściwości psychometryczne, do skali dodano stwierdzenia ze Skali Lubienia Siebie \textcite{tafarodi2001two}. W sumie na skalę składało się 13 stwierdzeń (w tym 6 odwrotnie kodowanych), np. \emph{Uważam, że jestem osobą wartościową przynajmniej w takim samym stopniu co inni, Uważam, że jako Polak posiadam wiele pozytywnych cech, Czasami czuję, że my Polacy jesteśmy bezwartościowi}.\\

Uczestników poproszono aby ocenili na ile zgadzają się z każdym stwierdzeniem przy użyciu 4-stopniowej skali od 1 \emph{Zdecydowanie nie zgadzam się} do 4 \emph{Zdecydowanie się zgadzam}. Rzetelność skali okazała się satysfakcjonująca. \\

\begin{table*}[htbp]
\vspace*{2em}
\centering
\begin{threeparttable}
\caption{Podstawowe statystyki opisowe dla skal użytych w Badaniu 1}
\label{tab:0}
\begin{tabular}{lrrrrrrr}
\midrule
\multicolumn{2}{c}{ } & \multicolumn{2}{c}{Ja} & \multicolumn{2}{c}{Ja Polak} & \multicolumn{2}{c}{My Polacy} \\
\cline{3-4}\cline{5-8}
            & $\alpha$ & $M$ & $SD$ & $M$ & $SD$ & $M$ & $SD$ \\
\midrule
\emph{Pozytywne} \\
Kompetencje & 0,67 & 4,89 & 1,03 & 4,89 & 0,89 & 4,98 & 0,93 \\
Moralność   & 0,77 & 5,31 & 0,92 & 4,88 & 1,27 & 3,84 & 0,89 \\
Ciepło      & 0,54 & 4,96 & 1,28 & 4,69 & 1,15 & 5,02 & 1,11 \\
\emph{Negatywne} \\
Kompetencje & 0,60 & 2,53 & 1,24 & 2,29 & 0,92 & 2,70 & 0,93 \\
Moralność   & 0,81 & 2,79 & 1,12 & 2,78 & 1,43 & 4,51 & 1,35 \\
Ciepło      & 0,57 & 2,90 & 1,43 & 3,02 & 1,20 & 3,03 & 1,29 \\
\\
Samoocena & 0,86 & 2,95 & 0,50 & 3,21 & 0,44 & 2,97 & 0,45 \\
\midrule
\end{tabular}

\begin{tablenotes}[para,flushleft]
{\small
\textit{Nota.} $N$ = 98
}
\end{tablenotes}
\end{threeparttable}

\end{table*}


\subsubsection{Wyniki}
\paragraph{Autoaskrypcja cech pozytywnych a poziom samooceny}
W pierwszej kolejności sprawdzono w jakim stopniu, w każdym z warunków eksperymentalnym, przypisywanie sobie cech pozytywnych jest związane z poziomem samooceny. W tym celu skonstruowano model regresji miary samooceny (S) na miary kompetencji (K), moralności (M) i ciepła (C):

\begin{equation}\label{eq:1}
\begin{split}
S_{i} & = \beta_{0j} + \beta_{1j}K_{ji} + \beta_{2j}M_{ji} + \beta_{3j}C_{ji} + \epsilon_{i} \\
K & \sim N(kompetencje, \sigma_{Kj}) \\
M & \sim N(moralnosc, \sigma_{Mj}) \\
C & \sim N(cieplo, \sigma_{Cj}) \\
S & \sim N(samoocena, \sigma_{Sj})
\end{split}
\end{equation}

gdzie indeks $j$ oznacza jeden z trzech warunków badania (ja vs. ja Polak vs. my Polacy), a indeks $i$ osobę. Aby uwzględnić zróżnicowanie poziomów rzetelności każdej z miar, w każdym z warunków, model regresji oparto na ich wskaźnikach latentnych, o rozkładach normalnych ze średnią równą wartości obserwowanej miary i odchyleniu standardowym równym jej błędowi pomiarowemu wyliczonemu ze wzoru:

\begin{equation}\label{eq:err}
\sigma = SD*\sqrt{1 - r}
\end{equation}

gdzie $SD$ oznacza odchylenie standardowe miary, a $r$ rzetelność skali. W modelu zastosowano nieinformatywne rozkłady prior dla wszystkich parametrów. Wyniki oszacowania modelu zaprezentowano w górnej części Tabeli \ref{tab:1}\footnote{Aby uzyskać rozkłady posterior parametrów modelu wykorzystano algorytm MCMC (Markov Chain Monte Carlo, konkretnie \emph{Hamiltonian Monte Carlo}) zaimplementowany w programie Stan \parencite{carpenter2016} z 4000 iteracji. Ocena wskaźnika PSRF nie dostarczyła powodów do odrzucenia symulacji ze względu na brak konwergencji, $Rs < 1,01$.}. \\

Wśród uczestników poproszonych o myślenie o sobie w kategoriach `ja' samoocena była wyznaczana przez poziom przypisywanych sobie kompetencji (parametr dla skali kompetencji był pozytywny z prawdopodobieństwem, $p = 0,99$) i nie pozostawała w związku z przypisywaniem sobie cech ze skal moralności i ciepła (prawdopodobieństwo, że którykolwiek z parametrów dla tych skal był pozytywny wynosiło odpowiednio $p = 0,40$ i $p = 0,46$). \\

U uczestników poproszonych o myślenie o sobie w kategoriach `ja jako Polaka' samoocena była wyznaczana jedynie przez przypisywane sobie cechy ze skali ciepła (parametr był pozytywny z prawdopodobieństwem, $p = 0,99$), natomiast przypisywanie sobie cech ze skal kompetencji lub moralności nie miało związku z poziomem samooceny u tych uczestników (wartości parametrów były pozytywne z prawdopodobieństwem odpowiednio, $p = 0,21$ i $p = 0,63$). \\

W końcu, w przypadku uczestników których poproszono o myślenie o sobie w kategoriach `my Polacy' zbiorowa samoocena pozostawała w pozytywnym związku z przypisywaniem sobie cech ze skali moralności (z prawdopodobieństwem, $p = 0,98$). Przypisywanie sobie cech ze skali kompetencji i ciepła nie miało u tych uczestników związku ze zbiorową samooceną (wartości parametrów dla tych skal byłe pozytywne odpowiednio z, $p = 0,30$ i $p = 0,83$).\\

W całej badanej próbie zaprezentowany model wyjaśniał -- z prawdopodobieństwem, $p = 0,95$ -- od 2 do 30\% wariancji ogólnej samooceny. Wykresy rozrzutu wraz z liniami przedstawiającymi dopasowanie przedstawiono na Rycinie \ref{fig:1}.\\

\begin{table*}[htbp]
\vspace*{2em}
\centering
\begin{threeparttable}
\caption{Pozytywne i negatywne aspekty Ja: kompetencje, moralność i ciepło jako predyktory samooceny -- podsumowanie rozkładów parametrów modeli regresyjnych.}
\label{tab:1}
\bgroup
\def\tabcolsep{4pt}
\begin{tabular}{lrrrrrrrrr}
\midrule
 &
\multicolumn{3}{c}{Ja} &
\multicolumn{3}{c}{Ja Polak} &
\multicolumn{3}{c}{My Polacy} \\
\cline{2-10}
 & $M_{post.}$    & $SD$   & $95\%\ CI$   & $M_{post.}$    & $SD$   & $95\%\ CI$   & $M_{post.}$    & $SD$   & $95\%\ CI$   \\
\midrule
 \multicolumn{10}{c}{\emph{Pozytywne}}  \\
 Stała       &  2,16 & 0,57 &  1,03;3,28 &  2,62 & 0,48 &  1,66;3,59 &  2,10 & 0,55 &  1,00;3,15 \\
 Kompetencje &  0,19 & 0,09 &  0,02;0,36 & -0,09 & 0,12 & -0,33;0,16 & -0,05 & 0,09 & -0,24;0,13 \\
 Moralność   & -0,02 & 0,09 & -0,21;0,17 &  0,02 & 0,09 & -0,16;0,20 &  0,20 & 0,09 &  0,03;0,39 \\
 Ciepło      & -0,00 & 0,07 & -0,14;0,12 &  0,19 & 0,08 &  0,05;0,34 &  0,07 & 0,07 & -0,07;0,21 \\
 \multicolumn{10}{c}{\emph{Negatywne}}  \\
 Stała       &  3,25 & 0,26 &  2,75;3,75 &  3,57 & 0,23 &  3,13;4,02 &  4,20  & 0,29 & 3,63;4,80 \\
 Kompetencje & -0,21 & 0,06 & -0,33;-0,10 &  0,07 & 0,11 & -0,15;0,29 & -0,09 & 0,07 & -0,24;0,05 \\
 Moralność   & -0,04 & 0,07 & -0,18;0,09 & -0,04 & 0,07 & -0,17;0,10 & -0,14 & 0,05 & -0,25;-0,04 \\
 Ciepło      &  0,12 & 0,05 &  0,01;0,22 & -0,14 & 0,07 & -0,28;-0,00& -0,11 & 0,05 & -0,22;-0,00 \\
\bottomrule
\end{tabular}
\egroup
\begin{tablenotes}[para,flushleft]
{\small
\textit{Nota.} $N = 98$. Dobroć dopasowania modelu dla cech pozytywnych: $R^2$ = 0,13, 95\% CI = [0,02;0,30]. Dobroć dopasowania modelu dla cech negatywnych: $R^2$ = 0,25, 95\% CI = [0,09;0,45]
}
\end{tablenotes}
\end{threeparttable}
\end{table*}


\begin{figure*}[htbp]
   \centering
   \fitfigure{study1.pdf}
   \caption{Przypisywanie sobie (ja vs. ja jako Polak vs. my Polacy) pozytywnych cech związanych w wymiarami kompetencji, moralności i ciepła, a poziom samooceny indywidualnej oraz kolektywnej. Punkty oznaczają latentne wyniki dla każdej osoby, natomiast pionowe i poziome kreski oznaczają błędy pomiarowe skal. Grubą linią przerywaną oznaczono najlepsze dopasowanie uzyskane w modelu regresyjnym, z cieńszymi liniami oznaczającymi błąd oszacowania.}
   \label{fig:1}
\end{figure*}

\paragraph{Autoaskrypcja cech negatywnych a poziom samooceny}
W drugim kroku analogiczne analizy powtórzono dla cech negatywnych tworzących skale kompetencji, moralności i ciepła. W tym przypadku celem było sprawdzenie w jakim stopniu zaprzeczanie posiadania określonych negatywnych cech jest związane z pozytywną samooceną. Wyniki zaprezentowano w dolnej części Tabeli \ref{tab:1}\footnote{Ponownie model oparto na wskaźnikach latentnych konstruktów oraz zastosowano nieinformatywne rozkłady prior dla wszystkich konstruktów. Ocena wskaźnika PSRF dla 4000 iteracji MCMC nie dostarczyła powodów do odrzucenia symulacji ze względu na brak konwergencji, $Rs < 1,01$.}. \\

Wśród uczestników poproszonych o myślenie o sobie w kategoriach `ja' jedynie zaprzeczanie własnej niekompetencji wyznaczało pozytywną samoocenę (parametr był negatywny z prawdopodobieństwem, $p > 0,999$). Zaprzeczanie własnej niemoralności nie pozostawało w związku z samooceną (parametr był negatywny z prawdopodobieństwem, $p = 0,72$), natomiast zaprzeczanie własnemu zimnu pozostawało z samooceną w związku odwrotnym do oczekiwań (parametr był pozytywny z prawdopodobieństwem, $p = 0,98$), co oznacza, że w badanej próbie -- przy stałym poziomie pozostałych predyktorów -- osoby przypisujące sobie cechy związane z wymiarem zimna cechowały się wyższą indywidualną samooceną.\\

Wśród uczestników poproszonych o myślenie o sobie w kategoriach `ja jako Polaka', ci zaprzeczający zimnu charakteryzowali się wyższym poziomem samooceny (parametr był negatywny z prawdopodobieństwem, $p = 0,98$), natomiast zaprzeczanie własnej niekompetencji lub niemoralności nie było wśród tych uczestników związane z poziomem samooceny  (parametry dla tych skali były negatywne z prawdopodobieństwem, odpowiednio $p = 0,26$ i $p = 0,69$).\\

W warunku `my' im bardziej uczestnicy zaprzeczali niemoralności i zimnu Polaków tym ich zbiorowa samoocena była wyższa (parametry dla tych skal były negatywne z prawdopodobieństwem, odpowiednio $p = 0,99$ i $p = 0,98$). Zaprzeczanie niekompetencji Polaków nie pozostawało w związku z samooceną (parametr dla tej skali był negatywny z prawdopodobieństwem, $p = 0,89$).\\

Model dla cech negatywnych wyjaśniał -- z prawdopodobieństwem $p = 0,95$ -- od 9 do 45\% wariancji samooceny. Wykres rozrzutu wraz z liniami dopasowania przedstawiono na Rycinie \ref{fig:study1b}.


\begin{figure*}[htbp]
   \centering
   \fitfigure{study1b.pdf}
   \caption{Przypisywanie sobie (ja vs. ja jako Polak vs. my Polacy) negatywnych cech związanych w wymiarami kompetencji, moralności i ciepła, a poziom samooceny indywidualnej oraz kolektywnej. Punkty oznaczają latentne wyniki dla każdej osoby, natomiast pionowe i poziome kreski oznaczają błędy pomiarowe skal. Grubą linią przerywaną oznaczono najlepsze dopasowanie uzyskane w modelu regresyjnym, z cieńszymi liniami oznaczającymi błąd oszacowania.}
   \label{fig:study1b}
\end{figure*}

\subsubsection{Dyskusja}

Badanie 1 przeprowadzono, aby sprawdzić które aspekty Ja -- związane z wymiarem sprawczości czy wspólnotowości -- są wyznacznikami ogólnej (indywidualnej i kolektywnej) samooceny. Uzyskane wyniki nie wykazały szczególnego znaczenia wymiaru sprawczości dla ogólnej samooceny. W badaniu wykazano wprawdzie, że wymiar kompetencji ma szczególne znaczenie dla indywidualnej samooceny: Osoby które w większym stopniu przypisywały sobie pozytywne cechy sprawcze cechowały się wyższym poziomem indywidualnej samooceny. Nie wykazano jednak, aby wymiar kompetencji miał szczególne znaczenie dla samooceny kolektywnej: przy kontroli pozostałych predyktorów cechy sprawcze nie pozostawały w związku z samooceną siebie jako Polaka lub grupy jako ogółu. \\

Analogicznie, uzyskane wyniki nie dostarczyły poparcia dla tezy o szczególnym znaczeniu wymiaru wspólnotowości dla ogólnej samooceny. W zebranej próbie podwymiary wspólnotowości (moralność i ciepło) miały znaczenie dla samooceny kolektywnej, tj. osoby oceniające siebie jako członka grupy lub grupę jako ogół jako moralne i ciepłe cechowały się wyższym poziomem kolektywnej samooceny. Mimo to, ani moralność ani ciepło nie odgrywały roli jako wyznaczniki pozytywnej indywidualnej samooceny.\\

Uzyskane wyniki okazały się być zgodne z hipotezą (H1) mówiącą, że wraz poziomami Ja (indywidualnym vs. kolektywnym) zmieniają się wyznaczniki samooceny. W przeprowadzonym badaniu samoocena indywidualna była wyznaczana przez aspekty związane z wymiarem kompetencji, samoocena siebie jako członka grupy przez cechy związane z wymiarem ciepła, natomiast samoocena grupy jako ogółu przez cechy moralne. \\

Warto zaznaczyć, że układ predyktorów w ramach pozytywnych i negatywnych aspektów Ja okazał się w przybliżeniu symetryczny. Tak jak aspekty świadczące o kompetencjach były głównym wyznacznikiem wysokiej samooceny, tak aspekty świadczące o braku kompetencji okazały się głównym wyznacznikiem niskiej samooceny. Analogicznie postrzegane ciepło wyznaczało wysoką samoocenę siebie jako Polaka, natomiast postrzegane zimno niską samoocenę siebie jako przedstawiciela narodu polskiego. W końcu postrzegana moralność wyznaczała wysoką samoocenę Polaków jako grupy, a brak moralności niską samoocenę grupową. \\

Podsumowując, uzyskane dane wskazały na konieczność dalszej eksploracji badanego zagadnienia i replikacji uzyskanego wzorca zależności na próbie niestudenckiej: być może szczególna rola wymiaru kompetencji dla indywidualnej samooceny wynikała ze specyfiki tej próby. Celem kolejnego badania było również sprawdzenie predyktorów kolektywnej samooceny wśród grup innych niż Polacy: możliwe, że szczególna rola ciepła i moralności wynikała z centralności tych wymiarów dla autostereotypu Polaków. \\

\newpage
\subsection{Badanie 2}
Celem Badania 2 była replikacja wyników uzyskanych w Badaniu 1. Aby w większym stopniu zgeneralizować zależności, niniejsze badanie przeprowadzono na niestudenckiej próbie nie-Polaków: tu Amerykanów. O ile w przypadku prowadzenia badań na polskiej próbie istniała możliwość, że uzyskiwane efekty wynikają z niskiego statusu Polski na scenie międzynarodowej i np. z efektów kompensacyjnych \parencite[zob. np.][]{judd2005fundamental, oldmeadow2010social}, tak w przypadku badań na próbie amerykańskiej -- o wysokim statusie -- taka możliwość wydała się mniej prawdopodobna.\\

W niniejszym badaniu postanowiono sprawdzić związek przypisywania sobie kompetencji, moralności i ciepła z indywidualnym i kolektywnym poziomem samooceny. W badaniu skupiono się na cechach pozytywnych, ponieważ wyniki poprzedniego badania wskazały na symetryczność relacji pomiędzy wymiarami pozytywnymi i negatywnymi a samooceną. Samoocena była tu rozumiana ponownie jako ogólne przekonanie o własnej wartościowości (lub wartościowości grupy własnej). Jednak dodatkowo postanowiono zweryfikować wyznaczniki samooceny rozumianej jako przekonania o własnej sprawczości i możliwości radzenia sobie w różnych sytuacjach. Choć ten drugi model cechował się redundancją (tzn. wyjaśnianie percepcji sprawczości przez aspekty związane z wymiarem kompetencji) pozwalał on na lepsze zrozumienie charakteru wyznaczników samooceny zwłaszcza w przypadku kolektywnego Ja.\\

\subsubsection{Metoda}

\paragraph{Uczestnicy i schemat badania}
Badanie przeprowadzono na $N=103$ użytkownikach amerykańskiego serwisu mTurk (45 kobiet, 58 mężczyzn). Wiek uczestników wahał się od 18 do 66 lat ($M=$ 31,55, $SD=$ 12,58). Podobnie jak w Badaniu 1 uczestnicy zostali losowo przydzieleni do jednego z trzech warunków: a) dotyczącego cech indywidualnych, b) dotyczącego cech siebie jako przedstawiciela grupy (Amerykanina), c) dotyczącego wspólnych cech grupy (Amerykanów). Procedurę losowania przeprowadzono z zachowaniem proporcji płci. Po wypełnieniu kwestionariusza każdy uczestnik przechodził na stronę, na której wyjaśniano mu cel badania.

\paragraph{Miary}
Tak jak w poprzednim badaniu, w każdym warunku uczestnicy wypełniali skale, w których stwierdzenia posiadały identyczne człony główne, a różniły się jedynie formą: \emph{ja, ja jako Amerykanin, my Amerykanie}. Podstawowe statystyki opisowe dla użytych skal przedstawiono w Tabeli \ref{tab:02}.

\subparagraph{Skala kompetencji, moralności i ciepła}
Po zapoznaniu się z instrukcjami wstępnymi uczestnicy byli proszeni o przypisanie sobie 9 pozytywnych cech związanych z wymiarami kompetencji (np. \emph{kompetentny -- competent}), moralności (np. \emph{uczciwy -- honest}) i ciepła (\emph{ciepły -- warm}). Były to stwierdzenia zaczerpnięte z badań \textcite{leach2007group}) i ich znaczenie było zbliżone do pozytywnych stwierdzeń zastosowanych w Badaniu 1. Zadaniem uczestnika była ocena na skali od 1 -- \emph{Zdecydowanie nie} do 7 -- \emph{Zdecydowanie tak}, na ile on, on jako Amerykanin lub Amerykanie jako grupa posiadają każdą z tych cech. Statystyki zgodności wewnętrznej dla wszystkich skal były satysfakcjonujące, $\alpha <$ 0,75, (zob. Tabela \ref{tab:02}). \\

\subparagraph{Skale lubienia siebie i sprawczości}
W następnej części uczestnicy wypełniali skalę samooceny składającą się z dwóch części: 10 pytań dotyczących lubienia siebie/nas (np. \emph{Czuję się komfortowo gdy myślę o sobie -- I am very comfortable with myself}) i 10 pytań dotyczących przekonań o własnej -- lub kolektywnej -- sprawczości (np. \emph{Jestem niezwykle efektywny w wykonywaniu różnych czynności -- I am highly effective at the things I do}). W ramach każdej części połowa stwierdzeń była kodowana odwrotnie. Zadaniem uczestnika była ocena na ile zgadza się z każdym ze stwierdzeń przy użyciu 4-stopniowej skali odpowiedzi od 1 -- \emph{Zdecydowanie się nie zgadzam} do 4 -- \emph{Zdecydowanie się zgadzam}.\\

Wykorzystana skala została opracowana przez \textcite{tafarodi1995self, tafarodi2001two}. W kolejnych badaniach wykazano 2-czynnikową strukturę skali \parencite{tafarodi2002decomposing}. W niniejszym badaniu obie podskale uzyskały satysfakcjonujące współczynniki zgodności wewnętrznej (zob. Tabela \ref{tab:02}).

\begin{table*}[htbp]
\vspace*{2em}
\centering
\begin{threeparttable}
\caption{Podstawowe statystyki opisowe dla skal użytych w Badaniu 2}
\label{tab:02}

\begin{tabular}{lrrrrrrr}

\midrule
\multicolumn{2}{c}{ } & \multicolumn{2}{c}{Ja} & \multicolumn{2}{c}{Ja Amerykanin} & \multicolumn{2}{c}{My Amerykanie} \\
\cline{3-8}
& $\alpha$ & $M$ & $SD$ & $M$ & $SD$ & $M$ & $SD$ \\
\midrule
Kompetencje & 0,79 & 5,72 & 0,92 & 5,68 & 1,00 & 5,14 & 1,14 \\
Moralność   & 0,78 & 5,90 & 0,75 & 5,57 & 0,98 & 4,04 & 1,15 \\
Ciepło      & 0,75 & 4,97 & 1,24 & 5,21 & 1,28 & 5,28 & 0,74 \\
Lubienie siebie & 0,93 & 2,91 & 0,54 & 3,04 & 0,62 & 3,05 & 0,44 \\
Sprawczość Ja   & 0,88 & 3,08 & 0,48 & 3,21 & 0,45 & 3,09 & 0,48 \\
\bottomrule

\end{tabular}

\begin{tablenotes}[para,flushleft]
{\small
\textit{Nota.} $N$ = 103
}
\end{tablenotes}
\end{threeparttable}
\end{table*}

\subsubsection{Wyniki}

\paragraph{Przypisywanie kompetencji, moralności i ciepła, a poziom lubienia siebie}

W pierwszym kroku analiz sprawdzono w jakim stopniu przypisywanie sobie cech ze skal kompetencji, moralności i ciepła wyznacza poziom lubienia siebie jako jednostki vs. siebie jako członka grupy vs. grupy jako całości. Ponownie wykorzystano model uwzględniający niepewność pomiarową (por. Równanie \ref{eq:1}). Tym razem zastosowano informatywne rozkłady prior oparte na wynikach Badania 1 (por. górna część Tabeli \ref{tab:1})\footnote{Uwzględnienie tej dodatkowej informacji, pozwoliło uzyskać dokładniejsze oszacowania parametrów modelu \parencite[zob. np.][]{gill2014bayesian}. Rozkłady posterior parametrów modelu otrzymano na podstawie symulacji MCMC z 4000 iteracji. Wskaźniki potencjalnej redukcji skali (PSRF) nie dostarczyły powodów do odrzucenia symulacji ze względu na brak konwergencji, Rs < 1,01}. Podsumowanie wyników przedstawiono w górnej części Tabeli \ref{tab:2}. \\

Wśród uczestników w warunku `ja', poziom lubienia siebie był wyznaczany przez przypisywanie sobie cech ze skal kompetencji oraz ciepła (parametry dla tych skal były z pozytywne odpowiednio z, $p > $ 0,999 i $p = $ 0,99). Lubienie siebie nie było jednak związane z przypisywaniem sobie cech ze skali moralności (parametr był pozytywny z prawdopodobieństwem, $p = $ 0,63). \\

U uczestników zapytanych o postrzeganie siebie `jako Amerykanina', lubienie siebie było pozytywnie związane jedynie z przypisywaniem sobie cech ze skali ciepła (z prawdopodobieństwem, $p >$ 0,999). Nie zaobserwowano w tym warunku związku lubienia siebie z przypisywaniem sobie cech ze skali kompetencji i moralności (parametry były pozytywne odpowiednio z $p =$  0,73 i $p =$ 0,72).\\

W trzecim warunku, lubienie `nas Amerykanów' było uzależnione od przypisywania Amerykanom cech ze skali moralności (parametr był pozytywny z prawdopodobieństwem, $p >$ 0,999), ale nie od przypisywania cech ze skal kompetencji i ciepła (parametry były pozytywne z prawdopodobieństwem odpowiednio, $p =$ 0,11 i $p =$ 0,87)

Ogólnie model wyjaśniał od 4 do 30\% procent wariancji lubienia siebie (z prawdopodobieństwem $p =$ 0,95). Wykresy rozrzutu wraz z liniami dopasowania przedstawiono na Rycinie \ref{fig:study2a}.\\

\begin{table*}[htbp]
\vspace*{2em}
\centering
\begin{threeparttable}
\caption{Kompetencje, moralność i ciepło jako predyktory lubienia siebie i przekonań o sprawczości Ja -- podsumowanie rozkładów parametrów modeli regresji.}
\label{tab:2}
\bgroup
\def\tabcolsep{4pt}
\begin{tabular}{lrrrrrrrrr}
\midrule
 &
\multicolumn{3}{c}{Ja} &
\multicolumn{3}{c}{Ja jako Amerykanin} &
\multicolumn{3}{c}{My Amerykanie} \\
\cline{2-10}
 & $M_{post.}$    & $SD$   & $95\%\ CI$   & $M_{post.}$    & $SD$   & $95\%\ CI$   & $M_{post.}$    & $SD$   & $95\%\ CI$   \\
\midrule
 \multicolumn{10}{c}{\emph{Lubienie siebie}}  \\
 Stała       &  1,05 & 0,54 & -0,01;2,12 &  1,42 & 0,46 &  0,52;2,31 &  2,33 & 0,40 &  1,56;3,12 \\
 Kompetencje &  0,18 & 0,06 &  0,06;0,30 &  0,05 & 0,08 & -0,10;0,20 & -0,07 & 0,06 & -0,19;0,04 \\
 Moralność   &  0,02 & 0,07 & -0,11;0,15 &  0,04 & 0,07 & -0,10;0,19 &  0,18 & 0,06 &  0,07;0,30 \\
 Ciepło      &  0,14 & 0,05 &  0,05;0,23 &  0,22 & 0,06 &  0,11;0,33 &  0,07 & 0,06 & -0,05;0,19 \\
 \multicolumn{10}{c}{\emph{Sprawczość Ja}}  \\
 Stała       &  0,80 & 0,45 & -0,09;1,71 &  1,90 & 0,36 &  1,20;2,62 &  1,35 & 0,33 &  0,70;1,99 \\
 Kompetencje &  0,28 & 0,05 &  0,18;0,38 &  0,10 & 0,06 & -0,02;0,22 &  0,13 & 0,05 &  0,03;0,23 \\
 Moralność   &  0,02 & 0,06 & -0,10;0,13 & -0,02 & 0,06 & -0,14;0,11 &  0,18 & 0,05 &  0,08;0,28 \\
 Ciepło      &  0,12 & 0,04 &  0,04;0,19 &  0,16 & 0,05 &  0,07;0,26 &  0,07 & 0,06 & -0,04;0,17 \\
\bottomrule
\end{tabular}
\egroup
\begin{tablenotes}[para,flushleft]
{\small
\textit{Nota.} $N = 103$. Dobroć dopasowania modelu dla lubienia siebie: $R^2$ = 0,15, 95\% CI = [0,04;0,30]. Dobroć dopasowania dla sprawczości: $R^2$ = 0,43, 95\% CI = [0,23;0,64]
}
\end{tablenotes}
\end{threeparttable}
\end{table*}


\begin{figure*}[htbp]
   \centering
   \fitfigure{study2a.pdf}
   \caption{Przypisywanie sobie (ja vs. ja jako Amerykanin vs. my Amerykanie) cech związanych w wymiarami kompetencji, moralności i ciepła, a poziom lubienia siebie/nas. Punkty oznaczają latentne wyniki dla każdej osoby, natomiast pionowe i poziome kreski oznaczają błędy pomiarowe skal. Grubą linią przerywaną oznaczono najlepsze dopasowanie uzyskane w modelu regresyjnym, z cieńszymi liniami oznaczającymi błąd oszacowania.}
   \label{fig:study2a}
\end{figure*}


\paragraph{Przypisywanie kompetencji, moralności i ciepła, a poziom przekonań o własnej sprawczości}
W drugim kroku analiz sprawdzono w jakim stopniu wymiary kompetencji, moralności i ciepła wyznaczają pozytywną samoocenę rozumianą jako przekonanie o radzeniu sobie w różnych sytuacjach. W tym celu powtórzono analizy z wykorzystaniem modelu z poprzedniej sekcji, ale jako zmienną objaśnianą uwzględniono skalę sprawczości Ja\footnote{Parametry posterior uzyskano stosując algorytm MCMC z 4000 iteracji. Wartość wskaźnika PSRF nie przekraczała, R = 1,01, i nie wskazywała na brak konwergencji symulacji.}. \\

W warunku `ja' przekonania o sprawczości uczestników były związane z przypisywanymi sobie cechami ze skal kompetencji (parametr był pozytywne z $p >$ 0,999), ale również ciepła (parametr był pozytywny z $p =$ 0,99). Nie były one jednak związane z przypisywanymi sobie cechami ze skali moralności (parametr był pozytywny $p =$ 0,60). \\

W drugim warunku, przekonania o sprawczości siebie `jako Amerykanina' były związane jedynie z przypisywanymi sobie przez uczestników cechami ze skali ciepła (parametr był pozytywny z, $p =$ 0,99), nie zaobserwowano związku z przypisywaniem sobie cech ze skali moralności (parametr był pozytywny z $p =$ 0,40). Zaobserwowano słaby związek z przypisywaniem sobie cech ze skali kompetencji (parametry był pozytywny z, $p =$ 0,93).\\

Wśród uczestników pytanych o postrzeganie `nas, Amerykanów', przekonania o sprawczości były związane z przypisywaniem Amerykanom cech ze skal moralności i kompetencji (parametry dla tych skal były pozytywne z prawdopodobieństwem odpowiednio, $p >$ 0,999 i $p =$ 0,99). Nie zaobserwowano związku z przypisywaniem Amerykanom cech ze skali ciepła (parametr był pozytywny z $p =$ 0,89). \\

Wykresy rozrzutu wraz liniami dopasowania przedstawiono na Rycinie \ref{fig:study2b}. Cały model wyjaśniał -- z $p =$ 0,95 od 23 do 64\% wariancji przekonań o sprawczości Ja.\\

\begin{figure*}[htbp]
   \centering
   \fitfigure{study2b.pdf}
   \caption{Przypisywanie sobie (ja vs. ja jako Amerykanin vs. my Amerykanie) cech związanych w wymiarami kompetencji, moralności i ciepła, a poziom przekonań o własnej sprawczości. Punkty oznaczają latentne wyniki dla każdej osoby, z kreskami oznaczającymi błędy pomiarowe. Grubą linią przerywaną oznaczono najlepsze dopasowanie uzyskane w modelu regresyjnym, z cieńszymi liniami oznaczającymi błąd oszacowania.}

   \label{fig:study2b}
\end{figure*}


\subsubsection{Dyskusja}
Celem przedstawionego badania była replikacja wyników uzyskanych w Badaniu 1 na próbie niestudenckiej i poza polskim kontekstem. Badanie przeprowadzone na amerykańskiej próbie internetowej potwierdziło, że wyznaczniki samooceny zależą od skupienia na poziomie tożsamości: indywidualnej vs. kolektywnej.\\
Poziom samooceny rozumianej jako lubienie siebie był wyznaczany: w przypadku ja indywidualnego przez postrzeganie siebie na wymiarze kompetencji, w przypadku siebie jako członka grupy przez postrzeganie siebie na wymiarze ciepła, natomiast w przypadku grupy jako całości przez postrzeganie jej na wymiarze moralności. Niezgodny z oczekiwania okazał się być zaobserwowany -- choć relatywnie niewielki -- związek postrzeganego ciepła z indywidualną samooceną. \\

W niniejszym badaniu postanowiono również sprawdzić w jakim stopniu przypisywanie kompetencji, moralności i ciepła są związane z indywidualną i kolektywną samooceną rozumianą jako przekonania o sprawczości Ja. Pomimo redundancji tego modelu i obserwowanej istotnej roli wymiaru kompetencji w każdym z warunków, przekonania o indywidualnej sprawczości były wyznaczane przede wszystkim przez wymiar kompetencji, przekonania o sprawczości siebie jako członka grupy przede wszystkim przez wymiar ciepła, natomiast przekonania o sprawczości Polaków jako grupy przede wszystkim przez wymiar moralności.\\


\newpage
\subsection{Badanie 3}
Celem Badania 3 było sprawdzenie, czy aktywizacja pozytywnych aspektów indywidualnego vs. kolektywnego Ja związanych z wymiarami kompetencji, moralności i ciepła zwiększy poziom poziom samooceny. O ile w poprzednich dwóch badaniach sprawdzano jedynie zależności korelacyjne, tak w tym podjęto się próby ustalenia zależności przyczynowo-skutkowej. \\

Na podstawie poprzednich dwóch badań postawiono hipotezę mówiącą, że aktywizacja aspektów dotyczących wymiaru kompetencji wpłynie pozytywnie jedynie na poziom indywidualnej samooceny. Z kolei aktywizacja dotyczących wymiaru ciepła i moralności wpłynie pozytywnie na poziom samooceny odpowiednio siebie jako Polaka i Polaków jako grupy. \\

\subsubsection{Metoda}

\paragraph{Uczestnicy i schemat badania}
W eksperymencie wzięło udział $N = 184$ (90 kobiet i 94 mężczyzn) użytkowników internetowego panelu badawczego w wieku od 20 do 35 lat ($M =$ 26,63; $SD =$ 3,84). Pierwotnie zebrano próbę w wielkości $N = 315$. Niestety wśród tych osób znaczna część osób (131 osób) nie poradziła sobie z testem sprawdzającym uważność w czasie badania. Ponieważ interpretacja uzyskanych wyników była uzależniona od tego, czy uczestnicy zwrócą uwagę na instrukcje i będą się do nich stosowali, postanowiono wykluczyć osoby nieuważnie wypełniające kwestionariusz.\\

Po wyrażeniu zgody na udział w badaniu uczestnicy byli losowo przydzielani do jednego z 4 warunków: w trzech z grup u uczestników aktywizowano treści związane z wymiarami kompetencji lub moralności lub ciepła, natomiast czwarta grupa stanowiła warunek kontrolny. Następnie w ramach każdego z warunków uczestnicy byli losowo przydzielani do jednego z trzech wariantów: `ja jako jednostka' lub `ja jako Polak' lub `my Polacy'. Tak więc w sumie uczestnicy byli przydzieleni do jednego z 12 warunków (średnia liczba uczestników na warunek wynosiła, $n = 15$); losowanie przeprowadzono z zachowaniem proporcji płci. Po przejściu procedury manipulacji każdy uczestnik wypełniał miarę samooceny oraz miarę skuteczności manipulacji, a następnie był przekierowywany na stronę na której wyjaśniano mu cel badania oraz cel oddziaływania eksperymentalnego.

\paragraph{Procedura manipulacji}
Każdy uczestnik był proszony o zapoznanie się z listą 8 cech, a następnie o przypomnienie sobie sytuacji w której on (lub w zależności od wariantu: on jako Polak lub Polacy jako grupa) wykazał/wykazali się którąś z cech z listy. Uczestnicy byli proszeni o zastanowienie się przez chwilę nad zdarzeniem, a następnie zaznaczenie trzech cech, którymi wykazali się w przypominanej sytuacji. W zależności od warunku lista cech składała się tylko z cech dotyczących jednego z wymiarów: kompetencji (np. \emph{sprawny, inteligentny}) lub moralności (np. \emph{moralny, sprawiedliwy}) lub ciepła (np. \emph{ciepły, przyjazny}). W warunku kontrolnym listę cech zastąpiono 8 stwierdzeniami dotyczącymi kwestii społeczno-politycznych (np. \emph{Globalizacja niesie za sobą więcej szkody niż pożytku}): W tym przypadku uczestnik był proszony o przypisanie do trzech najbardziej odpowiadających mu stwierdzeń (lub takich które odpowiadają mu jako Polakowi lub Polakom jako grupie).

\paragraph{Miary}
Podobnie jak w poprzednich badaniach, stwierdzenia w zastosowanych miarach posiadały identyczne człony główne, a różniły się pomiędzy wariantami jedynie formą: dotyczyły `ja' jako jednostki, jako Polaka lub Polaków jako grupy.

\subparagraph{Skala samooceny}
W eksperymencie zastosowano identyczną skalę samooceny jak w Badaniu 1. Składała się ona z 13 stwierdzeń (w tym 6 kodowanych odwrotnie), np. \emph{Uważam, że jestem osobą wartościową przynajmniej w takim samym stopniu co inni}. Uczestnicy odpowiadali na ile zgadzają się z każdym ze stwierdzeniem na skali od 1 -- \emph{Zdecydowanie się nie zgadzam} do 4 -- \emph{Zdecydowanie się zgadzam}. Skala cechowała się zadowalającą zgodnością wewnętrzną w każdym z 3 wariantów: ja jako jednostka, $\alpha$ = 0,91; ja jako Polak, $\alpha$ = 0,92; my Polacy, $\alpha$ = 0,85.
\subparagraph{Pomiar skuteczności manipulacji}
Aby sprawdzić, czy zastosowana procedura manipulacji prowadziła do aktywizacji treści związanych z wymiarami kompetencji, moralności i ciepła, po wypełnieniu skali samooceny każdy uczestnik przechodził do części, w której na osobnych stronach prezentowano mu 8 cech: 2 dwie pierwsze o charakterze buforowym (\emph{realistyczny, poważny}) oraz po dwie cechy związane z wymiarami kompetencji (\emph{dokładny, bystry}), moralności (\emph{prawy, szczery}), i ciepła (\emph{miły, troskliwy}). Żadna z prezentowanych cech nie pojawiała się w procedurze manipulacji. \\

Uczestnicy mieli ocenić na ile każda z cech jest dla nich charakterystyczna (lub w zależności od wariantu dla nich jako Polaków lub dla Polaków jako grupy) i zaznaczyć odpowiedź jak najszybciej to możliwe na skali od 1 -- \emph{Zdecydowanie nie} do 7 -- \emph{Zdecydowanie tak}. Co istotne, oprócz odpowiedzi dla każdej cechy kodowano czas w którym uczestnicy dokonali pierwszego kliknięcia. Założono, że aktywizacja wymiaru kompetencji, moralności lub ciepła będzie prowadziła do szybszej reakcji na cechy związane z każdym z wymiarów.

\subsubsection{Wyniki}
\paragraph{Sprawdzenie skuteczności manipulacji}
Aby upewnić się co do skuteczności manipulacji porównano czasy reakcji na cechy związane z wymiarami kompetencji, moralności i ciepła w ramach każdego z czterech warunków eksperymentalnych. W tym celu przeprowadzono 3 analizy kontrastów w ramach hierarchicznego modelu liniowego \parencite[odpowiednika  klasycznej jednoczynnikowej analizy wariancji, za:,][]{kruschke2014doing} z odpornymi błędami rozproszonymi zgodnie z rozkładem t Studenta (z liczbą stopni swobody, $\nu = 7$).\\

W ramach pierwszej analizy porównano czasy odpowiedzi na pytania dotyczące wymiaru kompetencji. Osoby przypominające sobie wydarzenia dotyczące kompetencji odpowiadały na te pytania o 1,40 sekundy szybciej niż osoby przypominające sobie o zdarzeniach dotyczących moralności lub ciepła -- różnica była większa od 0 z $p =$ 0,99 -- i o 0,75 sekundy szybciej niż w warunku kontrolnym -- różnica większa od 0 z $p =$ 0,99.\\

Następnie porównano czasy odpowiedzi na pytania dotyczące wymiaru moralności. U osób przypominających sobie wydarzenia dotyczące moralności odpowiedzi na te pytania były szybsze o 0,61 sekundy niż u osób przypominających sobie wydarzenia dotyczące kompetencji lub ciepła -- różnica większa od 0 z $p =$ 0,87 -- i szybsze o 0,46 sekundy niż w warunku kontrolnym -- różnica większa od 0 z $p =$ 0,97. \\

W końcu porównano odpowiedzi na pytania dotyczące wymiaru ciepła. U osób przypominających sobie wydarzenia dotyczące ciepła czasy odpowiedzi na pytania dotyczące tego wymiaru były szybsze o 0,16 sekundy w warunku kontrolny, -- różnica była większa od 0 z $p =$ 0,80. Nie zanotowano różnicy pomiędzy warunkiem dotyczącym ciepła a średnią z warunków w których dokonywano aktywizacji kompetencji lub moralności-- różnica była większa od 0 z $p =$ 0,49.\\

Analogiczne analizy dla treści odpowiedzi - tzn. przypisywanych sobie kompetencji, moralności i ciepła nie wykazały różnic pomiędzy warunkami. Podsumowując, choć zaobserwowane różnice okazały się nieznaczne, były one jednak zgodne z przewidywaniami, i wskazały, że zastosowana manipulacja była skuteczna w przypadku wzbudzania treści dotyczących kompetencji i moralności. Manipulacja okazała się również skuteczna -- choć w nieco mniejszym stopniu -- w przypadku wzbudzania treści dotyczących ciepła.\\

\paragraph{Aktywizacja wymiarów komptetencji, moralności i ciepła a poziom samooceny}
Aby zweryfikować hipotezę mówiącą, że wpływ aktywizacji wymiarów kompetencji, moralności i ciepła na samoocenę będzie różny w zależności od poziomu Ja, oraz że aktywizacja wymiaru kompetencji wpłynie na poziom indywidualnej samooceny a aktywizacja wymiarów moralności i/lub ciepła na poziom kolektywnej samooceny Polaków, dane poddano analizie w ramach hierarchicznego modelu liniowego \parencite[jako odpowiednika klasycznej dwuczynnikowej analizy wariancji, za:,][]{kruschke2014doing}.
\begin{equation}\label{eq:4}
\begin{split}
y_i  \sim\ & N(\mu_i, \sigma_y) \\
\mu_i  = \beta_0 + \sum_j\beta_{1[j]}x_{1[j]}(i)\ +\ &\sum_k\beta_{2[k]}x_{2[k]}(i)+\sum_{j,k}\beta_{1\times2[j,k]}x_{1\times2[j,k]}(i) \\
\sigma_y \sim U(0.01,6.41)\ &\ \beta_0 \sim N(2.80, 3.21) \\
\beta_1 \sim N(0, \sigma_{\beta1})\ &\ \beta_2 \sim N(0, \sigma_{\beta2}) \\
\beta_{1\times2} \sim\ & N(0, \sigma_{\beta1\times2}) \\
\sigma_{\beta1} \sim IG(1.28,0.88)\ \ \sigma_{\beta2} \sim\ & IG(1.28,0.88)\ \ \sigma_{\beta1\times2} \sim IG(1.28,0.88)
\end{split}
\end{equation}
gdzie $y$ oznacza poziom samooceny, $x_1$ warunek eksperymentalny, a $x_2$ wariant eksperymentu (oznaczenia rozkładów: $N$ -- rozkład Normalny; $U$ -- rozkład prostokątny; $IG$ -- odwrócony rozkład Gamma). Oszacowania średnich w każdym z warunków przedstawiono w Tabeli \ref{tab:3}. Należy jednak zaznaczyć, że ze względu na niewielkie liczebności w ramach każdego z warunków, podstawą do wnioskowania w przypadku tych analiz były nie porównania między grupami, ale analiza kontrastów. \footnote{Aby uzyskać oszacowania parametrów rozkładu posterior wykorzystano algorytm MCMC zaimplementowany w programie Stan \parencite{carpenter2016} z 4000 iteracji. Analiza wskaźnika PSRF nie dostarczyła powodów dla odrzucenia symulacji ze względu na brak konwergencji, $Rs <$ 1,02.}.  \\

W pierwsze kolejności sprawdzono, czy występują różnice średnich samooceny pomiędzy wariantami eksperymentu (ja vs. ja jako Polak vs. my Polacy). Przeprowadzone analizy nie wykazały jednak różnic pomiędzy poziomem samooceny indywidualnej i kolektywnej (prawdopodobieństwo różnicy pomiędzy którymkolwiek z wariantów było mniejsze niż, $p =$ 0,77). \\

W następnym kroku, sprawdzono czy wydobywania z pamięci wydarzeń świadczących o kompetencjach, moralności i cieple prowadziło do wzrostu poziomu samooceny (indywidualnej i kolektywnej). W ramach pierwszego kontrastu porównano osoby przypominające sobie wydarzenia świadczące o wysokich kompetencjach z osobami z pozostałych grup. Osoby przypominające sobie wydarzenie świadczące o kompetencjach cechowały się samooceną (indywidualną i kolektywną) wyższą o 0,42 punktu, prawdopodobieństwo wzrostu samooceny wynosiło $p =$ 0,90. W ramach drugiego kontrastu porównano osoby przypominające sobie wydarzenia świadczące o moralności z osobami z pozostałych grup. Osoby z tej pierwszej grupy cechowały się samooceną niższą o 0,01 punktu od pozostałych grup, prawdopodobieństwo wzrostu wynosiło $p =$ = 0,49. W ramach trzeciego kontrastu porównano osoby przypominające sobie wydarzenia świadczące o cieple z resztą grup. W tym przypadku osoby przypominające sobie o cieple cechowały się samooceną niższą o 0,19 od pozostałych grup, prawdopodobieństwo wzrostu wynosiło $p =$ 0,28. \\

W końcu sprawdzono, czy wymiary treściowe przyczyniające się do wzrostu samooceny różniły się w zależności od poziomu tożsamości. W tym celu porównano samoocenę łącznie: osób przypominających sobie wydarzenia świadczące o indywidualnych kompetencjach, osób przypominających sobie wydarzenia świadczące o cieple siebie jako Polak, i osób przypominających sobie wydarzenia świadczące o moralności Polaków z poziomem samooceny u osób w pozostałych warunkach. Wśród uczestników w pierwszych trzech warunkach zaobserwowano wzrost samooceny wynoszący 1,66 w porównaniu do osób w pozostałych grupach, prawdopodobieństwo wzrostu wynosiło $p =$ 0,98 (por. \ref{fig:study3}).\\

Dodatkowe analogiczne analizy przeprowadzono osobno dla każdego wariantu eksperymentu (ja vs. ja jako Polak vs. my Polacy). Osoby przypominające sobie o indywidualnych kompetencjach cechowały się samooceną wyższą o 0,57 od osób w pozostałych warunkach, prawdopobieństwo wzrostu wynosiło $p =$ 0,93. Osoby przypominające sobie o cieple siebie jako Polaka cechowały się samooceną wyższą o 0,58 niż osoby z pozostałych grup dla tego wariantu, prawdopodobieństwo wzrostu wynosiło $p =$ 0,93. Nie zaobserwowano jednak różnicy pomiędzy osobami przypominającymi sobie o kolektywnej moralności a osobami z pozostałych grup, różnica była pozytywna z $p =$ 0,57. Sprawdzono jednak, czy osoby przypominające sobie o kolektywnej moralności deklarowały wyższą kolektywną samoocenę niż osoby przypominające sobie wydarzenia świadczące o cieple lub kompetencjach. W tym przypadku różnica wynosiła 0,62 i była pozytywna z prawdopodobieństwem, $p =$ ???.

\subsubsection{Dyskusja}
Celem Badania 3 było sprawdzenie, czy aktywizacja pozytywnych aspektów Ja dotyczących wymiarów kompetencji, moralności lub ciepła podniesie poziom samooceny. Wyniki poprzednich badań sugerowały, że pozytywne aspekty Ja dotyczące wymiaru kompetencji będą wpływały jedynie na poziom indywidualnej samooceny, natomiast pozytywne aspekty Ja dotyczące wymiaru ciepła i moralności będą wpływały na poziom samooceny odpowiednio siebie jako członka grupy i dla samooceny Polaków jako grupy. Analiza nie uwzględniająca wielopoziomowości Ja wykazała, że myślenie o aspektach związanych z wymiarem kompetencji jest głównym czynnikiem prowadzącym do wzrostu samooceny (indywidualnej i kolektywnej). Mimo to, analiza uwzględniająca podział na Ja indywidualne i Ja kolektywne potwierdziła wyniki poprzednich badań sugerujące, że poziom tożsamości jest moderatorem tej zależności: przypomnienie o wydarzeniach świadczących o kompetencjach prowadziło do wzrostu indywidualnej samooceny uczestników, przypomnienie o wydarzeniach świadczących o cieple prowadziło do wzrostu samooceny siebie jako Polaka, natomiast przypominanie wydarzeniach świadczących o moralności grupy prowadziło do wzrostu kolektywnej samooceny (bardziej niż przypominanie o wydarzeniach świadczących o kompetencjach lub cieple grupy własnej). \\

Podsumowując, choć zaobserwowane różnice okazały się niewielkie dostarczyły one potwierdzenia dla modelu postulowanego w niniejszej rozprawie.

\begin{table*}[htbp]
\vspace*{2em}
\centering
\begin{threeparttable}
\caption{Aktywizacja kompetencji, moralności i ciepła a poziom samooceny indywidualnej, siebie jako Polaka i Polaków jako grupy -- podsumowanie średnich grupowych.}
\label{tab:3}
\bgroup
\def\arraystretch{0.85}
\begin{tabular}{lrrrr}
\midrule
                        & Ja        & Ja Polak  & My Polacy & \emph{Średnie brzegowe}\\
\midrule
Kontrolna               & 2,61      & 2,70      & 2,90      & 2,75 \\
Kompetencje             & 3,05      & 2,84      & 2,82      & 2,92 \\
Moralność               & 2,84      & 2,56      & 2,99      & 2,81 \\
Ciepło                  & 2,69      & 2,82      & 2,72      & 2,76 \\

\emph{Średnie brzegowe} & 2,80      & 2.73      & 2.86      & 2,81 \\

\bottomrule
\end{tabular}
\egroup
\begin{tablenotes}[para, flushleft]
{\small
\textit{Nota.} $N = 184$
}
\end{tablenotes}
\end{threeparttable}
\end{table*}

\begin{figure*}[htbp]
   \centering
   \fitfigure{study3.pdf}
   \caption{Aktywizacja kompetencji, moralności i ciepła a poziom samooceny indywidualnej i kolektywnej -- zobrazowanie rozkładów brzegowych dla nieaddytywnych/interakcyjnych odchyleń od średniej ogólnej.}
   \label{fig:study3}
\end{figure*}



\newpage
\subsection{Badanie 4}

W dotychczasowych badaniach aspekty związane z moralnością były głównymi wyznacznikami samooceny kolektywnej, natomiast aspekty związane z kompetencjami nie odgrywały znaczącej roli w wyznaczaniu kolektywnej samooceny (ani Polaków ani Amerykanów). Zgodnie z modelem perspektywy sprawcy i biorcy \parencite{abele2014communal} przyjęcie perspektywy biorcy prowadzi do skupienia poznawczego na treściach wspólnotowych (w tym moralności). W świetle tego, możliwa interpretacja uzyskiwanych dotychczas wyników mówi, że myślenie o różnych aspektach kolektywnego Ja domyślnie związane jest z przyjmowaniem perspektywy biorcy.\\

Jeżeli jednak przyjąć przedstawioną interpretacją, to można również oczekiwać, że wzbudzenie perspektywy (kolektywnego) sprawcy, tj. przedstawienie grupy własnej jako podmiotu zabezpieczającego dobra i interesy Ja, prowadzić będzie do wzrostu znaczenia wymiaru kompetencji. W efekcie wzrośnie rola wymiaru kompetencji jako wyznacznika kolektywnej samooceny. Tą hipotezę sprawdzano w przedstawionych poniżej dwóch badaniach -- badaniu pilotażowym i w badaniu właściwym.\\

\subsubsection{Badanie pilotażowe}
Przyjęto, że jednym ze sposobów skłonienia uczestników do przyjęcia perspektywy (kolektywnego) sprawcy może być przypomnienie o zagrożeniu dla egzystencji grupy własnej. Jeżeli ten rodzaj manipulacji prowadzi do przyjęcia perspektywy (kolektywnego) sprawcy, to pod jej wpływem należy oczekiwać wzrostu skupienia na cechach dotyczących wymiaru kompetencji i wzrostu ich dostępności poznawczej w ramach obrazu grupy własnej. Sprawdzenie tej zależności było celem przedstawionego badania pilotażowego.

\paragraph{Uczestnicy i metoda}.
W badaniu wzięło udział 170 (104 kobiety i 66 mężczyzn) użytkowników internetowego panelu badawczego ($M$ = 37,29; $SD$ = 11,07).\\

Uczestnicy byli losowo przydzielani do jednego z dwóch warunków. W ramach warunku kontrolnego uczestnicy od razu przechodzili do głównej części badania, natomiast w przypadku warunku z zagrożeniem, uczestnicy byli proszeni o zapoznanie się krótkim artykułem prasowym:
\blockquote{``Żyjemy w czasach permanentnego kryzysu. Te czasy stanowią realne zagrożenie dla wszystkich Polaków'' -- mówi profesor UW. Jeszcze nie ucichły echa kryzysu na Ukrainie -- i związanego z nim niepokoju: Gdzie zatrzyma się Rosja Putina? -- a już okazuje się, że możemy stać przed poważniejszym, bardziej długotrwałym problemem. Wojna w Syrii zapoczątkowała falę emigracji, która zalewa od kilku miesięcy Europę. Niedawne tragiczne wydarzenia na ulicach Paryża, pokazują czym kończy się lekceważenie ISIS. To wszystko skłania do zadania pytania: Czy Polacy mogą czuć się bezpieczni? Zdaniem Profesora UW Krzysztofa Wilczka, ta sytuacji stanowi zagrożenie dla wszystkich europejczyków, również Polaków. ``Polacy -- od kilkudziesięciu lat żyjący we względnej stabilizacji i spokoju -- zdążyli już zapomnieć czym jest zagrożenie dla bezpieczeństwa'' -- mówi Profesor -- ``Obawiam się, że w ciągu kilku lub kilkunastu miesięcy ten obserwowany na razie tylko poza naszymi granicami kryzys, przywędruje również do Polski i Polaków.''}

W treści artykułu pominięto określenia lub stwierdzenia odwołujące się wprost do wymiarów kompetencji, moralności i ciepła.\\

W dalszej kolejności u wszystkich uczestników dokonywano pomiaru dostępności poznawczej cech odnoszących się do grupy własnej. W tym celu zastosowano procedurę opartą na pomyśle \textcite{dodgson1998self}. Uczestników poproszono o kategoryzację wyświetlanych kolejno na ekranie cech na te pasujące lub niepasujące do Polaków. Na listę składało się 80 cech odnoszących się do wymiaru sprawczości (np. \emph{sprytni, bezradni}) i wspólnotowości (\emph{przyjaźni, chciwi}); połowę wszystkich cech stanowiły te odnoszące się do braku sprawczości i wspólnotowości. Główną zmienną w tym zadaniu był czas z jakim uczestnicy klasyfikowali wyświetlane cechy jako pasujące do Polaków.

\paragraph{Wyniki i dyskusja}
Analizę wyników przeprowadzono w schemacie 2 (zagrożenie: brak vs. obecne) x 2 (wartościowość cech: pozytywne vs. negatywne) x 2 (wymiar: sprawczość vs. wspólnotowość) o ostatnimi dwoma wymiarami wewnątrz osób i czasem reakcji jako zmienną zależną.\\

Niezależnie od wymiaru i warunku cechy pozytywne były kategoryzowane jako pasujące do Polaków szybciej niż cechy negatywne. Jednak ta różnica była znacząco większa w przypadku cech dotyczących wymiaru sprawczości niż wspólnotowości. Dodatkowo, była ona większa w przypadku warunku z zagrożeniem niż w warunku kontrolnym. W końcu, wykazano obecność potrójnej interakcji zagrożenia, wartościowości i wymiaru. Wzbudzenie egzystencjalnego zagrożenia prowadził do wzrostu dostępności poznawczej cech związanych ze sprawczością grupy własnej i spadku dostępności poznawczej cech związanych z brakiem sprawczości grupy własnej. Uzyskane zależności przedstawiono na Rycinie \ref{fig:studyP}.\\

Podsumowując, wyniki przeprowadzonego badania pilotażowego dostarczyły poparcia dla tezy, że wzbudzenie egzystencjalnego zagrożenia jest skutecznym sposobem na wzbudzenie perspektywy (kolektywnego) sprawcy.

\begin{figure*}[htbp]
   \centering
   \fitfigure{studyPilot.pdf}
   \caption{Szybkość kategoryzacji, jako charakterystyczne dla Polaków, pozytywnych i negatywnch cech sprawczych i wspólnotowych a obecność zagrożenia.}
   \label{fig:studyP}
\end{figure*}

\subsubsection{Badanie właściwe -- metoda}

Po uzyskaniu wstępnych wyników wskazujących, że wzbudzenie zagrożenia dla grupy własnej może prowadzić do przyjęcia  perspektywy (kolektywnego) sprawcy przystąpiono do realizacji badania właściwego. Postanowiono sprawdzić, czy przypomnienie o zagrożeniu dla grupy własnej zwiększy znaczenie wymiaru kompetencji jako wyznacznika kolektywnej samooceny.

\paragraph{Uczestnicy i schemat badania}
W eksperymencie wzięło udział $N=343$ (191 kobiet, 140 mężczyzn, 12 osób nie określiło swojej płci) użytkowników internetowego panelu badawczego w wieku od 18 od 69 lat ($M$ = 36,21; $SD$ = 12,82). Po wyrażeniu zgody na udział w badaniu uczestnicy byli losowo przydzielani do jednego z dwóch warunków: kontrolnego ($n = 170$) lub warunku z zagrożeniem ($n = 173$). Po wypełnieniu kwestionariusza każdy uczestnik był przekierowywany na stronę, na której wyjaśniano cel badania i zastosowaną manipulację eksperymentalną.

\paragraph{Procedura manipulacji}
Każdy uczestnik w warunku z zagrożeniem był przekierowywany na początku eksperymentu na stronę zawierającą spreparowany artykuł omawiający obecne zagrożenia dla Polaków i proszony o zapoznanie się z jego treścią. Aby uniknąć przypadkowego pominięcia artykułu specjalny skrypt na stronie uniemożliwiał uczestnikom przejście do kolejnych części badania przed upływem 30 sekund. Po upływie określonego czasu uczestnicy w warunku z zagrożeniem przystępowali do wypełniania kwestionariusza. Uczestnicy w grupie kontrolnej przystępowali do wypełniania kwestionariusza od razu po rozpoczęciu eksperymentu. \\

\paragraph{Miary}
\subparagraph{Skala kompetencji, moralności i ciepła}
Natychmiast po zapoznaniu się z artykułem (lub w warunku kontrolnym, po rozpoczęciu badania) uczestnicy w obu warunkach oceniali Polaków (\emph{my Polacy jesteśmy...}) przy użyciu 21 pozytywnych cech związanych z wymiarami kompetencji (np. \emph{zdolni, umiejętni}), moralności (np. \emph{uczciwi, sprawiedliwi}), lub ciepła (np. \emph{przyjaźni, sympatyczni}). W przypadku każdej cechy uczestnicy oceniali czy pasuje ona do Polaków czy nie przy użyciu 7-stopniowej skali od 1 -- \emph{Zdecydowanie nie} do 7 -- \emph{Zdecydowanie tak}.\\
Zastosowana wersja skali została zapożyczona z badań Wojciszke i współpracowników (niepublikowany raport z badań). Poszczególne subskale charakteryzowały się wysoką zgodnością wewnętrzną. Podsumowanie charakterystyk przedstawiono w Tabeli \ref{tab:5}.

\subparagraph{Skala kolektywnej samooceny}
Po dokonaniu ocen uczestnicy wypełniali składającą się z 13 pozycji skalę kolektywnej samooceny. Zastosowana skala była identyczna do skal zastosowanych w badaniach 1 i 3, a zawarte w niej stwierdzenia były przedstawione w formie odnoszącej się do Polaków jako grupy: ``my, Polacy'' Zadaniem każdego z uczestników była ocena na ile zgadzają się z każdym ze stwierdzeń na skali od 1 -- \emph{Zdecydowanie się nie zgadzam} do 4 -- \emph{Zdecydowanie się zgadzam}. Podsumowanie charakterystyk skali przedstawiono w Tabeli \ref{tab:5}.

\begin{table*}[htbp]
\vspace*{2em}
\centering
\begin{threeparttable}
\caption{Podstawowe statystyki opisowe dla skal użytych w Badaniu 4}
\label{tab:5}

\begin{tabular}{lrrrrr}

\midrule
& & \multicolumn{2}{c}{Kontrolna} & \multicolumn{2}{c}{Zagrożenie}  \\
\cline{3-6}
& $\alpha$ & $M$ & $SD$ & $M$ & $SD$  \\
\midrule
Kompetencje            & 0,93 & 5,28 & 1,00 & 5,18 & 1,05  \\
Moralność              & 0,94 & 4,71 & 1,12 & 4,52 & 1,20  \\
Ciepło                 & 0,93 & 5,06 & 1,03 & 4,82 & 1,15  \\
Kolektywna samoocena   & 0,84 & 2,94 & 0,47 & 2,81 & 0,52  \\

\bottomrule

\end{tabular}

\begin{tablenotes}[para,flushleft]
{\small
\textit{Nota.} $N$ = 343
}
\end{tablenotes}
\end{threeparttable}
\end{table*}

\subparagraph{Skala poczucia zagrożenia} Aby ocenić skuteczność manipulacji poproszono, aby każdy uczestnik ocenił prawdopodobieństwo (na skali od 0 do 100) zajścia każdego z trzech zdarzeń: \emph{}. Oceny charakteryzowały się zadowalającą zgodnością wewnętrzną, $\alpha$ = 0,69.

\subsubsection{Badanie właściwe -- wyniki}
\paragraph{Sprawdzenie skuteczności manipulacji}
Aby ocenić skuteczność manipulacji porównano średnie ocen prawdopodobieństwa zajścia zagrażających zdarzeń w warunku kontrolnym ($M$ = 40,08; $SD$ = 16,69) i w warunku z aktywizacją poczucia zagrożenia ($M$ = 45,38; $SD$ = 18,19). Przy obliczaniu średnich i ich różnicy uwzględniono niepewność związaną z błędem pomiarowym zmiennej zależnej (obliczonym według wzoru \ref{eq:err}). Średnia ocen wystąpienia zagrażających zdarzeń okazała się większa w warunku z aktywizacją zagrożenia niż w warunku kontrolnym, z prawdopodobieństwiem $p$ = 0,99. Standaryzowana różnica średnich wynosiła, d = 0,30 i z prawdopodobieństwem 95\% mieściła się w przedziale: [0,04;0,56]. Choć wzrost poczucia zagrożenia okazał się być subtelny, manipulacja eksperymentalna była skuteczna.

\paragraph{Zagrożenie a wyznaczniki samooceny}
Aby sprawdzić w jaki sposób zagrożenie decydować będzie o wyznacznikach kolektywnej samooceny skonstruowano model analogiczny do tego stosowanego w Badaniach 1 i 2 (patrz: Równanie \ref{eq:1}), tj. sprawdzono w jakim stopniu kompetencje, moralność i ciepło będą decydować o kolektywnej samoocenie, osobno w warunku kontrolnym i w warunku z zagrożeniem. Ponieważ, użyte miary charakteryzowały się bardzo wysoką zgodnością wewnętrzną nie uwzględniano niepewności wynikającej z błędu pomiarowego. Jako rozkład prior dla parametrów w warunku kontrolnym przyjęto wyniki uzyskane w ramach Badania 2 (patrz górna część Tabeli \ref{tab:2}). W warunku z zagrożeniem przyjęto nieinformatywny rozkład prior dla parametrów równania regresji\footnote{Aby otrzymać rozkład parametrów posterior wykorzystano algorytm MCMC zaimplementowany w programie Stan z 4000 iteracji. Wartości PSRF nie przekraczały, R = 1,01 i nie wskazywały na konieczność odrzucenia symulacji ze względu na brak konwergencji}. Podsumowanie parametrów przedstawiono w Tabeli \ref{tab:4}. \\

W przypadku osób w warunku kontrolnym poziom kolektywnej samooceny był uzależniony od ich ocen Polaków na wymiarach moralności, z prawdopodobieństwem, $p =$ 0,99, i ciepła, z $p =$ 0,97. U osób w ramach tego warunku, poziom kolektywnej samooceny nie był uzależniony od ocen Polaków na wymiarze kompetencji, $p =$ 0,73. \\

U osób u których wzbudzono poczucie zagrożenia, poziom kolektywnej samooceny okazał się być związany przede wszystkim z postrzeganiem Polaków jako kompetentnych, $p =$ 0,99. Z kolei postrzeganie Polaków jako moralnych i ciepłych okazało się mieć mniejsze znaczenie, odpowiednio $p =$ 0,96 i $p =$ 0,91.\\

W następnej kolejności, dla każdego wymiaru porównano wartości parametrów w warunku kontrolnym i w warunku z zagrożeniem. Dla wymiaru kompetencji wartość parametru w warunku z zagrożeniem była większa niż w warunku kontrolnym z prawdopodobieństwem, $p =$ 0,97. Parametry dla wymiarów moralności i ciepła były z kolei mniejsze w warunku z zagrożeniem niż w warunku kontrolnym, ale różnice pomiędzy warunkami były małe -- mniejsze od 0 z prawdopodobieństwem $p = $ 0,67 dla wymiaru moralności i $p =$ 0,56 dla wymiaru ciepła.\\

Cały model wyjaśniał z $p =$ 0,95 od 5 do 16\% wariancji kolektywnej samooceny. Porównanie linii regresji pomiędzy dwoma warunkami przedstawiono na Rycinie \ref{fig:study4}.\\

\begin{table*}[htbp]
\vspace*{2em}
\centering
\begin{threeparttable}
\caption{Kompetencje, moralność i ciepło jako predyktory kolektywnej samooceny a aktywizacja zagrażających treści -- podsumowanie rozkładów brzegowych parametrów modelu.}
\label{tab:4}
\begin{tabular}{lrrrrrr}

\midrule
 &
\multicolumn{3}{c}{Kontrolna} &
\multicolumn{3}{c}{Zagrożenie} \\
\cline{2-7}
 & $M_{post.}$    & $SD$   & $95\%\ CI$   & $M_{post.}$    & $SD$   & $95\%\ CI$  \\
\midrule
 Stała       &  1,87 & 0,16 &  1,55;2,18 &  1,34 & 1,02 &  1,02;1,66 \\
 Kompetencje &  0,02 & 0,04 & -0,06;0,09 &  0,14 & 0,05 &  0,04;0,24 \\
 Moralność   &  0,11 & 0,04 &  0,04;0.19 &  0,08 & 0,05 & -0,02;0,19 \\
 Ciepło      &  0,09 & 0,04 &  0,00;0,17 &  0,08 & 0,06 & -0,04;0,20 \\
\bottomrule
\end{tabular}
\begin{tablenotes}[para,flushleft]
{\small
\textit{Nota.} $N = 343$. Dobroć dopasowania modelu: $R^2$ = 0,10 95\% CI = [0,05;0,16].
}
\end{tablenotes}
\end{threeparttable}
\end{table*}

\begin{figure*}[htbp]
   \centering
   \fitfigure{study4.pdf}
   \caption{Kompetencje, moralność i ciepło jako wyznaczniki kolektywnej samooceny w warunku kontrolnym vs. w warunku z zagrożeniem. Na rysunku przedstawiono linie dopasowania wraz ze standardowymi błędami oszacowania.}
   \label{fig:study4}
\end{figure*}

\subsubsection{Dyskusja}
Celem Badania 4 było sprawdzenie w jaki sposób aktywizacja perspektywy kolektywnego sprawcy (poprzez wprowadzenie informacji o zagrożeniu bezpieczeństwa grupy własnej) wpłynie na siłę zależności pomiędzy wymiarami sprawczości i wspólnotowości a kolektywną samooceną. Stwierdzono, że w warunku kontrolnym, zgodnie z wcześniejszymi wynikami kolektywna samoocena (\emph{my Polacy}) była przewidywana jedynie przez wymiary wspólnotowości -- moralność i ciepło. Natomiast, w warunku z zagrożeniem jedynym predyktorem kolektywnej samooceny okazał się wymiar kompetencji. Co więcej, uzyskane wyniki wskazały na wzrost znaczenia wymiaru kompetencji w sytuacji zagrożenia.\\

Uzyskane wyniki są zgodne z przewidywaniami co do roli perspektywy sprawcy i biorcy w determinowaniu wyznaczników kolektywnej samooceny. O ile szczególne znaczenie wymiarów wspólnotowych wynika z przyjmowania perspektywy biorcy w myśleniu o grupie własnej, tak wzrost znaczenia roli kompetencji w sytuacji kolektywnego zagrożenia może być efektem związanego z tym zagrożeniem przyjęcia perspektywy kolektywnego sprawcy -- tj. zmiana perspektywy może być próbą zmiany nastawienia na takie, które ma pomóc w radzeniu sobie z zagrożeniem.\\

Zgodnie z przewidywaniami modelu sprawcy i biorcy, przyjęcie perspektywy kolektywnego sprawcy prowadzić powinno do porzucenia perspektywy biorcy i spadku znaczenia wymiarów wspólnotowych. Uzyskane wyniki nie potwierdzają jednak jednoznacznie takiej zależności. Z jednej strony, wymiary wspólnotowe okazały się być predyktorami kolektywnej samooceny jedynie w warunku kontrolnym i nie mieć szczególnego znaczenia w warunku z zagrożeniem. Z drugiej strony, spadek znaczenia wymiarów wspólnotowych (tj. porównanie pomiędzy warunkami) okazał się być niewielki. O ile Ja kolektywne faktycznie wiąże się domyślnie z przyjmowaniem perspektywy biorcy, ten wynik mógłby być oczekiwany -- tj. trudno oczekiwać, aby krótkie oddziaływanie prowadziło do zmian w przypadku silnie ukształtowanych schematów. Niemniej dokładny charakter tej relacji powinien być przedmiotem przyszłych badań. \\

\newpage
\section{Ogólna dyskusja}

\subsection{Podsumowanie uzyskanych wyników}

Celem przedstawionej linii badań było wskazanie, które aspekty indywidualnego i kolektywnego Ja wyznaczają pozytywną samoocenę. Dotychczasowe teorie opisujące funkcje samooceny \parencite[np.,][]{pyszczynski2004people, leary2000nature} nie dawały jasnej odpowiedzi na pytanie, z czego może wynikać wysokie poczucie własnej wartości. Co więcej, dotychczasowe teorie tylko w niewielkim stopniu próbowały uwzględniać fakt wielopoziomowości Ja, na które składa się nie tylko poczucie odrębności i unikalności siebie, ale również poczucie relacji z innymi i przynależności do grup społecznych. W efekcie, w różnych cytowanych pracach wskazywano na różne podstawowe aspekty Ja (przede wszystkim aspekty związane ze sprawczością lub wspólnotowością), które miałyby mieć szczególne znaczenie dla poczucia własnej wartości.\\

Jedno z możliwych rozwiązań istniejącej kontrowersji odwołuje się do poczynionej już przez \textcite{markus1991culture} obserwacji o różnych podstawach samooceny w ramach Ja niezależnego (w kulturach indywidualistycznych) i Ja współzależnego (w kulturach kolektywistycznych). O ile podstawą Ja niezależnego jest ekspansja Ja, to podstawą Ja współzależnego jest ograniczanie Ja pozwalające na lepsze dopasowanie do kontekstu społecznego. W ramach niniejszej pracy nie dokonywano jednak porównań międzykulturowych (pomiędzy kulturami indywidualistycznymi a kolektywistycznymi), ale przyjęto, że nawet w ramach kultury indywidualistycznej na tożsamość człowieka składają się zarówno aspekty indywidualne, jak i te związane z jego przynależnością do grup i kategorii społecznych \parencite[np.,][]{brewer1996we, turner1987rediscovering}. \\

W ramach podstawowej hipotezy weryfikowanej w niniejszej linii badań postulowano, że wyznaczniki pozytywnej samooceny będą zależały od poziomu Ja (indywidualnego vs. kolektywnego). W pierwszych trzech badaniach dokonywano bezpośrednich porównań układu predyktorów pozytywnej samooceny pomiędzy poziomem indywidualnym a kolektywnymi poziomami tożsamości. We wszystkich trzech badaniach zadbano o to, aby warunki indywidualne i kolektywne były do siebie jak najbardziej zbliżone, tj. użyto niemal identycznych pozycji kwestionariuszowych i identycznych sposobów oddziaływania eksperymentalnego. Jedyną różnicę stanowiły formy adresowania Ja (ja vs. ja jako przedstawiciel grupy vs. my grupa).\\

W każdym z trzech badań układ predyktorów pozytywnej samooceny okazał się inny. Znaczące różnice obserwowano zarówno w badaniach korelacyjnych jak i eksperymentalnym. Różnice obserwowano zarówno na próbie polskiej jak i amerykańskiej, na cechach dotyczących pozytywnych jak i negatywnych aspektów Ja. W końcu różnice zaobserwowano na trzech różnych miarach samooceny. Podsumowując, zebrane dane dość silnie wskazują, że wraz z przechodzeniem na inne poziomy tożsamości dochodzi do zmiany centralności różnych aspektów Ja. W efekcie w zależności od poziomu Ja zmieniają się wyznaczniki pozytywnej samooceny.\\

Na podstawie dotychczasowej literatury postawiono hipotezę, że podstawą indywidualnej samooceny będą cechy związane z wymiarem sprawczości. W dwóch badaniach korelacyjnych uzyskano potwierdzenie dla szczególnej roli tego wymiaru (wspólny efekt kompetencji na samoocenę w ramach Badania 1 i 2 wynosił $\beta$ = 0,31). Dodatkowego potwierdzenia dostarczyło badanie eksperymentalne w którym jedynie aktywizacja aspektów związanych ze sprawczością prowadziła do wzrostu pozytywnej samooceny. Nie wykazano szczególnej roli aspektów dotyczących wspólnotowości dla indywidualnej samooceny. Podążając za sugestią \textcite{brambilla2014importance}, we wszystkich przedstawionych badaniach, cechy związane z wspólnotowością podzielono na te świadczące o cieple (towarzyskości) i moralności. Jeżeli już, to efekty dla obu tych wymiarów okazywały się być niespójne. Zaobserwowano wprawdzie niewielki efekt ciepła, ale tylko w przypadku badania korelacyjnego na próbie amerykańskiej. Zaobserwowano również efekt dla cech świadczących o braku ciepła dla indywidualnej samooceny, ale był on odwrotny do oczekiwań (tj. przy stałym poziomie pozostałych predyktorów osoby przypisujące sobie więcej \emph{zimna} cechowały się wyższym poziomem samooceny).\\

W przypadku kolektywnej samooceny, w ramach przedstawionych hipotez postulowano, że jej podstawą będą cechy związane z wymiarami wspólnotowości. Ja kolektywne zoperacjonalizowano jako w mniejszym lub większym stopniu depersonalizujące (odpowiednio ja jako Polak/Amerykanin vs. my Polacy/Amerykanie). Ogólnie, uzyskiwane wyniki dostarczyły potwierdzenia dla stawianej hipotezy. We wszystkich przedstawionych badaniach systematycznie obserwowano związek ciepła oraz moralności z kolektywną samooceną. Zaobserwowano w tym przypadku interesujący, choć nieoczekiwany układ zależności. Aspekty związane z ciepłem okazały się być predyktorami zwłaszcza samooceny siebie jako przedstawiciela grupy społecznej (w ramach Badania 1, 2 wspólny efekt ciepła wynosił, $\beta$ = 0,45), natomiast aspekty związane z moralnością okazały się być predyktorami samooceny grupy jako całości (w ramach Badań 1, 2 i warunku kontrolnego w Badaniu 4 wspólny efekt moralności wynosił, $\beta$ = 0,26). Dokładne wyjaśnienie tej różnicy powinno być celem przyszłych projektów. \\

W sumie, w pierwszych trzech z prezentowanych badań obserwowano związek wymiaru kompetencji z indywidualną samooceną i brak związku wymiaru kompetencji z kolektywną samooceną. Taki układ zależności jest zgodny z modelem perspektywy sprawcy i biorcy \parencite{abele2014communal} jeżeli przyjąć, że ludzie dokonując oceny indywidualnego Ja przyjmują perspektywę sprawcy, natomiast dokonując oceny kolektywnego Ja przyjmują perspektywę biorcy. Jednak zgodnie z postulowanym modelem, również w przypadku kolektywnego Ja powinno być możliwe wzbudzenie perspektywy sprawcy, tj. przedstawienie grupy własnej jako realizującej cele podmiotu i sprawującego kontrolę na rzeczywistością. Istotnie, istnieją dane potwierdzające taką możliwość \parencite{fritsche2013power}. W ramach tych badań wzbudzenie poczucia braku kontroli nad biegiem zdarzeń prowadziło do konstruowania grupy własnej jako kolektywnego sprawcy pozwalającego na odzyskanie poczucia kontroli.\\

Podobny sposób myślenia przyjęto w ostatnim z raportowanych w niniejszej pracy badań. Poprzez wzbudzenie poczucia zagrożenia dla bezpieczeństwa grupy, skłoniono do przyjęcia perspektywy (kolektywnego) sprawcy. Przyjęcie tej perspektywy skutkowało wzrostem znaczenia aspektów dotyczących sprawczości grupy jako wyznaczników pozytywnej kolektywnej samooceny. To badanie jest częściowym potwierdzeniem dla mechanizmu, który miałby odpowiadać za to, że wraz z poziomami Ja zmieniają się wyznaczniki samooceny.\\

\subsection{Ograniczenia prezentowanych badań i przyszłe plany}

Przedstawionego ciągu badań z całą pewnością nie można określić jako zakończonego. Biorąc pod uwagę szereg słabości projektu, ale również szereg nowopojawiających się koncepcji i metodologii, wyjaśnienie wszystkich kwestii będzie wymagać dalszych badań na zróżnicowanych próbach. \\

\subsubsection{Uniwersalność uzyskanych wyników}

Kwestia uniwersalności prezentowanych wyników zasługuje w tym miejscu na szczególną uwagę. Niewątpliwą słabością prezentowanych badań był fakt ich realizacji na relatywnie mało licznych i niereprezentatywnych próbach. Uzyskiwane efekty pojawiały się wprawdzie systematycznie w kolejnych badaniach, jednak fakt nie zaobserwowania niektórych zależności mógł być wynikiem zbyt małej próby lub zwyczajnie wynikać z pewnych charakterystyk części populacji. Wnioski płynące z tych badań powinny być z całą pewnością zreplikowane w przyszłości na próbie reprezentatywnej. \\

Replikacja uzyskanych wyników powinna być również przedmiotem przyszłych badań międzykulturowych. W ramach przedstawionej linii zrealizowano badania na próbie polskiej i amerykańskiej. Jest to jednak stanowczo za mało, aby uznać że jest to zależność nie tyle uniwersalna, co przynajmniej replikująca się na przestrzeni kultur -- wszak kultura polska i amerykańska nie różnią się znacznie. W szczególności, w ramach przyszłych badań należy sprawdzić czy przedstawione wyniki replikują się nie tylko w kulturach indywidualistycznych, ale również w kulturach kolektywistycznych \parencite[patrz np.,][]{markus1991culture}. \\

W prezentowanych badaniach skupiono się wyłącznie na kolektywnych aspektach tożsamości związanych z przynależnością narodową. Bycie reprezentantem danej nacji jest niewątpliwie jednym z centralnych aspektów tożsamości kolektywnej współczesnych ludzi. Nie można jednak zapominać, że na tożsamość kolektywna składa się szereg różnych innych identyfikacji. Ludzie identyfikują się z grupami lub kategoriami o różnej wielkości lub różnym charakterze \parencite[patrz np.,][]{lickel2000varieties}. W przyszłych badaniach należy sprawdzić wyznaczniki kolektywnej samooceny dla grup o różnym charakterze.\\

Problematyka uniwersalności wyników odnosi się nie tylko do replikacji na różnych grupach, ale również do możliwości replikacji przy użyciu różnych miar. W ramach przedstawionych badań stosowano jedynie deklaratywne miary samooceny oraz pytano o deklaratywne aspekty Ja (sprawczość i wspólnotowość). W tym kontekście szczególnie interesujące wyniki mogą przynieść badania nad miarami utajonymi, np. wykorzystujące test utajonych skojarzeń jako miarę samooceny \parencite{greenwald2000using} lub procedurę podprogowego prymowania ewaluatywnego \parencite{dijksterhuis2004like}. Inną potencjalną linię stanowią badania z użyciem miar behawioralnych.\\

\subsubsection{Potwierdzenie modelu perspektywy sprawcy i biorcy?}

W niniejszej pracy za podstawę teoretyczną służącą wyjaśnieniu wyników przyjęto model sprawcy i biorcy \parencite{abele2014communal}. Wyniki Badania 4 okazały się być zgodne z przewidywaną przez model zmianą wyznaczników samooceny wraz ze zmianą perspektywy: z perspektywy biorcy na perspektywę sprawcy. Niemniej udowodnienie znaczenia tego modelu wymagać będzie dodatkowych badań.\\

W szczególności, w przyszłych badaniach należy bezpośrednio sprawdzić, czy poziomy Ja indywidualnego vs. kolektywnego są związane ze skłonnością do przyjmowania perspektywy sprawcy vs. perspektywy biorcy. Ponadto, w ramach przedstawionych badań sprawdzano efekty zmiany perspektywy dla wyznaczników kolektywnej samooceny. Należałoby jednak również zweryfikować komplementarną hipotezę, mówiącą że przyjęcie perspektywy biorcy prowadzić będzie do wzrostu znaczenia wspólnotowości jako wyznacznika samooceny.\\

Próbę weryfikacji tej ostatniej hipotezy, mówiącej że przekonanie o własnej wspólnotowości jest silniejszym predyktorem poczucia własnej wartości biorców niż sprawców, przedstawiła \textcite{bialobrzeska2015perspektywy} w swojej rozprawie doktorskiej. Wyniki okazały się jednak niekonkluzywne -- wartości siły efektu w ramach 4 badań $r$ wahały się od -0,11 do 0,23. \\

Co więcej, w Badaniu 4 dokonywano manipulacji perspektywą sprawcy poprzez wzbudzenie egzystencjalnego zagrożenia dla grupy własnej. O ile wyniki badania pilotażowego dostarczyły pewnego potwierdzenia dla skuteczności tej manipulacji jako wzbudzającej perspektywę kolektywnego sprawcy, w przyszłych badaniach należy bardziej bezpośrednio sprawdzić znaczenie przypominania o kolektywnym zagrożeniu; czy przykładowo nie prowadzi ono raczej do wzbudzenia poczucia braku kontroli \parencite[patrz, ][]{sullivan2010existential}, i przez to inny proces jest odpowiedzialny za obserwowane w Badaniu 4 zależności. \\

\subsection{Znaczenie projektu}

Pomimo szeregu niedoskonałości i potrzeby przeprowadzenia dalszych badań, przedstawiony projekt badawczy dostarcza szeregu dowodów wskazujących, że wyznaczniki samooceny zależą od poziomu Ja -- indywidualnego vs. kolektywnego. W końcowej części niniejszej rozprawy przedstawiono znaczenie teoretyczne, metodologiczne, oraz praktyczne uzyskanych wyników.\\

\subsubsection{Znaczenie teoretyczne uzyskanych wyników}


\paragraph{Znaczenie dla teorii samooceny} Przyjęcie, że wyznaczniki samooceny zależą od poziomu Ja, tj. samoocena indywidualna jest uwarunkowana aspektami związanymi z wymiarem sprawczości, natomiast samoocena kolektywna jest uwarunkowana aspektami związanymi z wymiarem wspólnotowości, może ułatwić zrozumienie kontrowersji dotyczących funkcji poczucia własnej wartości. Jak wskazano w pierwszym rozdziale niniejszej rozprawy, istnieje kilka teorii wyjaśniających do czego służy samoocena. Z jednej strony samoocena wspomaga ludzi w realizacji celów i implementacji działania \parencite{bandura1994self}, z drugiej strony informuje o zagrożeniu wykluczeniem społecznych \parencite{leary2000nature}. Wedle pierwszej z tych teorii, szczególne znaczenie dla samooceny będą miały aspekty Ja związane ze sprawczością, natomiast według drugiej aspekty związane z wspólnotowością. Aspekty związane ze sprawczością informują bowiem o adekwatności w radzeniu sobie z zadaniami, natomiast aspekty związane ze wspólnotowością informują o stopniu dopasowania do grupy społecznej. \\

Nieco mniej otwarcie do kwestii wyznaczników samooceny ustosunkowuje się teoria opanowania trwogi \parencite{pyszczynski2004people}. Według tej teorii, samoocena buforuje myśli związane ze śmiercią poprzez aktywizację kulturowo ukształtowanych konstruktów. Innymi słowy poczucie wyznawania pewnych uznawanych wartości, tj. bycia częścią większej kultury sprawia, że osoby mniej obawiają się własnej śmiertelności. Te wyznawane wartości mogą oczywiście odnosić się zarówno do aspektów związanych z wymiarem sprawczości jak i wspólnotowości. Biorąc jednak pod uwagę to, jak wielką wagę przykładają współczesne społeczeństwa do moralnego charakteru \parencite[np.,][]{goodwin2014moral}, spodziewać się można, że zgodnie z teorią opanowania trwogi to moralność będzie głównym wyznacznikiem samooceny.\\

Przedstawioną kontrowersję teoretyczną można rozwiązać przyjmując jedną z przedstawionych wyżej teorii i odrzucając wszystkie pozostałe. Można jednak również dążyć do unifikacji wymienionych teorii przyjmując, że być może samoocena nie jest konstruktem jednorodnym. Jednym ze sposobów na usystematyzowanie tej heterogeniczności samooceny jest uwzględnienie wielopoziomowości Ja, tj. przyjęcie innego charakteru samooceny indywidualnej i kolektywnej.\\

W przedstawionym podejściu, przewidywania teorii opisujących samoocenę jako inicjator działań należy tłumaczyć biorąc pod uwagę indywidualne aspekty Ja. Przewidywania teorii opisujących samoocenę jako socjometr należy tłumaczyć biorąc pod uwagę aspekty związane z Ja relacyjnym (kim jestem w stosunku do Innych lub w stosunku do grupy społecznej). Natomiast przewidywania teorii opisujących samoocenę jako bufor lęku przed śmiercią należy tłumaczyć biorąc pod uwagę kolektywne (najbardziej abstrakcyjne) aspekty Ja.\\

\paragraph{Znaczenie dla teorii tożsamości społecznej} W psychologii społecznej panuje konsensus co do tego, że świadomość przynależności do pewnych grup i kategorii społecznych stanowi istotny element Ja \parencite{ellemers2012group}. O ile jednak znaczna część prac dotyczących kolektywnego Ja (tj. kolektywnej samooceny) skupiała się na jej negatywnych aspektach, takich jak uprzedzenia i dyskryminacja \parencite[np.,][]{branscombe1994collective, crocker1990collective, jetten1997distinctiveness}, relatywnie niewiele prac omawiało pozytywne konsekwencje wysokiej kolektywnej samooceny. Oczywiście istnieją dowody na pozytywne konsekwencje kolektywnej samooceny. Przykładowo, \textcite{butler2005collective} wykazali, że wysoki poziom kolektywnej samooceny siebie jako przedstawiciela grupy zawodowej zmniejsza ryzyko wypalenia zawodowego.\\

Wnioski przedstawione w niniejszej pracy, wskazujące na szczególne znaczenie wspólnotowości dla kolektywnej samooceny pozwalają lepiej rozumieć zarówno pozytywne jak i negatywne konsekwencje pozytywnego wartościowania kolektywnych aspektów Ja. Z jednej strony pozwalają zrozumieć to, że identyfikacja z grupą własną może budować poczucie przynależności do wspólnoty podzielającej wartości etyczne i moralne, i przez to może redukować poziom niepokoju związanego z wizją własnej śmiertelności \parencite{castano2004case} oraz poczucie niepewności \parencite{hogg2000subjective}. Z drugiej strony, jeżeli chodzi o negatywne konsekwencje, przedstawione wnioski wpisują się w prace dotyczące roli kolektywnych emocji wstydu i winy -- związanych z moralną transgresją -- dla dyskryminacji i uprzedzeń \parencite[np.,][]{gausel2011concern}.\\

Interesujący z teoretycznego punktu widzenia jest również wynik wskazujący na inne podstawy samooceny siebie jako przedstawiciela grupy vs. grupy jako całości. O ile ten pierwszy rodzaj samooceny był wyznaczany przez postrzeganie siebie w kategoriach ciepła, ten drugi rodzaj był wyznaczany przez postrzeganie siebie w kategoriach moralności. Różnicę pomiędzy tymi dwoma przedstawieniami kolektywnego Ja można rozumieć przynajmniej na dwa sposoby.\\

Pierwszy sposób odwołuje się procesu depersonalizacji \parencite[np.,][]{turner1987rediscovering}, tj. tego na ile wyraziste są aspekty związane indywidualnym vs. kolektywnym Ja. W tym rozróżnieniu, samoocena grupy jako całości jest w większym stopniu zdepersonalizowana niż ocena siebie jako przedstawiciela grupy. Jeżeli tak jak pokazano w niniejszej rozprawie, samoocena \emph{nas Polaków/Amerykanów} jest wyznaczana przez aspekty dotyczące moralności w większym stopniu niż samoocena \emph{siebie jako Polaka/Amerykanina}, pozwala to lepiej rozumieć wyniki uzyskiwane przez \textcite{reicher1995social} wskazujące, że depersonalizacja sprzyja skłonności podporządkowywania się normom grupowym.\\

Drugi sposób rozumienia różnicy pomiędzy przedstawianiem siebie jako członka grupy i grupy jako całości odwołuje się do rozróżnienia zaproponowanego przez \textcite{brewer1996we}, na Ja relacyjne i Ja kolektywne. O ile ocena siebie jako przedstawiciela grupy zdaje się dotyczyć raczej tego pierwszego (tj. może być związana z postrzeganiem siebie w stosunku do innych Polaków), tak ocena grupy jako całości dotyczy raczej tego drugiego. Jeżeli tak jak sugerują \textcite{brewer1996we} podstawową motywacją w przypadku Ja relacyjnego jest dbałość o dobro innych, łatwiej zrozumieć dlaczego samoocena siebie jako przedstawiciela grupy jest wyznaczana przede wszystkim przez cechy związane z wymiarem ciepła (czyli z takim, które są korzystne dla innych).\\

\paragraph{Znaczenie dla modelu sprawcy i biorcy} Podstawową ramą teoretyczną dla niniejszej rozprawy stanowił model sprawcy i biorcy \parencite{abele2014communal}. Celem badań w niej przestawionych było określenie użyteczności tej koncepcji dla zrozumienia różnic pomiędzy indywidualnym i kolektywnym Ja. Warto w tym miejscu przestrzec przed zbyt pochopnym rozumieniem wniosków z nich płynących.\\

Jeden z najbardziej ogólnych postulatów modelu sprawcy i biorcy głosi, że oceny o charakterze sprawczym dominują w percepcji siebie, natomiast oceny o charakterze wspólnotowym dominują w percepcji innych \parencite[patrz,][]{wojciszke2005morality}. W związku z tym jedna z możliwych interpretacji przedstawionych w tej rozprawie wyników głosiłaby, że oceny siebie są wyznaczane przez wymiar sprawczości, natomiast oceny Polaków czyli innych są wyznaczane przez wymiar wspólnotowości. Jeżeli przyjąć taką interpretację, oryginalny aspekt niniejszej rozprawy będzie niewielki. Wydaje się jednak, że taka interpretacja jest błędna. Przynajmniej z dwóch powodów.\\

Po pierwsze, uczestnicy w prezentowanych badaniach nigdy nie oceniali Polaków/Amerykanów jako Innych. Wszystkie stwierdzenia dotyczące Polaków były operacjonalizowane jako `ja jako Polak/Amerykanin' lub `my Amerykanie'. Nawet jeżeli przyjąć, że kategoria `my' mogła wydawać się uczestnikom badań zbyt abstrakcyjna (i mogli oni w rzeczywistości oceniać po prostu Polaków), jest wielce prawdopodobne, że kategoria `ja jako Polak' odnosiła się raczej do Ja niż do Innych. Mimo to, w we wszystkich badaniach samoocena siebie jako przedstawiciela grupy była wyznaczana przez wymiar ciepła, a nie sprawczości.\\

Po drugie, wyniki Badania 4 pokazały że wzbudzenie zagrożenia dla dobra grupy własnej prowadzi do wzrostu znaczenia wymiaru sprawczości w wyjaśnianiu kolektywnej samooceny. Innymi słowy, przypomnienie o zagrożeniu sprawia, że wzorzec wyznaczników samooceny zaczyna przypominać ten, którego zgodnie z modelem perspektywy sprawcy i biorcy można spodziewać się w przypadku samooceny indywidualnej. Jedno z możliwych wyjaśnień tego efektu głosiłoby, że w warunku kontrolnym Polacy byli postrzegani jako inni, natomiast dopiero poczucie zagrożenia prowadziło do postrzegania ich jako części Ja. Jeżeli jednak tak by było, to pod wpływem zagrożenia można byłoby się również spodziewać wzrostu kolektywnej samooceny. Takiego wzrostu jednak nie zaobserwowano -- w rzeczywistości poziom kolektywnej samooceny w warunku z zagrożeniem był minimalnie mniejszy niż w warunku kontrolnym (patrz Tabela \ref{tab:5}). W świetle tej argumentacji mało prawdopodobne jest wyjaśnienie odwołujące się do zmiany w centralności grupy własnej w obrazie Ja.\\

Możliwe jest jednak, że inny proces odpowiada za zmianę wyznaczników kolektywnej samooceny -- być może jest nim zmiana w postrzeganiu stopnia współzależności z pozostałymi przedstawicielami grupy własnej. To wyjaśnienie byłoby zgodne z wynikami \textcite{wojciszke2008primacy}, gdzie postrzeganie innych jako symbiotycznych wobec Ja prowadziło do wzrostu znaczenia sprawczości jako kryterium ich oceny. Rozstrzygnięcie tego problemu powinno być przedmiotem przyszłych badań. \\

Podsumowując, przeprowadzone badania wskazały na użyteczność modelu sprawcy i biorcy również przy interpretowaniu zachowań grupowych. W szczególności pokazały, że grupa własna może być postrzegana z perspektywy biorcy -- wtedy treści wspólnotowe decydują o jej ocenie -- jak i z perspektywy sprawcy -- wtedy o jej ocenie decydują treści sprawcze. To rozróżnienie może nieść szereg konsekwencji dla badań nad uprzedzeniami, ale również nad kolektywnym działaniem. Przykładowo, warto zauważyć że w modelu kolektywnego działania opartym na tożsamości społecznej \parencite{van2008toward} postrzegana skuteczność grupy własnej jest jednym z głównych czynników wystąpienia kolektywnego działania. Dalsze badania nad wzbudzaniem perspektywy (kolektywnego) sprawcy mogłyby dostarczyć nowych metod zachęcania ludzi do kolektywnego działania i wprowadzania zmiany społecznej.\\

\subsubsection{Znaczenie metodologiczne uzyskanych wyników}

Przedstawione wyniki mają również znaczenie metodologiczne dla przyszłych badań nad wyznacznikami samooceny.\\

Po pierwsze, jak wykazano, traktowanie samooceny indywidualnej i kolektywnej nierozłącznie może prowadzić do trudności w otrzymaniu konkluzywnych wyników. Nie idzie tylko o wybór odpowiednich skal, ale również o to czy w ramach badania wyznaczników kolektywnej samooceny pojawia się kontekst grupowy i aktywizowane są treści związane z kolektywnym Ja czy nie. \\

Po drugie, nawet w przypadku badań dotyczących tylko i wyłącznie kolektywnej samooceny warto zwrócić uwagę na to w jaki sposób przedstawiana jest w nich grupa własna. Czy uczestników prosi się o ocenę pewnej kategorii społecznej, np. `Polacy', czy raczej aktywizuje się ich poczucie przynależności do pewnej kategorii przez odpowiednią prymę, np. `my Polacy' lub `ja jako Polak'. Co istotne, jak wykazały przedstawione badania, nawet charakter prymy może mieć znaczenie dla tego, który aspekt Ja ulegnie aktywizacji. \\

Po trzecie, zgodnie z wcześniejszymi badaniami \parencite[np.,][]{leach2007group} warto pytać o aspekty związane nie z dwoma wymiarami -- sprawczością i wspólnotowością -- ale przynajmniej z trzema -- sprawczością, moralnością i ciepłem interpersonalnym. Mimo, że wielokrotnie wskazywano na silne związki wymiarów moralności i ciepła, i przez to traktowano je łącznie, np. jako wspólnotowość, uzyskane dane wskazały, że oba wymiary mogą mieć inne znaczenie i prowadzić do odmiennych wniosków.

\subsubsection{Znaczenie praktyczne uzyskanych wyników}

Wyniki przedstawione w niniejszej rozprawie mogą mieć również znaczenie praktyczne. Należy jednak pamiętać, że przedstawione badania mają charakter podstawowy. W związku z tym, kwestia praktycznej możliwości ich wykorzystania wymaga jeszcze dodatkowej weryfikacji.\\

Szczególna rola moralności dla kolektywnej samooceny może być wykorzystywana przez firmy w celu budowania satysfakcji i kolektywnej samooceny pracowników organizacji. Istotnie, istnieją dowody na to, że stosowanie strategii tzw. społecznej odpowiedzialności biznesu podnosi ocenę moralności firmy, przez co pozytywnie wpływa na lojalność pracowników wobec pracodawcy \parencite{bauman2012corporate, ellemers2011corporate}. Analizy przedstawione w niniejszej rozprawie wskazują jednak na pułapki zbyt pochopnego rozumienia tej zależności. Wprowadzanie etycznych standardów jako składowych misji firmy niekoniecznie będzie miało związek z poziomem indywidualnej samooceny pracowników. Błędem będzie również jeżeli pracodawca poleci swoim pracownikom angażować się w działania o charakterze moralnym (np. będzie zachęcał do uczestniczenia w zbiórkach pieniędzy na cele charytatywne) i jednocześnie będzie oczekiwał, że przez to indywidualna samoocena pracowników istotnie wzrośnie.\\

Pewnym polem zastosowań wyników niniejszej rozprawy jest również obszar polityki, szczególnie jeżeli chodzi o kwestię  lojalności elektoratów wobec partii politycznych. W świetle przedstawionych danych można oczekiwać, że większość wyborców będzie oczekiwała od swoich partii przestrzegania zasad i norm moralnych, a zachowania niemoralne swojej partii mogą przyczyniać się do spadków poparcia. Warto jednak pamiętać, że według przedstawionych badań, w szczególnych sytuacjach, np. w sytuacji zagrożenia dla dobra wszystkich obywateli, oczekiwania elektoratów mogą ulec zmianie. W ekstremalnych sytuacjach poparcie dla partii może zależeć od postrzegania jej jako kompetentnej i sprawczej.\\

\subsection{Konkluzje}

Określenie z czego wynika wysokie poczucie własnej wartości i pozytywna samoocena jest niezwykle istotne dla wyjaśniania zachowania człowieka, zarówno jako indywidualnej jednostki jak i jako przedstawiciela pewnej grupy społecznej. Pozwala bowiem zrozumieć, jakie czynniki mogą popychać, lub powstrzymywać, osoby lub grupy do określonych, mniej lub bardziej chlubnych. Co więcej, pozwala zrozumieć dlaczego pewne zachowania jednostek lub grup występują, pomimo że są one oceniane przez Innych jako złe lub naganne.\\

W niniejszej rozprawie przedstawiono czym jest samoocena i jakie funkcje regulacyjne ona pełni. Wskazano, że ma ona znaczenie dla regulacji zachowania człowieka w sytuacjach zadaniowych, w sytuacjach interakcji społecznych, jak i w kontekście grupowym i kulturowym. Następnie, opisano że na obraz Ja i samoocenę człowieka składają się nie tylko aspekty indywidualne ale również te kolektywne. Zarówno w przypadku indywidualnym jak i kolektywnym samoocena jest wyznaczana przez mnogość różnych cech, które najogólniej można określić jako dotyczące postrzeganej sprawczości oraz wspólnotowości.\\

Zarówno sprawczość jak i wspólnotowość były często wskazywane w literaturze jako podstawowe aspekty wyznaczające pozytywną samoocenę. Za szczególnym znaczeniem każdego z tych wymiarów przemawiały zarówno kwestie teoretyczne, jak i zebrany materiał empiryczny. Ponieważ poszczególni autorzy postulowali zazwyczaj, że samoocena jest wyznaczana tylko przez jeden z tych wymiarów, kwestia wyznaczników samooceny była (i nadal jest) źródłem kontrowersji.\\

W ramach przedstawionego w niniejszej rozprawie modelu, postulowano że zarówno sprawczość jak i wspólnotowość mogą stanowić podstawę pozytywnej samooceny. O ile jednak sprawczość decyduje o samoocenie indywidualnej, to wspólnotowość decyduje o samoocenie kolektywnej. Tą różnicę można zrozumieć, jako różnicę perspektyw z jakich oceniane są indywidualne i zbiorowe aspekty Ja: O ile te pierwsze oceniane są z perspektywy sprawcy, te drugie oceniane są z perspektywy biorcy. Przeprowadzone badania dostarczają wstępnego potwierdzenie dla proponowanego modelu, ale również wskazują na konieczność dalszej jego weryfikacji.\\


\printbibliography
\end{document}
