\documentclass[man]{apa6}

\usepackage[utf8]{inputenc}
\usepackage[T1]{fontenc}
\usepackage[MeX]{polski}
\usepackage{todonotes}

\makeatletter
\renewcommand\efloat@iwrite[1]{%
   \immediate\expandafter\protected@write\csname efloat@post#1\endcsname{}}
\makeatother

\newcommand{\rowgroup}[1]{\hspace{-1em}#1}
\newcommand{\tdinline}[1]{\todo[inline]{#1}}

\usepackage{csquotes}
\usepackage[style=apa,sortcites=true,sorting=nyt,backend=biber]{biblatex}
\DeclareLanguageMapping{english}{american-apa}
\addbibresource{biblio.bib}

\usepackage{geometry}
\geometry{letterpaper}
\usepackage{graphicx}

\usepackage{threeparttable}
\usepackage{amsmath}

\title{Jakie aspekty indywidualnego i kolektywnego `Ja' wyznaczają
       pozytywną samoocenę. Znaczenie perspektywy sprawcy i biorcy.}
\shorttitle{Wyznaczniki pozytywnej samooceny}
\author{Wiktor Soral}
\affiliation{Uniwersytet Warszawski}

\abstract{Jakie aspekty indywidualnego i kolektywnego `Ja' wyznaczają pozytywną samoocenę. Znaczenie perspektywy sprawcy i biorcy.}

\keywords{indywidualna samoocena, kolektywna samoocena, model podwojnej perspektywy, sprawczosc, wspólnotowosc}


\begin{document}
\maketitle

Przekonania na temat siebie stanowią nieodłączny element zachowania człowieka \parencite[np.,][]{bandura1991social, deci2000and, rosenberg1965society, tesser1988toward}. Z jednej strony, przekonania te konstytuują osiągnięcia szkolne i zawodowe, zdrowie psychiczne oraz fizyczne, dobre relacjami z bliższymi jak i dalszymi innymi, czy choćby ogólne poczucie szczęścia i zadowolenia z życia. Z drugiej strony, rozmaite doświadczenia życiowe takie jak zdany bądź niezdany egzamin, akceptacja bądź odrzucenie przez grupę, zachowanie się w zgodzie lub niezgodzie z wewnętrznymi standardami kształtują przekonania wobec siebie (schematy Ja) składające się na ogólny obraz Ja \parencite{markus1977self}. \\

To na ile ukształtowany obraz Ja zostanie zaakceptowany lub odrzucony przekłada się na poziom globalnej samooceny człowieka \parencite{rosenberg1965society}. Osoby lubiące poszczególne elementy obrazu siebie i uważające poszczególne aspekty Ja za wartościowe charakteryzowały się będą wysokim poziomem samooceny. Natomiast osoby, które nie wartościują pozytywnie różnych aspektów Ja, bądź których postawy wobec poszczególnych aspektów Ja są ambiwalentne lub niepewne charakteryzować się będą niskim poziomem samooceny \parencite{baumeister1989self}.\\

Tak opisana samoocena jest rodzajem sądu społecznego tyle, że dotyczącego własnej osoby. Podobnie jak wszystkie sądy opiera się ona tych elementach obrazu Ja, które są dostępne w pamięci osoby (\emph{available}) i które uległy aktywizacji (\emph{accessible}) w momencie dokonywania samooceny \parencite[zob. np., ][]{higgins1996knowledge}. ...\\

Celem niniejszej rozprawy jest określenie, które treści stanowią centralne aspekty obrazu Ja i są głównymi wyznacznikami pozytywnej samooceny. Choć jest to pytanie zasadnicze, na które niejednokrotnie i na różne sposoby próbowano odpowiedzieć \parencite[np.,][]{brambilla2014importance, gebauer2013agency, wojciszke2011self}, dyskusja nad podstawami samooceny wydaje się być nadal nierozstrzygnięta.\\

Brak rozstrzygnięcia wynika pośrednio z faktu niejasności co do funkcji samooceny i ogólniej co do funkcji Ja. Istnieje co najmniej kilka różnych teorii mówiących o tym, do czego służy człowiekowi samoocena \parencite[np., ][]{leary1995self, pyszczynski2004people}, ale jak na razie brak jest jednej teorii -- unifikującej. Przeglądowi teorii opisujących funkcjonalne aspekty samooceny poświęcony zostanie pierwszy rozdział tej rozprawy.\\

Na pytanie o wyznaczniki samooceny nie można też odpowiedzieć bez głębszego zrozumienia, czym jest tożsamość człowieka i że na Ja składa się nie tylko poczucie odrębności od innych, ale również poczucie wspólnoty z innymi (Markus?). W związku z tym na obraz Ja składają się nie tylko aspekty właściwe jednostce, ale również takie które są właściwe jednostce jako członkowi grupy społecznej lub grupie jako całości. Dotychczasowe prace badawcze nad wyznacznikami samooceny w niewielkim stopniu próbowały radzić sobie z tą wielopoziomowością tożsamości i obrazu Ja. To zagadnienie zostanie przedstawione w drugim rozdziale rozprawy, ale będzie się pojawiało również w innych jej częściach. W szczególności rozróżnienie na indywidualne i kolektywne aspekty Ja posłuży do częściowego wyjaśnienia, dlaczego w dotychczasowych pracach trudno było o konsensus, co do podstawowych wyznaczników pozytywnej samooceny.\\

Mówiąc o podstawowych wyznacznikach pozytywnej samooceny łatwo popaść w pułapkę wielowymiarowości. Istnieje ogrom rozmaitych cech, które mogą być używane zarówno do oceny siebie jak i innych. Jednak wśród badaczy postrzegania społecznego istnieje dość silny konsensus mówiący, że ten ogrom cech jest w znacznej większości tłumaczony przez dwa podstawowe wymiary \parencite{judd2005fundamental, fiske2007universal}. Pierwszy z tych wymiarów odnosi się do funkcjonowania w sytuacjach zadaniowych (\emph{sprawczość}), natomiast drugi do funkcjonowania w sytuacjach społecznych \emph{wspólnotowość}. Te dwa wymiary postrzegania zostaną omówione w trzecim rozdziale rozprawy.\\

Czwarty rozdział rozprawy stanowi próba połączenia koncepcji dotyczących samooceny, tożsamości oraz treści postrzegania społecznego, której efektem jest sformułowanie przewidywań odnośnie tego, które aspekty indywidualnego i kolektywnego Ja wyznaczają pozytywną samoocenę. Jako podstawę teoretyczną dla tych przewidywań przyjęto model dwóch perspektyw - aktora vs. biorcy \parencite{abele2014communal} który pozwala określić kiedy w sądach społecznych większe znaczenie mają treści sprawcze, a kiedy treści wspólnotowe. Konkretnie w tym rozdziale podjęta zostanie próba nakreślenia relacji pomiędzy tożsamością indywidualną vs. kolektywną a perspektywą aktora vs. biorcy.\\

Przewidywania teoretyczne są weryfikowane i omawiane w ostatnich rozdziałach, gdzie zaprezentowano wyniki 4 badań - dwóch korelacyjnych i dwóch eksperymentalnych. Rozprawę zamyka omówienie i dyskusja wyników. W tej ostatniej części przedstawiono znaczenie teoretyczne, metodologiczne i praktyczne otrzymanych rezultatów. \\

\newpage 
\section{Część teoretyczna}
\subsection{Funkcje samooceny}

`Dlaczego ludzie potrzebują samooceny?' -- takie pytanie postawione przez \textcite{pyszczynski2004people} dobrze odzwierciedla pogląd większości współczesnych badaczy \parencite[zob. również, ][]{bandura1994self, leary1995self}, że samoocena człowieka pełni określone funkcje. Nie jest ona tylko dodatkiem do ludzkiego zachowania, ale elementem niezbędnym dla prawidłowego działania. Nie chodzi tylko o to, żeby mieć pozytywną samoocenę, ale raczej bardziej ogólnie o to, aby dysponować możliwością dokonywania sądów na temat własnej osoby oraz wglądu w obraz Ja.\\
Wiedzę dotyczącą Ja można najogólniej rozumieć, jako rodzaj systemu monitorującego zachowanie człowieka  i informującego, kiedy kontynuować powziętą czynność, a kiedy tej czynności zaprzestać \parencite{higgins1996self}. W tym rozdziale zostaną przedstawione trzy obszary, w których samoocena działa jako monitor ludzkiego zachowania. Pierwszy, z tych obszarów obejmuje sytuacje zadaniowe, związane z implementacją intencji i sprawowaniem kontroli. Drugi obszar odnosi się do funkcjonowania w sytuacjach społecznych i grupowych, w których celem nadrzędnym jest uzyskanie aprobaty grupy i uniknięcie wykluczenia społecznego. Trzeci obszar opisujący zachowanie w szerszym kontekście społecznym i kulturowym, odwołuje się do potrzeby afirmacji siebie i wykraczania poza postrzeganie siebie tylko jako jednostkę.\\

\subsubsection{Samoocena w sytuacjach zadaniowych}

Samoocena jako inicjator działań \parencite{bandura1994self}

\subsubsection{Samoocena w sytuacjach społecznych}

Samoocena jako socjometr \parencite{leary1995self, leary2000nature}

\subsubsection{Samoocena w kontekście kulturowym}

Samoocena jako bufor lęku przed śmiercą \parencite{pyszczynski2004people}
\newpage
\subsection{Tożsamość człowieka, samoocena indywidualna i kolektywna}

Teoria autokategoryzacji \parencite{turner1994self}.
Tożsamość społeczna odnosi się do podzielanej 

Poziomy ja \parencite{brewer2007importance}
Kolektywna samoocena \parencite{crocker1990collective}
\newpage
\subsection{Podstawowe wymiary postrzegania społecznego}
Wymiary postrzegania świata \parencite {abele2007agency}

Model sprawcy i biorcy \parencite{wojciszke2011self}
\newpage
\subsection{Model dwóch perspektyw a wyznaczniki samooceny indywidualnej i kolektywnej}

Rola moralności \parencite{ellemers2008better, ellemers2012morality, leach2007group, pagliaro2011sharing}
Moral licensing \parencite{merritt2010moral, monin2001moral}
\newpage
\section{Część empiryczna}

\begin{APAitemize}
\item H1.  W zależności od poziomu tożsamości (indywidualnej vs. kolektywnej) zmieniały się będą źródła pozytywnej samooceny.
\item H1a. O samoocenie na poziomie indywidualnym będą decydowały aspekty związane z postrzeganą sprawczością.
\item H1b. O samoocenie na poziomie kolektywnym będą decydowały aspekty związane z postrzeganą wspólnotowością.
\item H2.  Zmianie z perspektywy biorcy na perspektywę sprawcy towarzyszyć będzie wzrost znaczenia sprawczości jako wyznacznika samooceny.
\end{APAitemize}

\subsection{Strategia badawcza}

W niniejszej pracy zaprezentowano cztery badania (łącznie przeprowadzone na, N = 728 osobach), których celem było opisanie relacji pomiędzy wymiarami sprawczości i wspólnotowości a samooceną na poziomie indywidualnym i kolektywnym. Dwa z przedstawionych badań miały charakter korelacyjny, a w przypadku dwóch pozostałych zastosowano schemat eksperymentalny. W trzech pierwszych badaniach starano się sprawdzić, które wymiary są związane lub wpływają na poziom indywidualnej i kolektywnej samooceny, a w ostatnim sprawdzono czy poprzez wzbudzenie perspektywy kolektywnego sprawcy zmianie ulegnie kształt zależności pomiędzy wymiarami sprawczości i wspólnotowości, a kolektywną samooceną. \\
Aspekty związane ze sprawczością zoperacjonalizowano jako postrzegany poziom własnej kompetencji (np. \emph{skuteczności, inteligencji}), a pominięto te elementy definicji sprawczości związane z percepcją kontroli lub cechami związanymi z temperamentem. Aspekty związane ze wspólntowością rozdzielono i zoperacjonalizowano jako postrzegany poziom moralności (np. \emph{uczciwości, wiarygodności}) oraz jako postrzegany poziom ciepła w relacjach interpersonalnych (np. postrzegania własnej towarzyskości). Podobną operacjonalizację wymiarów można znaleźć np. w pracy \textcite{leach2007group}.\\
W ramach porównywania tożsamości indywidualnej i kolektywnej, poziomy Ja rozdzielono na te związane z obrazem siebie jako jednostki, siebie jako członka grupy narodowej lub grupy narodowej jako całości. Należy zaznaczyć, że ostatnie dwie kategorie stanowią tożsamość kolektywną, ale rozumianą na dwa różne, przeplatające się w literaturze sposoby. W niniejszej pracy skupiono się na kolektywnej tożsamości narodowej, a pominięto inne ważne rodzaje tożsamości zbiorowych.\\
Samoocenę w niniejszym projekcie zoperacjonalizowano jako poziom poziom ogólnych, jawnych przekonań o własnej wartościowości. W ramach niniejszej rozprawy nie poruszano wątków związanych z samooceną utajoną, ani np. z samooceną o charakterze narcystycznym.

\subsection{Strategia analityczna}
Analizy przedstawione w niniejszej rozprawie -- w znaczącej mierze bazujące na ogólnym modelu liniowym -- przeprowadzono w podejściu bayesowkim \parencite[zob. np.,][]{gelman2014bayesian,gill2014bayesian, kruschke2014doing}. Główną zaletą podejścia bayesowskiego jest możliwość uwzględnienia w analizach przeszłej wiedzy (tzw. prior) i przez to analitycznego powiązania ze sobą wyników linii badań (w odróżnieniu od klasycznego podejścia, w którym analizy są prowadzone tak jak gdyby badania były prowadzone w próżni, w oderwaniu pozostałych). Dzięki uwzględnieniu przeszłej wiedza wzrasta pewność co do analizowanych efektów i maleje szansa uzyskania czysto przypadkowych zależności. Ponadto, modele w podejściu bayesowkim cechuje znaczna elastyczność (tj. możliwość kwantyfikowania niepewności pochodzącej z różnych źródeł), przy podobnych lub często mniejszych wymogach co do liczebności badanej próby. Choć metody bayesowkie są niezwykle powszechne w statystyce, w pracach psychologicznych, zdominowanych przez podejście frekwentystyczne, są nadal rzadkością. Z tego względu w tej sekcji przedstawiono krótkie wprowadzenie mające ułatwić zrozumienie terminologii zawartej w części analitycznej rozprawy.\\

W ramach podejścia bayesowskiego zarówno dane, jak i parametry traktowane są jako zmienne losowe i jako takie posiadają ustalone rozkłady\footnote{W przypadku podejścia frekwentystycznego losowe są jedynie dane, natomiast parametry są traktowane jako posiadające jedną stałą wartość.}. Podstawowe równanie określa zależność pomiędzy rozkładami danych i parametrów:

\begin{equation}\label{eq:bayes}
    P(\theta | \mathcal{D}) =
    \frac{P(\mathcal{D} | \theta)\times P(\theta)}
    {P(\mathcal{D})}
\end{equation}

gdzie $P(\theta | \mathcal{D})$ oznacza rozkład prawdopodobieństwa parametrów przy określonych wartościach danych (tzw. rozkład posterior -- po dokonaniu obserwacji), $P(\mathcal{D} | \theta)$ oznacza rozkład prawdopodobieństwa danych przy określonych wartościach parametrów (inaczej funkcja wiarygodności związana z obserwacjami), $P(\theta)$ oznacza bezwarunkowy rozkład parametrów (tzw. rozkład prior -- przed dokonaniem obserwacji), a $P(\mathcal{D})$ oznacza bezwarunkowy rozkład danych (traktowany jako stała i czasami pomijany przy zachowaniu proporcjonalności obu stron równania). Mniej technicznym językiem: przekonania na temat występujących w populacji zależności (posterior) można ująć jako iloczyn wcześniejszej wiedzy (prior) i dokonanych obserwacji (funkcja wiarygodności). Lub jeszcze inaczej: równanie określa w jaki sposób obserwacje zmieniają przekonania badacza dotyczące interesującego go
wycinka rzeczywistości. \\
Celem analizy w podejściu bayesowkim jest uzyskanie rozkładu posterior (oraz takich jego charakterystyk jak wartość oczekiwana, wariancja, itd.) interesujących badacza parametrów. Zgodnie z równaniem \ref{eq:bayes} konieczne są tego informacje na temat trzech elementów. Funkcja wiarygodności jest obliczana podobnie jak w podejściu frekwentystycznym (przykładowo przy estymowaniu parametrów w równaniu regresji), tj. poprzez dobór wartości parametrów maksymalizujących wartość funkcji wiarygodności przy określonych danych. Rozkład prior jest ustalany na podstawie substantywnej wiedzy (pochodzącej z wcześniejszych badań, metaanaliz, na podstawie wiedzy eksperckiej, itd.). Rozkład prior dla niektórych parametrów może być również ustalony na podstawie tego co powszechnie o nich wiadomo (np. wartość odchylenia standardowego nigdy nie jest mniejsza od 0, a wartość absolutna współczynnika korelacji nigdy nie jest większa od 1). Gdy ustalenie rozkładu prior jest trudne lub niemożliwe, powszechne jest stosowanie tzw. nieinformatywnych lub referencyjnych rozkładów, tzn. takich które nie faworyzują żadnej określonej wartości (takim rozkładem może być np. rozkład jednostajny, w przypadku parametrów oznaczających dyspersję powszechne jest stosowanie rozkładu Gamma, uciętego rozkładu Normalnego, lub uciętego rozkładu Cauchy'ego).\\
Analityczne ustalenie ostatniego elementu równania -- rozkładu brzegowego danych -- jest z pozoru najbardziej kłopotliwe. W przypadku rozkładów ciągłych (np. rozkładu Normalnego) taka operacja wymaga policzenia całki:
\begin{equation}
    \int P(\mathcal{D} | \theta)\times P(\theta)\ \mathrm{d}\theta
\end{equation}
co zazwyczaj jest operacją trudną nawet dla komputera, a dość często jest to niemożliwe. \\
Obecnie ten problem jest rozwiązywany przez zastosowanie symulacji Monte Carlo, a w przypadku modeli wielozmiennowych przez jej wariantu -- tzw. Markov chain Monte Carlo \parencite[MCMC,][]{hastings1970monte}. Ten wariant symulacji polega na uruchomieniu ciągu (\emph{chain}) losowych, w niewielkim stopniu skorelowanych wartości, który ``przeszukuje'' przestrzeń parametrów modelu. Po pewnym czasie ciąg osiąga stan tzw. konwergencji a jego wartości można traktować jako wylosowane niezależnie z rozkładu posterior i po odrzuceniu wartości początkowych (przed konwergencją) posłużyć się pozostałymi wartościami do podsumowania rozkładu. Aby ustalić, czy ciąg osiągnął konwergencję, wykonuje się zazwyczaj jeden (lub kilka) z opracowanych testów diagnostycznych; w niniejszej rozprawie do tego celu wykorzystano m.in. wskaźnik potencjalnej redukcji skali \parencite[\emph{potential scale reduction factor},][]{gelman1992inference}. Wartości tego współczynnika poniżej 1,1 interpretowane są jako brak powodów do odrzucenia ciągu ze względu na brak konwergencji.\\
Rozład posterior podsumowuje się zazwyczaj podając jego wartość centralną (np. średnią) i miarę dyspersji (np. odchylenie standardowe). Z perspektywy badacza najbardziej istotne jest poznanie przedziału wiarygodności (\emph{credible interval}). Przedział wiarygodności określa fragment przestrzeni parametrów, w którym skupiony jest określony procent rozkładu (np. 95\%); czyli innymi słowy gdzie określonym z prawdopodobieństwem można spodziewać się wartości parametrów\footnote{Choć ta definicja może nasuwać skojarzenia z przedziałem ufności, nie należy mylić ze sobą tych dwóch koncepcji -- przedział ufności ma zdecydowanie inna mniej intuicyjną interpretację \parencite[zob. np.,][]{gill2014bayesian}.}. Przy wykorzystaniu rozkładu posterior można również testować hipotezy i określić prawdopodobieństwo z jakim parametr (np. współczynnik regresji) lub różnica parametrów (np. różnica średnich) jest większa lub mniejsza od określonej wartości (np. 0). Należy podkreślić, że takie prawdopodobieństwo należy traktować jako prawdopodobieństwo warunkowe (tj. prawdopodobieństwo przyjęcia przez parametry określonych wartości przy uwzględnieniu obserwacji badacza). Ponadto, pomimo, że w niniejszej rozprawie prawdopodobieństwo posterior jest raportowane w podobnej formie jak wartość $p$, należy zaznaczyć, że są to zupełnie różne koncepcje i nie należy mylić ich ze sobą.


\newpage
\subsection{Badanie 1}

Badanie 1 miało charakter eksploracyjny. Jego celem było sprawdzenie w jakim stopniu poziom uogólnionej (indywidualnej i kolektywnej) samooceny jest wyznaczany przez przypisywane sobie cechy związane z wymiarem sprawczości (tj. kompetencji) i wspólnotowości (tj. moralności i ciepła).\\
Dotychczasowe prace postulowały jedną z dwóch możliwości: wariancja samooceny indywidualnej i kolektywnej będzie w znaczącej większości wyjaśniana przez cechy sprawcze \parencite{wojciszke2011self} lub wariancja samooceny indywidualnej i kolektywnej będzie w znaczącej większości tłumaczona przez cechy wspólnotowe \parencite[przede wszystkim moralność, np.][]{leach2007group}. Trzecia - najmniej zbadana - z możliwości mówi, że układ predyktorów będzie różny dla samooceny indywidualnej i kolektywnej.\\
Dodatkowym celem było sprawdzenie, czy w ramach każdego z warunków układ predyktorów samooceny będzie analogiczny dla skal złożonych z cech pozytywnych jak i negatywnych, tj. czy predykcyjna rola niekompetencji, niemoralności oraz zimna będzie symetryczna do roli kompetencji, moralności i ciepła.

\subsubsection{Metoda}

\paragraph{Uczestnicy i schemat badania}
W badaniu wzięło udział $N=98$ studentów (74 kobiety, 21 mężczyzn, 3 osoby nie zadeklarowały swojej płci) w wieku od 19 do 27 lat ($M = $ 21,26; $SD =$ 1,81). Uczestnicy wypełniali jedną z trzech losowo przydzielonych wersji kwestionariusza: a) dotyczącą indywidualnych cech i indywidualnej samooceny ($n = 34$), b) dotyczącą cech i samooceny siebie jako Polaka ($n = 31$), lub c) dotyczącą kolektywnych cech i kolektywnej samooceny Polaków ($n = 33$). Procedurę losowania do warunków przeprowadzono z zachowaniem proporcji płci. Badanie przeprowadzano w salach wykładowych w grupach od 15 do 30 osób. Po zebraniu kwestionariuszy eksperymentator wyjaśniał grupie cel badania i odpowiadał na pytania.

\paragraph{Miary}
W każdym z trzech warunków uczestnicy wypełniali identyczne miary, a zmianom ulegała jedynie forma stwierdzeń. W pierwszym warunku stwierdzenia rozpoczynały się od zaimka \emph{ja}, w warunku drugim stwierdzenia rozpoczynały się od frazy \emph{ja jako Polak}, natomiast w trzecim warunku od zaimka \emph{my} lub frazy \emph{my Polacy}. Podsumowanie statystyk dla każdej ze skali w trzech warunkach przedstawiono w Tabeli \ref{tab:0}.

\subparagraph{Skala kompetencji, moralności i ciepła}
W pierwszej kolejności uczestnicy dokonywali autoaskrypcji 18 cech -- pozytywnych i negatywnych -- związanych z wymiarami kompetencji (np. kompetentny, słaby), moralności (np. moralny, dwulicowy) i ciepła (np. ciepły, ponury); tj. po 3 dla każdej z 6 subskal. Zadaniem każdej osoby była ocena na skali od 1 -- \emph{Zdecydowanie nie} do 7 -- \emph{Zdecydowanie tak} na ile dana cecha do niego/nich pasuje. Miarę zaczerpnięto z poprzednich badań (Soral, Kofta, niepublikowana praca magisterska) -- jej trafność czynnikową i rzetelność zweryfikowano w dodatkowym badaniu pilotażowym. Mimo to dla niektórych ze skal zaobserwowano nieakceptowalne współczynniki rzetelności, $\alpha< 0,60$  (patrz Tabela \ref{tab:0}). Aby poradzić sobie z tą słabością pomiaru, w części analitycznej postanowiono zastosować korektę o błędy pomiarowe dla każdej ze skal.

\subparagraph{Skala samooceny}
W dalszej części kwestionariusza każdy uczestnik wypełniał skalę samooceny opartą na mierze \textcite{rosenberg1965society}. Skala została zmodyfikowana, aby usunąć z niej stwierdzenia pytające wprost o komponenty związane z wymiarem sprawczości (czyli aby uniknąć redundancji) -- wyboru stwierdzeń dokonano na podstawie wcześniejszego badania pilotażowego. Dodatkowo aby uzyskać lepsze właściwości psychometryczne, do skali dodano stwierdzenia ze skali lubienia siebie Tafarodi i Milne (????). W sumie na skalę składało się 13 stwierdzeń (w tym 6 odwrotnie kodowanych), np. \emph{Uważam, że jestem osobą wartościową przynajmniej w takim samym stopniu co inni, Uważam, że jako Polak posiadam wiele pozytywnych cech, Czasami czuję, że my Polacy jesteśmy bezwartościowi}.\\
Uczestników poproszono aby ocenili na ile zgadzają się z każdym stwierdzeniem przy użyciu 4-stopniowej skali od 1 \emph{Zdecydowanie nie zgadzam się} do 4 \emph{Zdecydowanie się zgadzam}. Rzetelność skali okazała się satysfakcjonująca, $\alpha > 0,84$ . \\

\begin{table*}[htbp]
\vspace*{2em}
\centering
\begin{threeparttable}
\caption{Podstawowe statystyki opisowe dla skal użytych w Badaniu 1}
\label{tab:0}
\begin{tabular}{lrrrrrrr}
\midrule
\multicolumn{2}{c}{ } & \multicolumn{2}{c}{Ja} & \multicolumn{2}{c}{Ja Polak} & \multicolumn{2}{c}{My Polacy} \\
\cline{3-4}\cline{5-8}
            & $\alpha$ & $M$ & $SD$ & $M$ & $SD$ & $M$ & $SD$ \\
\midrule
\emph{Pozytywne} \\
Kompetencje & 0,67 & 4,89 & 1,03 & 4,89 & 0,89 & 4,98 & 0,93 \\
Moralność   & 0,77 & 5,31 & 0,92 & 4,88 & 1,27 & 3,84 & 0,89 \\
Ciepło      & 0,54 & 4,96 & 1,28 & 4,69 & 1,15 & 5,02 & 1,11 \\
\emph{Negatywne} \\
Kompetencje & 0,60 & 2,53 & 1,24 & 2,29 & 0,92 & 2,70 & 0,93 \\
Moralność   & 0,81 & 2,79 & 1,12 & 2,78 & 1,43 & 4,51 & 1,35 \\
Ciepło      & 0,57 & 2,90 & 1,43 & 3,02 & 1,20 & 3,03 & 1,29 \\
\\
Samoocena & 0,86 & 2,95 & 0,50 & 3,21 & 0,44 & 2,97 & 0,45 \\
\midrule
\end{tabular}

\begin{tablenotes}[para,flushleft]
{\small
\textit{Nota.} $N$ = 98
}
\end{tablenotes}
\end{threeparttable}

\end{table*}


\subsubsection{Wyniki}
\paragraph{Autoaskrypcja cech pozytywnych a poziom samooceny}
W pierwszej kolejności sprawdzono w jakim stopniu, w każdym z warunków eksperymentalnym, przypisywanie sobie cech pozytywnych jest związane z poziomem samooceny. W tym celu skonstruowano model regresji miary samooceny (S) na miary kompetencji (K), moralności (M) i ciepła (C):

\begin{equation}\label{eq:1}
\begin{split}
S_{i} & = \beta_{0j} + \beta_{1j}K_{ji} + \beta_{2j}M_{ji} + \beta_{3j}C_{ji} + \epsilon_{i} \\
K & \sim N(kompetencje, \sigma_{Kj}) \\
M & \sim N(moralnosc, \sigma_{Mj}) \\
C & \sim N(cieplo, \sigma_{Cj}) \\
S & \sim N(samoocena, \sigma_{Sj})
\end{split}
\end{equation}

gdzie indeks $j$ oznacza jeden z trzech warunków badania (ja vs. ja Polak vs. my Polacy), a indeks $i$ osobę. Aby uwzględnić zróżnicowanie poziomów rzetelności każdej z miar w każdym z warunków, model regresji oparto na ich wskaźnikach latentnych, o rozkładach normalnych ze średnią równą wartości obserwowanej miary i odchyleniu standardowym równym jej błędowi pomiarowemu wyliczonemu ze wzoru:

\begin{equation}\label{eq:err}
\sigma = SD*\sqrt{1 - r}
\end{equation}

gdzie $SD$ oznacza odchylenie standardowe miary, a $r$ rzetelność skali. W modelu zastosowano nieinformatywne rozkłady prior dla wszystkich parametrów. Wyniki oszacowania modelu zaprezentowano w górnej części Tabeli \ref{tab:1}\footnote{Aby uzyskać rozkłady posterior parametrów modelu wykorzystano algorytm MCMC (Markov Chain Monte Carlo, konkretnie \emph{Hamiltonian Monte Carlo}) zaimplementowany w programie Stan \parencite{carpenter2016} z 4000 iteracji. Ocena wskaźnika PSRF nie dostarczyła powodów do odrzucenia symulacji ze względu na brak konwergencji, $Rs < 1,01$.}. \\

Wśród uczestników poproszonych o myślenie o sobie w kategoriach `ja' samoocena była wyznaczana przez poziom przypisywanych sobie kompetencji (parametr dla skali kompetencji był pozytywny z prawdopodobieństwem, $p = 0,99$) i nie pozostawała w związku z przypisywaniem sobie cech ze skal moralności i ciepła (prawdopodobieństwo, że którykolwiek z parametrów dla tych skal był pozytywny wynosiło odpowiednio $p = 0,40$ i $p = 0,46$). \\

U uczestników poproszonych o myślenie o sobie w kategoriach `ja jako Polaka' samoocena była wyznaczana jedynie przez przypisywane sobie cechy ze skali ciepła (parametr był pozytywny z prawdopodobieństwem, $p = 0,99$), natomiast przypisywanie sobie cech ze skal kompetencji lub moralności nie miało związku z poziomem samooceny u tych uczestników (wartości parametrów były pozytywne z prawdopodobieństwem odpowiednio, $p = 0,21$ i $p = 0,63$). \\

W końcu, w przypadku uczestników których poproszono o myślenie o sobie w kategoriach `my Polacy' zbiorowa samoocena pozostawała w pozytywnym związku z przypisywaniem sobie cech ze skali moralności (z prawdopodobieństwem, $p = 0,98$). Przypisywanie sobie cech ze skali kompetencji i ciepła nie miało u tych uczestników związku ze zbiorową samooceną (wartości parametrów dla tych skal byłe pozytywne odpowiednio z, $p = 0,30$ i $p = 0,83$).\\

W całej badanej próbie zaprezentowany model wyjaśniał -- z pradowpodobieństwem, $p = 0,95$ -- od 2 do 30\% wariancji ogólnej samooceny. Wykresy rozrzutu wraz z liniami przedstawiającymi dopasowanie przedstawiono na Rycinie \ref{fig:1}.\\

\begin{table*}[htbp]
\vspace*{2em}
\centering
\begin{threeparttable}
\caption{Pozytywne i negatywne aspekty Ja: kompetencje, moralność i ciepło jako predyktory samooceny -- podsumowanie rozkładów brzegowych parametrów modeli.}
\label{tab:1}
\bgroup
\def\tabcolsep{4pt}
\begin{tabular}{lrrrrrrrrr}
\midrule
 &
\multicolumn{3}{c}{Ja} &
\multicolumn{3}{c}{Ja Polak} &
\multicolumn{3}{c}{My Polacy} \\
\cline{2-10}
 & $M_{post.}$    & $SD$   & $95\%\ CI$   & $M_{post.}$    & $SD$   & $95\%\ CI$   & $M_{post.}$    & $SD$   & $95\%\ CI$   \\
\midrule
 \multicolumn{10}{c}{\emph{Pozytywne}}  \\
 Stała       &  2,16 & 0,57 &  1,03;3,28 &  2,62 & 0,48 &  1,66;3,59 &  2,10 & 0,55 &  1,00;3,15 \\
 Kompetencje &  0,19 & 0,09 &  0,02;0,36 & -0,09 & 0,12 & -0,33;0,16 & -0,05 & 0,09 & -0,24;0,13 \\
 Moralność   & -0,02 & 0,09 & -0,21;0,17 &  0,02 & 0,09 & -0,16;0,20 &  0,20 & 0,09 &  0,03;0,39 \\
 Ciepło      & -0,00 & 0,07 & -0,14;0,12 &  0,19 & 0,08 &  0,05;0,34 &  0,07 & 0,07 & -0,07;0,21 \\
 \multicolumn{10}{c}{\emph{Negatywne}}  \\
 Stała       &  3,25 & 0,26 &  2,75;3,75 &  3,57 & 0,23 &  3,13;4,02 &  4,20  & 0,29 & 3,63;4,80 \\
 Kompetencje & -0,21 & 0,06 & -0,33;-0,10 &  0,07 & 0,11 & -0,15;0,29 & -0,09 & 0,07 & -0,24;0,05 \\
 Moralność   & -0,04 & 0,07 & -0,18;0,09 & -0,04 & 0,07 & -0,17;0,10 & -0,14 & 0,05 & -0,25;-0,04 \\
 Ciepło      &  0,12 & 0,05 &  0,01;0,22 & -0,14 & 0,07 & -0,28;-0,00& -0,11 & 0,05 & -0,22;-0,00 \\
\bottomrule
\end{tabular}
\egroup
\begin{tablenotes}[para,flushleft]
{\small
\textit{Nota.} $N = 98$. Dobroć dopasowania modelu dla cech pozytywnych: $R^2$ = 0,13, 95\% CI = [0,02;0,30]. Dobroć dopasowania modelu dla cech negatywnych: $R^2$ = 0,25, 95\% CI = [0,09;0,45]
}
\end{tablenotes}
\end{threeparttable}
\end{table*}


\begin{figure*}[htbp]
   \centering
   \fitfigure{study1.pdf}
   \caption{Autoaskrypcja pozytywnych cech związanych w wymiarami kompetencji, moralności i ciepła, a poziom samooceny indywidualnej oraz kolektywnej. Punkty oznaczają latentne wyniki dla każdej osoby, z kreskami oznaczającymi błędy pomiarowe skal. Grubą linią przerywaną oznaczono najlepsze dopasowanie uzyskane w modelu regresyjnym, z cieńszymi liniami oznaczającymi błąd oszacowania.}
   \label{fig:1}
\end{figure*}

\paragraph{Autoaskrypcja cech negatywnych a poziom samooceny}
W drugim kroku analogiczne analizy powtórzono dla cech negatywnych tworzących skale kompetencji, moralności i ciepła. W tym przypadku celem było sprawdzenie w jakim stopniu zaprzeczanie posiadania określonych negatywnych cech jest związane z pozytywną samooceną. Wyniki zaprezentowano w dolnej części Tabeli \ref{tab:1}\footnote{Ponownie model oparto na wskaźnikach latentnych konstruktów oraz zastosowano nieinformatywne rozkłady prior dla wszystkich konstruktów. Ocena wskaźnika PSRF dla 4000 iteracji MCMC nie dostarczyła powodów do odrzucenia symulacji ze względu na brak konwergencji, $Rs < 1,01$.}. \\

Wśród uczestników poproszonych o myślenie o sobie w kategoriach `ja' jedynie zaprzeczanie własnej niekompetencji wyznaczało pozytywną samoocenę (parametr był negatywny z prawdopodobieństwem, $p > 0,999$). Zaprzeczanie własnej niemoralności nie pozostawało w związku z samooceną (parametr był negatywny z prawdopodobieństwem, $p = 0,72$), natomiast zaprzeczanie własnemu zimnu pozostawało z samooceną w związku odwrotnym do oczekiwań (parametr był pozytywny z prawdopodobieństwem, $p = 0,98$), co oznacza, że w badanej próbie -- przy stałym poziomie pozostałych predyktorów -- osoby przypisujące sobie cechy związane z wymiarem zimna cechowały się wyższą indywidualną samooceną.\\

Wśród uczestników poproszonych o myślenie o sobie w kategoriach `ja jako Polaka', ci zaprzeczający zimnu charakteryzowali się wyższym poziomem samooceny (parametr był negatywny z prawdopodobieństwem, $p = 0,98$), natomiast zaprzeczanie własnej niekompetencji lub niemoralności nie było wśród tych uczestników związane z poziomem samooceny  (parametry dla tych skali były negatywne z prawdopodobieństwem, odpowiednio $p = 0,26$ i $p = 0,69$).\\

W warunku `my' im bardziej uczestnicy zaprzeczali niemoralności i zimnu Polaków tym ich zbiorowa samoocena była wyższa (parametry dla tych skal były negatywne z prawdopodobieństwem, odpowiednio $p = 0,99$ i $p = 0,98$). Zaprzeczanie niekompetencji Polaków nie pozostawało w związku z samooceną (parametr dla tej skali był negatywny z prawdopodobieństwem, $p = 0,89$).\\

Model dla cech negatywnych wyjaśniał -- z prawdopodobieństwem $p = 0,95$ -- od 9 do 45\% wariancji samooceny. Wykres rozrzutu wraz z liniami dopasowania przedstawiono na Rycinie \ref{fig:study1b}.


\begin{figure*}[htbp]
   \centering
   \fitfigure{study1b.pdf}
   \caption{Autoaskrypcja negatywnych cech związanych w wymiarami kompetencji, moralności i ciepła, a poziom samooceny indywidualnej oraz kolektywnej. Punkty oznaczają latentne wyniki dla każdej osoby, z kreskami oznaczającymi błędy pomiarowe skal. Grubą linią przerywaną oznaczono najlepsze dopasowanie uzyskane w modelu regresyjnym, z cieńszymi liniami oznaczającymi błąd oszacowania.}
   \label{fig:study1b}
\end{figure*}

\subsubsection{Dyskusja}

Badanie 1 przeprowadzono, aby sprawdzić które aspekty Ja -- związane z wymiarem sprawczości czy wspólnotowości -- są wyznacznikami uogólnionej (indywidualnej i kolektywnej) samooceny. Uzyskane wyniki nie potwierdziły hipotezy o szczególnym znaczeniu wymiaru sprawczości dla uogólnionej samooceny. W badaniu wykazano wprawdzie, że wymiar kompetencji ma szczególne znaczenie dla indywidualnej samooceny: osoby które w większym stopniu przypisywały sobie pozytywne cechy sprawcze cechowały się wyższym poziomem indywidualnej samooceny. Nie wykazano jednak, aby wymiar kompetencji miał szczególne znaczenie dla samooceny kolektywnej: przy kontroli pozostałych predyktorów cechy sprawcze nie pozostawały w związku z samooceną siebie jako Polaka lub grupy jako ogółu. \\
Analogicznie, uzyskane wyniki nie dostarczyły poparcia dla tezy o szczególnym znaczeniu wymiaru wspólnotowości dla uogólnionej samooceny. W zebranej próbie podwymiary wspólnotowości (moralność i ciepło) miały znaczenie dla samooceny kolektywnej, tj. osoby oceniające siebie jako członka grupy lub grupę jako ogół jako moralne i ciepłe cechowały się wyższym poziomem kolektywnej samooceny. Mimo to, ani moralność ani ciepło nie odgrywały roli jako wyznaczniki pozytywnej indywidualnej samooceny.\\
Uzyskane wyniki zdają się bardziej przemawiać za tezą, że wraz poziomami Ja (indywidualnym i kolektywnym) zmieniają się wyznaczniki samooceny. W przeprowadzonym badaniu samoocena indywidualna była wyznaczana przez aspekty związane z wymiarem kompetencji, samoocena siebie jako członka grupy przez cechy związane z wymiarem ciepła, natomiast samoocena grupy jako ogółu przez cechy moralne. \\
Warto zaznaczyć, że układ predyktorów w ramach pozytywnych i negatywnych aspektów Ja okazał się w przybliżeniu symetryczny. Tak jak aspekty świadczące o kompetencjach były głównym wyznacznikiem wysokiej sammoceny, tak aspekty świadczące o braku kompetencji okazały się głównym wyznacznikiem niskiej samooceny. Analogicznie postrzegane ciepło wyznaczało wysoką samoocenę siebie jako Polaka, natomiast postrzegane zimno niską samoocenę siebie jako członka narodu polskiego. W końcu postrzegana moralność wyznaczała wysoką samoocenę Polaków jako grupy, a nie niemoralność niską samoocenę grupową. \\
Podsumowując, uzyskane dane wskazały na konieczność dalszej eksploracji badanego zagadnienia i replikacji uzyskanego wzorca zależności na próbie niestudenckiej: być może szczególna rola wymiaru kompetencji dla indywidualnej samooceny wynikała ze specyfiki tej próby. W kolejnym badaniu warto było również sprawdzić predyktory zbiorowej samoooceny wśród grup innych niż Polacy: możliwe, że szczególna rola ciepła i moralności wynikała z centralności tych wymiarów dla autostereotypu Polaków. \\

\newpage
\subsection{Badanie 2}
Celem Badania 2 była replikacja wyników uzyskanych w pierwszym badaniu. Aby w większym stopniu zgeneralizować zależności, niniejsze badanie postanowiono przeprowadzić na niestudenckiej próbie nie-Polaków: tu Amerykanów. O ile w przypadku prowadzenia badań na polskiej próbie istniała możliwość, że efekty są uwarunkowane niskim statusem Polski na scenie międzynarodowej i wynikają np. z efektów kompensacyjnych \parencite[zob. np.][]{judd2005fundamental}, tak w przypadku badań na próbie amerykańskiej taka możliwość wydała się mniej prawdopodobna.\\
W niniejszym badaniu postanowiono sprawdzić związek autoaskrypcji kompetencji, moralności i ciepła z indywidualnym i kolektywnym poziomem samooceny. W badaniu skupiono się na cechach pozytywnych, ponieważ wyniki poprzedniego badania wskazały na symetryczność relacji pomiędzy wymiarami pozytywnymi i negatywnymi a samooceną. Samoocena była tu rozumiana ponownie jako ogólne przekonanie o własnej wartościowości (lub wartościowości grupy własnej) i poziom lubienia siebie. Jednak dodatkowo postanowiono skontrastować model dla tak rozumianej samooceny, z modelem dla samooceny jako przekonań o własnej sprawczości i możliwości radzenia sobie w różnych sytuacjach. Choć ten drugi model cechował się redundancją (tzn. wyjaśnianianie percepcji sprawczości przez autoaskrypcje cech związanych z wymiarem kompetencji) pozwalał on na lepsze zrozumienie charakteru wyznaczników samooceny zwłaszcza w przypadku kolektywnego Ja.\\

\subsubsection{Metoda}

\paragraph{Uczestnicy i schemat badania}
Badanie przeprowadzono na $N=103$ użytkownikach amerykańskiego serwisu mTurk (45 kobiet, 58 mężczyzn). Wiek uczestników wahał się od 18 do 66 lat ($M=$ 31,55, $SD=$ 12,58). Podobnie jak w Badaniu 1 uczestnicy zostali losowo przydzieleni do jednego z trzech warunków: a) dotyczącego cech indywidualnych, b) dotyczącego cech siebie jako przedstawiciela grupy (Amerykanina), c) dotyczącego wspólnych cech grupy (Amerykanów). Procedurę losowania przeprowadzono z zachowaniem proporcji płci. Po wypełnieniu kwestionariusza każdy uczestnik przechodził na stronę, na której wyjaśniano mu cel badania.

\paragraph{Miary}
Tak jak w poprzednim badaniu w każdym warunku uczestnicy wypełniali skale, w których stwierdzenia posiadały identyczne człony główne, a różniły się jedynie formą: \emph{ja, ja jako Amerykanin, my Amerykanie}. Podstawowe statystyki opisowe dla użytych skal przedstawiono w Tabeli \ref{tab:02}.

\subparagraph{Skala kompetencji, moralności i ciepła}
Po zapoznaniu się z instrukcjami wstępnymi uczestnicy dokonywali autoaskrypcji 9 pozytywnych cech związanych z wymiarami kompetencji (np. \emph{kompetentny -- competent}), moralności (np. \emph{uczciwy -- honest}) i ciepła (\emph{ciepły -- warm}). Były to stwierdzenia zaczerpnięte z badań \textcite{leach2007group}) i ich znaczenie było zbliżone do pozytywnych stwierdzeń zastosowanych w Badaniu 1. Zadaniem uczestnika była ocena na skali od 1 -- \emph{Zdecydowanie nie} do 7 -- \emph{Zdecydowanie tak}, na ile on, on jako Amerykanin lub Amerykanie jako grupa posiadają każdą z tych cech. \\
Statystyki zgodności wewnętrzenej dla wszystkich skal były satysfakcjonujące, $\alpha <$ 0,75, (zob. Tabela \ref{tab:02}). \\
\subparagraph{Skale lubienia siebie i sprawczości}
W następnej cześci uczestnicy wypełniali skalę samooceny składającą się z dwóch części: 10 pytań dotyczących lubienia siebie/nas (np. \emph{Czuję się komfortowo gdy myślę o sobie -- I am very comfortable with myself}) i 10 pytań dotyczących przekonań o własnej -- lub kolektywnej -- sprawczości (np. \emph{Jestem niezwykle efektywne w czynnościach które wykonuję -- I am highly effective at the things I do}). W ramach każdej części połowa stwierdzeń była kodowana odwrotnie. Zadaniem uczestnika była ocena na ile zgadza się z każdym ze stwierdzeń przy użyciu 4-stopniowej skali odpowiedzi od 1 -- \emph{Zdecydowanie się nie zgadzam} do 4 -- \emph{Zdecydowanie się zgadzam}.\\
Wykorzystana skala została opracowana przez \textcite{tafarodi1995self}. W kolejnych badanich wykazano 2-czynnikową strukturę skali \parencite{tafarodi2002decomposing}. W niniejszym badaniu obie podskale uzyskały satysfakcjonujące współczynniki zgodności wewnętrznej w ramach każdego z warunków (zob. Tabela \ref{tab:02}).

\begin{table*}[htbp]
\vspace*{2em}
\centering
\begin{threeparttable}
\caption{Podstawowe statystyki opisowe dla skal użytych w Badaniu 2}
\label{tab:02}

\begin{tabular}{lrrrrrrr}

\midrule
\multicolumn{2}{c}{ } & \multicolumn{2}{c}{Ja} & \multicolumn{2}{c}{Ja Amerykanin} & \multicolumn{2}{c}{My Amerykanie} \\
\cline{3-8}
& $\alpha$ & $M$ & $SD$ & $M$ & $SD$ & $M$ & $SD$ \\
\midrule
Kompetencje & 0,79 & 5,72 & 0,92 & 5,68 & 1,00 & 5,14 & 1,14 \\
Moralność   & 0,78 & 5,90 & 0,75 & 5,57 & 0,98 & 4,04 & 1,15 \\
Ciepło      & 0,75 & 4,97 & 1,24 & 5,21 & 1,28 & 5,28 & 0,74 \\
Lubienie siebie & 0,93 & 2,91 & 0,54 & 3,04 & 0,62 & 3,05 & 0,44 \\
Sprawczość Ja   & 0,88 & 3,08 & 0,48 & 3,21 & 0,45 & 3,09 & 0,48 \\
\bottomrule

\end{tabular}

\begin{tablenotes}[para,flushleft]
{\small
\textit{Nota.} $N$ = 103
}
\end{tablenotes}
\end{threeparttable}
\end{table*}

\subsubsection{Wyniki}

\paragraph{Autoaskrypcja kompetencji, moralności i ciepła, a poziom lubienia siebie}

W pierwszym kroku analiz sprawdzono w jakim stopniu przypisywanie sobie cech ze skal kompetencji, moralności i ciepła wyznacza poziom lubienia siebie jako jednostki vs. siebie jako członka grupy vs. grupy jako całości. Ponownie wykorzystano model uwzględniający niepewność pomiarową (por. Równanie \ref{eq:1}). Tym razem zastosowano informatywne rozkłady prior oparte na wynikach Badania 1 (por. górna część Tabeli \ref{tab:1})\footnote{Uwzględnienie tej dodatkowej informacji, pozwoliło uzyskać dokładniejsze oszacowania parametrów modelu \parencite[zob. np.][]{gill2014bayesian}. Rozkłady posterior parametrów modelu otrzymano na podstawie symulacji MCMC z 4000 iteracji. Wskaźniki potencjalnej redukcji skali (PSRF) nie dostarczyły powodów do odrzucenia symulacji ze względu na brak konwergencji, Rs < 1,01}. Podsumowanie wyników przedstawiono w górnej części Tabeli \ref{tab:2}. \\

Wśród uczestników w warunku `ja', poziom lubienia siebie był wyznaczany przez przypisywanie sobie cech ze skal kompetencji oraz ciepła (parametry dla tych skal były z pozytywne odpowiednio z, $p > $ 0,999 i $p = $ 0,99). Lubienie siebie nie było jednak związane z przypisywaniem sobie cech ze skali moralności (parametr był pozytywny z prawdopodobieństwem, $p = $ 0,63). \\

U uczestników zapytanych o postrzeganie siebie `jako Amerykanina', lubienie siebie było pozytywnie związane jedynie z przypisywaniem sobie cech ze skali ciepła (z prawdopodobieństwem, $p >$ 0,999). Nie zaobserwowano w tym warunku związku lubienia siebie z przypisywaniem sobie cech ze skali kompetencji i moralności (parametry były pozytywne odpowiednio z $p =$  0,73 i $p =$ 0,72).\\

W trzecim warunku, lubienie `nas Amerykanów' było uzależnione od przypisywania grupie własnej cech ze skali moralności (parametr był pozytywny z prawdopodobieństwem, $p >$ 0,999), ale nie od przypisywania cech ze skal kompetencji i ciepła (parametry były pozytywne z prawdopodobieństwem odpowiednio, $p =$ 0,11 i $p =$ 0,87)

Ogólnie model wyjaśniał od 4 do 30\% procent wariancji lubienia siebie (z prawdopodobieństwem $p =$ 0,95). Wykresy rozrzutu wraz z liniami dopasowania przedstawiono na Rycinie \ref{fig:study2a}.\\

\begin{table*}[htbp]
\vspace*{2em}
\centering
\begin{threeparttable}
\caption{Kompetencje, moralność i ciepło jako predyktory lubienia siebie i przekonań o sprawczości Ja -- podsumowanie rozkładów brzegowych parametrów modeli.}
\label{tab:2}
\bgroup
\def\tabcolsep{4pt}
\begin{tabular}{lrrrrrrrrr}
\midrule
 &
\multicolumn{3}{c}{Ja} &
\multicolumn{3}{c}{Ja jako Amerykanin} &
\multicolumn{3}{c}{My Amerykanie} \\
\cline{2-10}
 & $M_{post.}$    & $SD$   & $95\%\ CI$   & $M_{post.}$    & $SD$   & $95\%\ CI$   & $M_{post.}$    & $SD$   & $95\%\ CI$   \\
\midrule
 \multicolumn{10}{c}{\emph{Lubienie siebie}}  \\
 Stała       &  1,05 & 0,54 & -0,01;2,12 &  1,42 & 0,46 &  0,52;2,31 &  2,33 & 0,40 &  1,56;3,12 \\
 Kompetencje &  0,18 & 0,06 &  0,06;0,30 &  0,05 & 0,08 & -0,10;0,20 & -0,07 & 0,06 & -0,19;0,04 \\
 Moralność   &  0,02 & 0,07 & -0,11;0,15 &  0,04 & 0,07 & -0,10;0,19 &  0,18 & 0,06 &  0,07;0,30 \\
 Ciepło      &  0,14 & 0,05 &  0,05;0,23 &  0,22 & 0,06 &  0,11;0,33 &  0,07 & 0,06 & -0,05;0,19 \\
 \multicolumn{10}{c}{\emph{Sprawczość Ja}}  \\
 Stała       &  0,80 & 0,45 & -0,09;1,71 &  1,90 & 0,36 &  1,20;2,62 &  1,35 & 0,33 &  0,70;1,99 \\
 Kompetencje &  0,28 & 0,05 &  0,18;0,38 &  0,10 & 0,06 & -0,02;0,22 &  0,13 & 0,05 &  0,03;0,23 \\
 Moralność   &  0,02 & 0,06 & -0,10;0,13 & -0,02 & 0,06 & -0,14;0,11 &  0,18 & 0,05 &  0,08;0,28 \\
 Ciepło      &  0,12 & 0,04 &  0,04;0,19 &  0,16 & 0,05 &  0,07;0,26 &  0,07 & 0,06 & -0,04;0,17 \\
\bottomrule
\end{tabular}
\egroup
\begin{tablenotes}[para,flushleft]
{\small
\textit{Nota.} $N = 103$. Dobroć dopasowania modelu dla lubienia siebie: $R^2$ = 0,15, 95\% CI = [0,04;0,30]. Dobroć dopasowania dla sprawczości: $R^2$ = 0,43, 95\% CI = [0,23;0,64]
}
\end{tablenotes}
\end{threeparttable}
\end{table*}


\begin{figure*}[htbp]
   \centering
   \fitfigure{study2a.pdf}
   \caption{Autoaskrypcja cech związanych w wymiarami kompetencji, moralności i ciepła, a poziom lubienia siebie/nas. Punkty oznaczają latentne wyniki dla każdej osoby, z kreskami oznaczającymi błędy pomiarowe. Grubą linią przerywaną oznaczono najlepsze dopasowanie uzyskane w modelu regresyjnym, z cieńszymi liniami oznaczającymi błąd oszacowania.}
   \label{fig:study2a}
\end{figure*}


\paragraph{Autoaskrypcja kompetencji, moralności i ciepła, a poziom przekonań o własnej sprawczości}
W drugim kroku analiz sprawdzono w jakim stopniu wymiary kompetencji, moralności i ciepła wyznaczają pozytywną samoocenę rozumianą jako przekonanie o radzenia sobie w różnych sytuacjach. W tym celu powtórzono analizy z wykorzystaniem modelu z poprzedniej sekcji, ale jako zmienną objaśnianą uzwględniono skalę sprawczości Ja\footnote{Parametry posterior uzyskaną stosując algorytm MCMC z 4000 iteracji. Wartość wskaźnika PSRF nie przekraczała, R = 1,01, i nie wskazywała na brak konwergencji symulacji.}. \\

W warunku `ja' przekonania o sprawczości uczestników były związane z przypisywanymi sobie cechami ze skal kompetencji (parametr był pozytywne z $p >$ 0,999), ale również ciepła (parametr był pozytywny z $p =$ 0,99). Nie były one jednak związane z przypisywanymi sobie cechami ze skali moralności (parametr był pozytywny $p =$ 0,60). \\

W drugim warunku, przekonania o sprawczości siebie `jako Amerykanina' były związane jedynie z przypisywanymi sobie przez uczestników cechami ze skali ciepła (parametr był pozytywny z, $p =$ 0,99), nie zaobserwowano związku z przypisywaniem sobie cech ze skali moralności (parametr był pozytywny z $p =$ 0,40). Zaobserwowano słaby związek z przypisywaniem sobie cech ze skali kompetencji (parametry był pozytywny z, $p =$ 0,93).\\

Wśród uczestników pytanych o postrzeganie `nas, Amerykanów', przekonania o sprawczości były związane z przypisywaniem Amerykanom cech ze skal moralności i kompetencji (parametry dla tych skal były pozytywne z prawdopodobieństwem odpowiednio, $p >$ 0,999 i $p =$ 0,99). Nie zaobserwowano związku z przypisywaniem Amerykanom cech ze skali ciepła (parametr był pozytywny z $p =$ 0,89). \\

Wykresy rozrzutu wraz liniami dopasowania przedstawiono na Rycinie \ref{fig:study2b}. Cały model wyjaśniał -- z $p =$ 0,95 od 23 do 64\% wariancji przekonań o sprawczości Ja.\\

\begin{figure*}[htbp]
   \centering
   \fitfigure{study2b.pdf}
   \caption{Autoaskrypcja cech związanych w wymiarami kompetencji, moralności i ciepła, a poziom przekonań o własnej sprawczości. Punkty oznaczają latentne wyniki dla każdej osoby, z kreskami oznaczajacymi błędy pomiarowe. Grubą linią przerywaną oznaczono najlepsze dopasowanie uzyskane w modelu regresyjnym, z cieńszymi liniami oznaczającymi błąd oszacowania.}

   \label{fig:study2b}
\end{figure*}


\subsubsection{Dyskusja}
Celem przedstawionego badania była replikacja wyników uzyskanych w Badaniu 1 na próbie niestudenckiej i poza polskim kontekstem. Badanie przeprowadzone na amerykańskiej quasi-reprezentatywnej próbie internetowej potwierdziło, że wyznaczniki samooceny zależą od skupienia na poziomie tożsamości: indywidualnej vs. kolektywnej.\\
Poziom samooceny rozumianej jako lubienie siebie był wyznaczany: w przypadku ja indywidualnego przez postrzeganie siebie na wymiarze kompetencji, w przypadku siebie jako członka grupy przez postrzeganie siebie na wymiarze ciepła, natomiast w przypadku grupy jako całości przez postrzeganie jej na wymiarze moralności.\\
W niniejszym badaniu postanowiono również sprawdzić w jakim stopniu autoaskrypcje kompetencji, moralności i ciepła są związane z indywidualną i kolektywną samooceną rozumianą jako przekonania o sprawczości Ja. Choć z pozoru taka analiza wydaje się redundantna -- spodziewać się tu można szczególnie silnej roli cech związanych z wymiarem kompetencji jako predyktorów tak rozumianej samooceny -- uzyskane wyniki wskazały na podobny charakter zależności jak w przypadku samooceny rozumianej jako lubienie siebie. Przekonania o indywidualnej sprawczości były wyznaczane przez wymiar kompetencji, przekonania o sprawczości siebie jako członka grupy przez wymiar ciepła, natomiast przekonania o sprawczości Polaków jako grupy przez wymiar moralności.\\



\newpage
\subsection{Badanie 3}
Celem Badania 3 było sprawdzenie, czy aktywizacja pozytywnych asocjacji wymiarów kompetencji, moralności i ciepła z indywidualnym vs. kolektywnym Ja wpłynie na poziom samooceny. O ile w poprzednich dwóch badaniach sprawdzano jedynie zależności korelacyjne, tak w tym podjęto się próby ustalenia zależności przyczynowo skutkowej. Zastosowanie schematu eksperymentalnego miało też pozwolić na lepszą kontrolę procedury badawczej i sprawdzić w jakim stopniu sprawczość i wspólnotowość wyznaczają indywidualną i kolektywną samoocenę rozumianą nie jako cecha, ale raczej jako stan.\\
Na podstawie poprzednich dwóch badań postawiono hipotezę mówiącą, że aktywizacja asocjacji wymiaru kompetencji z Ja wpłynie pozytywnie jedynie na poziom indywidualnej samooceny. Z kolei aktywizacja asocjacji Ja z wymiarami ciepła i moralności wpłynie pozytywnie na poziom samooceny odpowiednio siebie jako Polaka i Polaków jako grupy. \\

\subsubsection{Metoda}

\paragraph{Uczestnicy i schemat badania}
W eksperymencie wzięło udział $N = 184$ (90 kobiet i 94 mężczyzn) użytkowników internetowego panelu badawczego w wieku od 20 do 35 lat ($M =$ 26,63; $SD =$ 3,84). Po wyrażeniu zgody na udział w badaniu uczestnicy byli losowo przydzielani do jednego z 4 warunków: w trzech z grup u uczestników aktywizowano treści związane z wymiarami kompetencji lub moralności lub ciepła, natomiast czwarta grupa stanowiła warunek kontrolny. Następnie w ramach każdego z warunków uczestnicy byli losowo przydzielani do jednego z trzech wariantów: `ja jako jednostka' lub `ja jako Polak' lub `my Polacy'. Tak więc w sumie uczestnicy byli przydzieleni do jednego z 12 warunków (średnia liczba uczestników na warunkek wynosiła, $n = 15$); losowanie przeprowadzono z zachowaniem proporcji płci. Po przejściu procedury manipulacji każdy uczestnik wypełniał miarę samooceny oraz miarę skuteczności manipulacji, a następnie był przekierowywany na stronę na której wyjaśniano mu cel badania oraz cel oddziaływania eksperymentalnego.

\paragraph{Procedura manipulacji}
Każdy uczestnik był proszony o zapoznanie się z listą 8 cech, a następnie o przypomnienie sobie sytuacji w której on (lub w zależności od wariantu: on jako Polak lub Polacy jako grupa) wykazał/wykazali się którąś z cech z listy. Uczestnicy byli proszeni o zastanowienie się przez chwilę nad zdarzeniem, a następnie o przypisanie sobie do trzech cech, którymi wykazali się w przypominanej sytuacji (poprzez przeniesienie cechy do sąsiadującego okna). W zależności od warunku lista cech składała się tylko z cech dotyczących jednego z wymiarów: kompetencji (np. \emph{sprawny, inteligentny}) lub moralności (np. \emph{moralny, sprawiedliwy}) lub ciepła (np. \emph{ciepły, przyjazny}). W warunku kontrolnym listę cech zastąpiono 8 stwierdzeniami dotyczącymi kwestii społeczno-politycznych (np. \emph{Globalizacja niesie za sobą więcej szkody niż pożytku}) i w tym przypadku uczestnik był proszony o przypisanie do trzech najbardziej odpowiadających mu stwierdzeń (lub takich które odpowiadają mu jako Polakowi lub Polakom jako grupie).

\paragraph{Miary}
Podobnie jak w poprzednich badaniach, stwierdzenia w zastosowanych miarach posiadały identyczne człony główne, a różniły się pomiędzy wariantami jedynie formą: dotyczyły `ja' jako jednostki, jako Polaka lub Polaków jako grupy.

\subparagraph{Skala samooceny}
W eksperymencie zastosowano identyczną skalę samooceny jak w Badaniu 1. Składała się ona z 13 stwierdzeń (w tym 6 kodowanych odwrotnie), np. \emph{Uważam, że jestem osobą wartościową przynajmniej w takim samym stopniu co inni}. Uczestnicy odpowiadali na ile zgadzają się z każdym ze stwierdzeniem na skali od 1 -- \emph{Zdecydowanie się nie zgadzam} do 4 -- \emph{Zdecydowanie się zgadzam}. \\
Skala cechowała się zadowalającą zgodnością wewnętrzną w każdym z 3 wariantów: ja jako jednostka, $\alpha$ = 0,91; ja jako Polak, $\alpha$ = 0,92; my Polacy, $\alpha$ = 0,85.
\subparagraph{Pomiar skuteczności manipulacji}
Aby sprawdzić, czy zastosowana procedura manipulacji prowadziła do aktywizacji treści związanych z wymiarami kompetencji, moralności i ciepła po wypełnieniu skali samooceny każdy uczestnik przechodził do części, w której na osobnych stronach prezentowano mu 8 cech: 2 dwie pierwsze o charakterze buforowym (\emph{realistyczny, poważny}) oraz po dwie cechy związane z wymiarami kompetencji (\emph{dokładny, bystry}), moralności (\emph{prawy, szczery}), i ciepła (\emph{miły, troskliwy}). Żadna z prezentowanych cech nie pojawiała się w procedurze manipulacji. \\
Uczestnicy mieli ocenić na ile każda z cech jest dla nich charakterystyczna (lub w zależności od wariantu dla nich jako Polaków lub dla Polaków jako grupy) i zaznaczyć odpowiedź jak najszybciej to możliwe na skali od 1 -- \emph{Zdecydowanie nie} do 7 -- \emph{Zdecydowanie tak}. Co istotne, oprócz odpowiedzi dla każdej cechy kodowano czas w którym uczestnicy dokonali pierwszego kliknięcia. Założono, że aktywizacja wymiaru kompetencji, moralności lub ciepła będzie prowadziła do szybszej reakcji na cechy związane z każdym z wymiarów.

\subsubsection{Wyniki}
\paragraph{Sprawdzenie skuteczności manipulacji}
Aby upewnić się co do skuteczności manipulacji porównano czasy reakcji na cechy związane z wymiarami kompetencji, moralności i ciepła w ramach każdego z czterech warunków eksperymentalnych. W tym celu przeprowadzono 3 analizy kontrastów w ramach hierarchicznego modelu liniowego \parencite[odpowiednika  klasycznej jednoczynnikowej analizy wariancji, za:,][]{kruschke2014doing} z odpornymi błędami rozproszonymi zgodnie z rozkładem t Studenta (z liczbą stopni swobody, $\nu = 7$).\\
W ramach pierwszej analizy porównano czasy odpowiedzi na pytania dotyczące wymiaru kompetencji. Osoby przypominające sobie wydarzenia dotyczące kompetencji odpowiadały na te pytania o 1,40 sekundy szybciej niż osoby w pozostałych warunków -- różnica była większa od 0 z $p =$ 0,99 -- i o 0,75 sekundy szybciej niż w warunku kontrolnym -- różnica większa od 0 z $p =$ 0,99.\\
Następnie porównano czasy odpowiedzi na pytania dotyczące wymiaru moralności. U osób przypominających sobie wydarzenia dotyczące moralności odpowiedzi na te pytania były szybsze o 0,61 sekundy niż w pozostałych warunkach -- różnica większa od 0 z $p =$ 0,87 -- i szybsze o 0,46 sekundy niż w warunku kontrolnym -- różnica większa od 0 z $p =$ 0,97. \\
W końcu porównano odpowiedzi na pytania dotyczące wymiaru ciepła. U osób przypominających sobie wydarzenia dotyczące ciepła czasu odpowiedzi na pytania dotyczące tego wymiaru były szybsze o 0,16 sekundy -- różnica była większa od 0 z $p =$ 0,80. Nie zanotowano różnicy pomiędzy warunkiem dotyczącym ciepła a średnią z pozostałych warunków -- różnica była większa od 0 z $p =$ 0,49.\\
Analogiczne analizy dla odpowiedzi nie przyniosły różnic pomiędzy warunkami. Pomimo umiarkowanego poparcia dla skuteczności manipulacji (zwłaszcza w przypadku aktywizacji wymiaru ciepła), postanowiono sprawdzić w jakim stopniu zastosowane oddziaływanie wpłynęło na poziom samooceny.\\

\paragraph{Aktywizacja wymiarów komptetencji, moralności i ciepła a poziom uogólnionej samooceny}
Aby zweryfikować hipotezę mówiącą, że wpływ aktywizacji wymiarów kompetencji, moralności i ciepła na samoocenę będzie różny w zależności od poziomu Ja (tj. że aktywizacja wymiaru kompetencji wpłynie na poziom indywidualnej samooceny, aktywizacja wymiaru moralności na poziom samooceny Polaków jako grupy, a aktywizacja ciepła na poziom samooceny siebie jako Polaka) utworzono liniowy model hierarchiczny \parencite[jako odpowiednik klasycznej dwuczynnikowej analizy wariancji, za:,][]{kruschke2014doing}.
\begin{equation}\label{eq:4}
\begin{split}
y_i  \sim\ & N(\mu_i, \sigma_y) \\
\mu_i  = \beta_0 + \sum_j\beta_{1[j]}x_{1[j]}(i)\ +\ &\sum_k\beta_{2[k]}x_{2[k]}(i)+\sum_{j,k}\beta_{1\times2[j,k]}x_{1\times2[j,k]}(i) \\
\sigma_y \sim U(0.01,6.41)\ &\ \beta_0 \sim N(2.80, 3.21) \\
\beta_1 \sim N(0, \sigma_{\beta1})\ &\ \beta_2 \sim N(0, \sigma_{\beta2}) \\
\beta_{1\times2} \sim\ & N(0, \sigma_{\beta1\times2}) \\
\sigma_{\beta1} \sim IG(1.28,0.88)\ \ \sigma_{\beta2} \sim\ & IG(1.28,0.88)\ \ \sigma_{\beta1\times2} \sim IG(1.28,0.88)
\end{split}
\end{equation}
gdzie $y$ oznacza poziom samooceny, $x_1$ warunek eksperymentalny, a $x_2$ wariant eksperymentu (oznaczenia rozkładów: $N$ -- rozkład Normalny; $U$ -- rozkład prostokątny; $IG$ -- odwrócowny rozkład Gamma). Dzięki takiemu sformułowaniu modelu możliwe było obliczenie odchyleń od średniej ogólnej dla efektów głównych warunku oraz wariantu eksperymentu, a także sformułowanie kontrastów. \\
Aby uzyskać oszacowania parametrów rozkładu posterior wykorzystano algorytm MCMC zaimplementowany w programie Stan \parencite{carpenter2016} z 4000 iteracji. Analiza wskaźnika PSRF nie dostarczyła powodów dla odrzucenia symulacji ze względu na brak konwergencji, $Rs <$ 1,02. Podsumowanie parametrów posterior przedstawiono w Tabeli \ref{tab:3} \\
W pierwsze kolejności sprawdzono, czy występują różnice średnich samooceny pomiędzy wariantami eksperymentu (ja vs. ja jako Polak vs. my Polacy). Analiza rozkładów wiarygodności odchyleń dla tych wariantów wskazała, że w stosunkowo dużej proporcji pokrywają się one ze sobą. Procedura porównań parami wykazała, że prawdopodobieństwo różnicy pomiędzy którymkolwiek z wariantów jest mniejsze niż, $p =$ 0,77. \\
W następnym kroku zweryfikowano alternatywne hipotezy mówiące o tym, że osoby przypominające sobie wydarzenia świadczące o wysokich kompetencjach lub moralności lub cieple będą cechowały się wyższym poziomem samooceny niezależnie od wariantu. W tym celu sformułowano trzy kontrasty. W ramach pierwszego porównano osoby przypominające sobie wydarzenia świadczące o wysokich kompetencjach z osobami z pozostałych grup. Osoby przypominające sobie wydarzenie świadczące o kompetencjach cechowały się samooceną wyższą o 0,42 -- różnica była większa od 0 z $p =$ 0,90. W ramach drugiego kontrastu porównano osoby przypominające sobie wydarzenia świadczące o moralności z osobami z pozostałych grup. Osoby z tej pierwszej grupy cechowały się samooceną niższą o 0,01 od pozostałych grup -- różnica była większa od 0 z $p =$ = 0,49. W ramach trzeciego kontrastu porównano osoby przypominające sobie wydarzenia świadczące o cieple z resztą grup. W tym przypadku osoby z pierwszej grupy cechowały się samooceną niższą o 0,19 od pozostałych grup -- różnica była większa od wartości 0 z $p =$ 0,28. \\
W ostatnim kroku postanowiono zweryfikować główną hipotezę mówiącą o tym, że treści zwiększające poziom samooceny będą zależały od poziomu Ja. W tym celu porównano w ramach jednego klastra: osoby przypominające sobie wydarzenia świadczące o kompetencjach dla wariantu indywidualnego, osoby przypominające sobie wydarzenia świadczące o cieple w ramach wariantu ja jako Polak, oraz osoby przypominające sobie wydarzenia świadczące o moralności z osobami z pozostałych grupy. Osoby z pierwszego klastra cechowały się samooceną wyższą o 1,66 niż osoby z pozostałych grup -- różnica była większa od 0 z $p =$ 0,98 (por. \ref{fig:study3}). Dodatkowe analogiczne analizy przeprowadzono osobno dla każdego z poziomów. W ramach wariantu indywidualnego osoby przypominające sobie o kompetencjach cechowały się samooceną wyższą o 0,57 od osób pozostałych grup w wariancie indywidualnym -- różnica była większa od zera z $p =$ 0,93. W ramach wariantu ja jako Polak osoby przypominające sobie o cieple cechowały się samooceną wyższą o 0,58 niż osoby z pozostałych grup dla tego wariantu -- różnica była większa od zera z $p =$ 0,93. W końcu w ramach wariantu my Polacy osoby przypominające sobie o moralności Polaków cechowały się samooceną wyższą o 0,62 niż osoby przypominające sobie o cieple lub kompetencjach Polaków. Jak można zauważyć na Rycinie \ref{fig:study3}, w ramach tego wariantu różnica pomiędzy warunkiem kontrolnym w warunkiem z przypominaną moralnością była bliska wartości 0 -- większa od 0 z $p =$ 0,57.
\subsubsection{Dyskusja}
Celem tego badania było sprawdzenie, czy wzbudzenie pozytywnych asocjacji Ja z wymiarami kompetencji, moralności lub ciepła podniesie poziom samooceny na różnych poziomach tożsamości. Wyniki poprzednich badań 1 i 2 sugerowały, że pozytywne asocjacje z wymiarem kompetencji będą podnosiły jedynie poziom indywidualnej samooceny, natomiast pozytywne asocjacje z wymiarami ciepła i moralności będą podnosiły poziom samooceny odpowiednio siebie jako członka grupy i dla samooceny Polaków jako grupy. Analiza danych przyniosła potwierdzenie dla tego interakcyjnego układu zależności: przypomnienie o wydarzeniach świadczących o kompetencjach podnosiło poziom indywidualnej samooceny (w porównaniu do pozostałych warunków w ramach wariantu), a przypomnienie o wydarzeniach świadczących o cieple podnosiło poziom samooceny siebie jako Polaka (w porównaniu do pozostałych warunków). W końcu przypomnienie o wydarzeniach świadczących o moralności podnosiło poziom samooceny Polaków jako grupy (w porównaniu do warunków dotyczących wymiarów kompetencji i ciepła). \\


\begin{table*}[htbp]
\vspace*{2em}
\centering
\begin{threeparttable}
\caption{Aktywizacja kompetencji, moralności i ciepła a poziom samooceny indywidualnej, siebie jako Polaka i Polaków jako grupy -- podsumowanie rozkładów przegowych parametrów modelu.}
\label{tab:3}
\bgroup
\def\arraystretch{0.85}
\begin{tabular}{lrrrr}

\midrule
 & $M_{post.}$    & $SD_{post.}$   & $2.5\%\ CI$ &  $97.5\%\ CI$  \\
\midrule

                        & Ja        & Ja Polak  & My Polacy & \emph{Średnie brzegowe}\\
Kontrolna               & 2,61      & 2,70      & 2,90      & 2,75 \\
Kompetencje             & 3,05      & 2,84      & 2,82      & 2,92 \\
Moralność               & 2,84      & 2,56      & 2,99      & 2,81 \\
Ciepło                  & 2,69      & 2,82      & 2,72      & 2,76 \\
\emph{Średnie brzegowe} & 2,80      & 2.73      & 2.86      & 2,81 \\

\bottomrule
\end{tabular}
\egroup
\begin{tablenotes}[para, flushleft]
{\small
\textit{Nota.} $N = 184$
}
\end{tablenotes}
\end{threeparttable}
\end{table*}

\begin{figure*}[htbp]
   \centering
   \fitfigure{study3.pdf}
   \caption{Aktywizacja kompetencji, moralności i ciepła a poziom samooceny indywidualnej i kolektywnej -- zobrazowanie rozkładów brzegowych dla nieaddytywnych/interakcyjnych odchyleń od średniej ogólnej.}
   \label{fig:study3}
\end{figure*}



\newpage
\subsection{Badanie 4}

W Badaniach 1 i 2 aspekty związane z moralnością były głównymi wyznacznikami samooceny grupowej, natomiast aspekty związane z kompetencjami nie odgrywały znaczącej roli w wyznaczaniu kolektywnej samooceny ani Polaków ani Amerykanów. Jak wskazuje model perspektywy sprawcy i biorcy \parencite{wojciszke2011self} dominacja treści wspólnotowych (w tym moralności) jest charakterystyczna dla osób przyjmujących perspektywę biorcy. Dlatego dominacja moralności jako wyznacznika kolektywnej samooceny może wskazywać, że aktywizacja treści związanych z kolektywnym Ja aktywizuje jednocześnie perspektywę biorcy. Jak wykazały raportowane w tej pracy badania, może tak być nawet wtedy gdy ludzie myślą, nie o abstrakcyjnej grupie, ale o sobie jako o członku grupy. \\
Z drugiej strony jak wykazano w Badaniu 2, wymiar kompetencji jest ważnym -- obok wymiaru moralności -- wyznacznikiem kolektywnej samooceny rozumianej jako przekonanie o zdolności radzenia sobie w różnych sytuacjach. Niewątpliwe zdarzają się sytuacje, gdy grupa własna przestaje być bezpieczną przystanią i jej członkowie muszą zacząć wspólnie działać jako kolektywni sprawcy. Wyniki Badania 2 mogą sugerować, że wymiar kompetencji jest nie tyle nieistotny dla kolektywnego Ja, co uaktywniany w specjalnych, kryzysowych sytuacjach, gdy członkowie grupy przyjmują perspektywę kolektywnego sprawcy zamiast domyślnej perspektywy biorcy. Celem Badania 4 była weryfikacja tej hipotezy. \\
Przyjęto, że jednym z czynników mogących skłaniać do wchodzenia w perspektywę kolektywnego sprawcy jest aktywizacja myśli o egzystencjalnym zagrożeniu dla grupy własnej, np. przypomnienie o zagrożeniu terrorystycznym lub o zbliżających się działaniach wojennych. Soral i Kofta (niepublikowany raport z badań) pokazali, że przypomnienie o tego rodzaju zagrożeniach może zwiększać dostępność poznawczą cech związanych specyficznie z wymiarem kompetencji. W tym badaniu eksperymentalnym postanowiono wykazać, że przypomnienie o zagrożeniach prowadzić będzie do zwiększenie znaczenia wymiaru kompetencji jako wyznacznika kolektywnej samooceny.

\subsubsection{Metoda}

\paragraph{Uczestnicy i schemat badania}
W eksperymencie wzięło udział $N=343$ (191 kobiet, 140 mężczyzn, 12 osób nie określiło swojej płci) użytkowników internetowego panelu badawczego w wieku od 18 od 69 lat ($M$ = 36,21; $SD$ = 12,82). Po wyrażeniu zgody na udział w badaniu uczestnicy byli losowo przydzielani do jednego z dwóch warunków: kontrolnego ($n = 170$) lub warunku z zagrożeniem ($n = 173$). Po wypełnieniu kwestionariusza każdy uczestnik był przekierowywany na stronę, na której wyjaśniano cel badania i zastosowaną manipulację eksperymentalną.

\paragraph{Procedura manipulacji}
Każdy uczestnik w warunku z zagrożeniem był przekierowywany na początku eksperymentu na stronę zawierającą spreprarowany artykuł omawiający obecne zagrożenia dla Polaków i proszony o zapoznanie się z jego treścią. Aby uniknąć przypadkowego pominięcia artykułu specjalny skrypt na stronie uniemożliwiał uczestnikom przejście do kolejnych części badania przed upływem 30 sekund. Po upływie określonego czasu uczestnicy w warunku z zagrożeniem przystępowali do wypełniania kwestionariusza. Uczestnicy w grupie kontrolnej przystępowali do wypełniania kwestionariusza od razu po rozpoczęciu eksperymentu. \\
W treści artykułu pominięto określenia lub stwierdzenia odwołujące się wprost do wymiarów kompetencji, moralności i ciepła. Uczestnicy zapoznawali się z następującą treścią:
\blockquote{``Żyjemy w czasach permanentnego kryzysu. Te czasy stanowią realne zagrożenie dla wszystkich Polaków'' -- mówi profesor UW. Jeszcze nie ucichły echa kryzysu na Ukrainie -- i związanego z nim niepokoju: Gdzie zatrzyma się Rosja Putina? -- a już okazuje się, że możemy stać przed poważniejszym, bardziej długotrwałym problemem. Wojna w Syrii zapoczątkowała falę emigracji, która zalewa od kilku miesięcy Europę. Niedawne tragiczne wydarzenia na ulicach Paryża, pokazują czym kończy się lekceważenie ISIS. To wszystko skłania do zadania pytania: Czy Polacy mogą czuć się bezpieczni? Zdaniem Profesora UW Krzysztofa Wilczka, ta sytuacji stanowi zagrożenie dla wszystkich europejczyków, również Polaków. ``Polacy -- od kilkudziesięciu lat żyjący we względnej stabilizacji i spokoju -- zdążyli już zapomnieć czym jest zagrożenie dla bezpieczeństwa'' -- mówi Profesor -- ``Obawiam się, że w ciągu kilku lub kilkunastu miesięcy ten obserwowany na razie tylko poza naszymi granicami kryzys, przywędruje również do Polski i Polaków.''}
Badanie pilotażowe wykazało, że ten rodzaj manipulacji jest skuteczny we wzbudzaniu poczucia zagrożenia.

\paragraph{Miary}
\subparagraph{Skala kompetencji, moralności i ciepła}
Natychmiast po zapoznaniu się z artykułem (lub w warunku kontrolnym, po rozpoczęciu badania) uczestnicy w obu warunkach oceniali Polaków (\emph{my Polacy jesteśmy...}) przy użyciu 21 pozytywnych cech związanych z wymiarami kompetencji (np. \emph{zdolni, umiejętni}), moralności (np. \emph{uczciwi, sprawiedliwi}), lub ciepła (np. \emph{przyjaźni, sympatyczni}). W przypadku każdej cechy uczestnicy oceniali czy pasuje ona do Polaków czy nie przy użyciu 7-stopniowej skali od 1 -- \emph{Zdecydowanie nie} do 7 -- \emph{Zdecydowanie tak}.\\
Zastosowana wersja skali została zapożyczona z badań Wojciszke i współpracowników (niepublikowany raport z badań). Poszczególne subskale charakteryzowały się wysoką zgodnością wewnętrzną. Podsumowanie charakterystyk przedstawiono w Tabeli \ref{tab:5}.

\subparagraph{Skala kolektywnej samooceny}
Po dokonaniu askrypcji cech uczestnicy wypełniali składającą się z 13 pozycji skalę kolektywnej samooceny. Zastosowana skala była identyczna do skal zastosowanych w badaniach 1 i 3, a zawarte w niej stwierdzenia były przedstawione w formie odnoszącej się do Polaków jako grupy: ``my, Polacy'' Zadaniem każdego z uczestników była ocena na ile zgadzają się z każdym ze stwierdzeń na skali od 1 -- \emph{Zdecydowanie się nie zgadzam} do 4 -- \emph{Zdecydowanie się zgadzam}. Podsumowanie charakterystyk skali przedstawiono w Tabeli \ref{tab:5}.

\begin{table*}[htbp]
\vspace*{2em}
\centering
\begin{threeparttable}
\caption{Podstawowe statystyki opisowe dla skal użytych w Badaniu 4}
\label{tab:5}

\begin{tabular}{lrrrrrr}

\midrule
& \multicolumn{3}{c}{Kontrolna} & \multicolumn{3}{c}{Zagrożenie}  \\
\cline{2-7}
& $\alpha$ & $M$ & $SD$ & $\alpha$ & $M$ & $SD$  \\
\midrule
Kompetencje            & 0,93 & 5,28 & 1,00 & 0,92 & 5,18 & 1,05  \\
Moralność              & 0,94 & 4,71 & 1,12 & 0,94 & 4,52 & 1,20  \\
Ciepło                 & 0,93 & 5,06 & 1,03 & 0,93 & 4,82 & 1,15  \\
Kolektywna samoocena   & 0,84 & 2,94 & 0,47 & 0,85 & 2,81 & 0,52  \\

\bottomrule

\end{tabular}

\begin{tablenotes}[para,flushleft]
{\small
\textit{Nota.} $N$ = 343
}
\end{tablenotes}
\end{threeparttable}
\end{table*}

\subparagraph{Skala poczucia zagrożenia} Aby ocenić skuteczność manipulacji poproszono, aby każdy uczestnik ocenił prawdopodobieństwo (na skali od 0 do 100) zajścia każdego z trzech zdarzeń: \emph{}. Oceny charakteryzowały się zadowalającą zgodnością wewnętrzną, $\alpha$ = 0,69.

\subsubsection{Wyniki}
\paragraph{Sprawdzenie skuteczności manipulacji}
Aby ocenić skuteczność manipulacji porównano średnie ocen prawdopodobieństwa zajścia zagrażających zdarzeń w warunku kontrolnym ($M$ = 40,08; $SD$ = 16,69) i w warunku z aktywizacją poczucia zagrożenia ($M$ = 45,38; $SD$ = 18,19). Przy obliczaniu średnich i ich różnicy uwzględniono niepewność związaną z błędem pomiarowym zmiennej zależnej (obliczonym według wzoru \ref{eq:err}). Średnia ocen wystąpienia zagrażających zdarzeń okazała się większa w warunku z aktywizacją zagrożenia niż w warunku kontrolnym, z prawdopodobieństwiem $p$ = 0,99. Standaryzowana różnica średnich wynosiła, d = 0,30 i z prawdopodobieństwm 95\% mieściła się w przedziale: [0,04;0,56]. Choć wzrost poczucia zagrożenia okazał się być subtelny, manipulacja eksperymentalna była skuteczna.

\paragraph{Zagrożenie a wyznaczniki samooceny}
Aby sprawdzić w jaki sposób zagrożenie decydować będzie o wyznacznikach kolektywnej samooceny skonstruowano model analogiczny do tego stosowanego w Badaniach 1 i 2 (patrz: Równanie \ref{eq:1}), tj. sprawdzono w jakim stopniu kompetencje, moralność i ciepło będą decydować o kolektywnej samoocenie, osobno w warunku kontrolnym i w warunku z zagrożeniem. Ponieważ, użyte miary charakteryzowały się bardzo wysoką zgodnością wewnętrzną nie uwzględniano niepewności wynikającej z błędu pomiarowego. Jako rozkład prior dla parametrów w warunku kontrolnym przyjęto wyniki uzyskane w ramach Badania 2 (patrz górna część Tabeli \ref{tab:2}). W warunku z zagrożeniem, ze względu na wprowadzone oddziaływanie przyjęto nieinformatywny rozkład prior dla parametrów równania regresji.\\
Aby otrzymać rozkład parametrów posterior wykorzystano algorytm MCMC zaimplementowany w programie Stan z 4000 iteracji. Wartości PSRF nie przekraczały, R = 1,01 i nie wskazywały na konieczność odrzucenia symulacji ze względu na brak konwergencji. Podsumowanie parametrów przedstawiono w Tabeli \ref{tab:4}. \\
W ramach warunku kontrolnego pozytywnym predkyktorem kolektywnej samooceny okazał się wymiar moralności -- z prawdopodobieństwem, $p =$ 0,99 -- oraz wymiar ciepła -- z $p =$ 0,97. Wartość parametru dla wymiaru kompetencji była zbliżona do 0 i pozytywna z $p =$ 0,73. \\
W ramach warunku z zagrożeniem pozytywnym predyktorem kolektywnej samooceny okazał się wymiar kompetencji -- z prawdopodobieństwem, $p =$ 0,99. Wartości parametrów dla wymiaru moralności i ciepła były pozytywne z prawdopodobieństwami, odpowiednio $p =$ 0,96 i $p =$ 0,91.\\
W następnej kolejności, dla każdego wymiaru porównano wartości parametrów w warunku kontrolnym i w warunku z zagrożeniem. Dla wymiaru kompetencji wartość parametru w warunku z zagrożeniem była większa niż w warunku kontrolnym z prawdopodobieństwem, $p =$ 0,97. Parametry dla wymiarów moralności i ciepła były z kolei mniejsze w warunku z zagrożeniem niż w warunku kontrolnym, ale różnice pomiędzy warunkami były małe -- mniejsze od 0 z prawdopodobieństwem $p = $ 0,67 dla wymiaru moralności i $p =$ 0,56 dla wymiaru ciepła.\\
Cały model wyjaśniał z $p =$ 0,95 od 5 do 16\% wariancji kolektywnej samooceny. Porównanie linii regresji pomiędzy dwoma warunkami przedstawiono na Rycinie \ref{fig:study4}.\\

\begin{table*}[htbp]
\vspace*{2em}
\centering
\begin{threeparttable}
\caption{Kompetencje, moralność i ciepło jako predyktory kolektywnej samooceny a aktywizacja zagrażających treści -- podsumowanie rozkładów brzegowych parametrów modelu.}
\label{tab:4}
\begin{tabular}{lrrrrrr}

\midrule
 &
\multicolumn{3}{c}{Kontrolna} &
\multicolumn{3}{c}{Zagrożenie} \\
\cline{2-7}
 & $M_{post.}$    & $SD$   & $95\%\ CI$   & $M_{post.}$    & $SD$   & $95\%\ CI$  \\
\midrule
 Stała       &  1,87 & 0,16 &  1,55;2,18 &  1,34 & 1,02 &  1,02;1,66 \\
 Kompetencje &  0,02 & 0,04 & -0,06;0,09 &  0,14 & 0,05 &  0,04;0,24 \\
 Moralność   &  0,11 & 0,04 &  0,04;0.19 &  0,08 & 0,05 & -0,02;0,19 \\
 Ciepło      &  0,09 & 0,04 &  0,00;0,17 &  0,08 & 0,06 & -0,04;0,20 \\
\bottomrule
\end{tabular}
\begin{tablenotes}[para,flushleft]
{\small
\textit{Nota.} $N = 343$. Dobroć dopasowania modelu: $R^2$ = 0,10 95\% CI = [0,05;0,16].
}
\end{tablenotes}
\end{threeparttable}
\end{table*}

\begin{figure*}[htbp]
   \centering
   \fitfigure{study4.pdf}
   \caption{Kompetencje, moralność i ciepło jako wyznaczniki kolektywnej samooceny w warunku kontrolnym vs. w warunku z zagrożeniem. Na rysunku przedstawiono linie dopasowania wraz ze standardowymi błędami oszacowania.}
   \label{fig:study4}
\end{figure*}

\subsubsection{Dyskusja}
Celem Badania 4 było sprawdzenie w jaki sposób aktywizacja perspektywy kolektywnego sprawcy (poprzez wprowadzenie informacji o zagrażeniu dla dobra grupy własnej) wpływnie na charakter zależności pomiędzy wymiarami sprawczości i wspólnotowości a kolektywną samooceną. Uzyskane wyniki wykazały wzrost znaczenia wymiaru kompetencji jako wyznacznika kolektywnej samooceny w sytuacji kolektywnego zagrożenia. Nie wykazano, aby zastosowane oddziaływanie wpłynęło na relację pomiędzy wymiarami wspólnotowymi a kolektywną samooceną.\\
Uzyskane wyniki wydają częściowo się potwierdzać rolę perspektywy sprawcy vs. biorcy w determinowaniu wyznaczników kolektywnej samooceny. Jeżeli treści związane z kolektywnym Ja aktywizują domyślnie perspektywę biorcy i przez to zmniejszają rolę sprawczości jako wyznacznika kolektywnej samooceny, to aktywizacja perspektywy kolektywnego sprawcy (poprzez przywołanie zagrażających informacji) zgodnie z modelem \textcite{wojciszke2011self} prowadzić będzie do wzrostu znaczenia sprawczości jako wyznacznika kolektywnej samooceny. Taką zależność zaobserwowano w niniejszym badaniu. Z drugiej strony aktywizacja kolektywnego sprawcy, zgodnie z proponowanym modelem, prowadzić powinna do spadku znaczenia wymiarów wspólnotowych, ale uzyskane dane zdają się nie przemawiać za taką zależnością: Wymiary wspólnotowe wyznaczały kolektywną samooceną zarówno w warunku kontrolnym jak i w warunku z zagrożeniem. Być może zastosowane oddziaływanie było zbyt krótkie i zbyt słabe, aby doprowadzić do dezaktywizacji treści wspólnotowych w myślenie o kolektywnym Ja. Możliwe jest jednak również, że w przypadku zagrożonego kolektywnego Ja dochodzi do równoczesnej aktywizacji dwóch perspektyw. Weryfikacja tej możliwości jest celem przyszłych badań. \\






\section{Ogólna dyskusja}
\subsection{Podsumowanie uzyskanych wyników}

Wysoka samoocena przy jednoczesnej wrogości i zimnie wobec innych osób może być przejawem samooceny defensywnej \parencite[zob.][]{raskin1991narcissistic}. Ponieważ jednak taka relacja była nieoczekiwana i nie można wykluczyć, że jest ona artefaktem metodologicznym jej interpretacja byłaby w tym miejscu przedwczesna.


Dla wariantu, w którym uczestnicy oceniali Polaków jako grupę nie zaobserwowano różnic pomiędzy warunkiem kontrolnym a warunkiem dotyczącym moralności. Możliwe, że ten brak różnic wynikał ze specyfiki treści obecnych w warunku kontrolnym -- dotyczyły one przekonań o charakterze społeczno-politycznym i jako takie mogły sprzyjać procesowi autoafirmacji \parencite{steele1988psychology}. Możliwe jest również, że myślenie o moralności grupy własnej ma charakter myślenia autoafirmacyjnego, niewątpliwie istotnego czynnika w buforowaniu myśli związanych ze śmiercią \parencite{pyszczynski2004people}. Niewielkie różnice pomiędzy warunkiem dotyczącym moralności i warunkiem kontrolnym mogą świadczyć o tym, że zadania w obu warunkach mogły spełniać podobne funkcje w ramach myślenia o grupie własnej. Z drugiej strony znaczne różnice pomiędzy warunkiem kontrolnym a warunkami kompetencji i ciepła w ramach pozostałych poziomów Ja, wydają się przemawiać za tezą, że proces autoafirmacji może być szczególnie skuteczny w ramach myślenia o kategoriach kolektywnych. Te hipotezy należałoby zweryfikować w przyszłych projektów badawczych.

Uzyskane dane wskazały, że wyznaczniki samooceny zależą od poziomu tożsamości. Nie można jednak pominąć dwóch wyników, które odróżniały wnioski z Badanie 2 od wniosków z Badania 1 i były niezgodne ze stawianymi w niniejszej rozprawie hipotezami. Pierwszy z nich, to nieoczekiwany związek wymiaru ciepła z indywidualną samooceną (rozumianą zarówna jako lubienie siebie jak i jako przekonania o własnej sprawczości). Drugi, to związek wymiaru kompetencji z przekonaniami o sprawczości Amerykanów jako grupy.\\

Odnosząc się do pierwszej niespójności, należy zaznaczyć, że miary użyte w Badaniach 1 i 2 dotyczyły ogólnych przekonań o sobie i o grupie własnej. W związku z tym można powiedzieć, że dotyczyły raczej względnie stałych cech i dyspozycji niż stanów. Trudno oczekiwać, aby cechy i dyspozycje były niezmienne na przekroju całej populacji. Wiadomo o istniejących różnicach indywidualnych w zakresie orientacji sprawczej i wspólnotowej \parencite{wojciszke2010skale}, a także, że związek wspólnotowości z samooceną (jako cechą) może być uzależniony od kultury, religijności, wieku i płci \parencite{gebauer2013agency}. Z drugiej strony model sprawcy i biorcy \parencite{wojciszke2011self} omawia dominację treści sprawczych w myśleniu o Ja jako efekt nastawienia specyficznego dla sytuacji implementacji działania \parencite{gollwitzer1997implementation}. Stąd można oczekiwać, że związek pomiędzy sprawczością i indywidualną samooceną oraz brak związku pomiędzy wspólnotowością a samooceną, będzie obserwowany w przypadku samooceny rozumianej raczej jako stan niż cecha. Sprawdzenie tej tezy jest trudne w ramach podejścia korelacyjnego i wymagałoby zastosowania schematu eksperymentalnego (ten problem starano się w rozwiązać w Badaniu 3).\\
Druga niespójność -- nieoczekiwana rola wymiaru kompetencji jako wyznacznika przekonań o kolektywnej sprawczości -- może być wytłumaczona bliskością treściową użytych skal. Niejasny jest jednak w takiej sytuacji brak związku pomiędzy wymiarem kompetencji a przekonaniami o sprawczości siebie jako członka grupy. Możliwe jest, że omawiany związek nie jest tylko artefaktem metody, ale że poprzez zadanie pytań dotyczących sprawczości adresowano inny niedomyślny obraz grupy. Jeżeli dominacja wspólnotowości jako wyznacznika kolektywnej samooceny wynika z domyślnego przyjmowania perspektywy biorcy w myśleniu o grupie (gdy grupa jest rodzajem \emph{bezpiecznej przystani}), to z pewnością istnieją sytuacje, w których członkowie grupy przyjmują również perspektywę kolektywnego sprawcy. Przekonania o sprawczości grupy zdają się dotyczyć właśniej takich sytuacji. W związku z tym możliwe, że eksperymentalne wzbudzenie perspektywy kolektywnego sprawcy zwiększy rolę wymiaru kompetencji jako wyznaczniki ogólnej kolektywnej samooceny (ten problem starano się rozwiązać w Badaniu 4).


\subsubsection{Do czego ludziom potrzebna jest kolektywna samoocena?}


\printbibliography

\end{document}
